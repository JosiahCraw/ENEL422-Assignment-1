\subsection{Real World Feasibility}

\subsubsection{Binary Polar Signalling}
Binary polar signalling is a very robust signalling method as is has a much higher noise tolerance. 
However, it has a much lower throughput compared to other options such as 4-PAM and 8-PAM. The noise
tolerance makes binary polar signalling suitable for wireless communications as this environment can have
much more noise over a much further distance.


\subsubsection{4-PAM}
4-PAM is also a reasonably robust signalling method as there is still some noise tolerance as the different levels
are reasonably well defined. This signaling method also allows double the data rate when compared to binary polar signalling.
These properties make this method well suited to wired communication as noise is less noticeable in the signal as the medium
can be controlled.

% 4-PAM is still usable in the real world, especially within a wired communication system. This is because the signal can be easily affected by noise as the amplitude
% spacing between `00' and `01' is smaller than the difference between binary polar signaling, assuming that the highest and lowest amplitudes are the same for both signals.
% However, 4-PAM provided greater data rates when compared to binary polar encoding as twice the number of bit can be sent in the same bandwidth.

\subsubsection{8-PAM}
8-PAM however, has comparatively poor noise performance, which could require significant error correction with large overhead which could negate the
improved throughput compared to other methods.

\subsubsection{Recommedation}
Overall, I would recommend using 4-PAM as it has a strong balance between data through put and noise tolerance. As 8-PAM has only 50\% more
throughput but half the separation between signals of 4-PAM. Overall the double throughput of 4-PAM to binary signalling is worth
the decreased noise resilience.

\subsection{Power Spectral Density of 4-PAM (Rectangular Pulse Shaping)}

With a rectangular pulse shaping the PSD of a 4-PAM signal is shown in (\ref{eqn:PSD-4-PAM-rect}) and the values for $S_x$ 
and $R_n$ can be found in (\ref{eqn:sx}) and (\ref{eqn:r0}, \ref{eqn:rn}) respectively.

\begin{equation}
    \label{eqn:rn}
    R_n=\lim_{N\to\infty} \frac{1}{N} \sum_{k}a_k a_{k-n}
\end{equation}

Assuming that the data is random:

\begin{equation}
    \label{eqn:ak-dist}
    P_{a_k} = 
    \begin{dcases}
        \frac{1}{4} \quad a_k = 3 \\
        \frac{1}{4} \quad a_k = 1 \\
        \frac{1}{4} \quad a_k = -1 \\
        \frac{1}{4} \quad a_k = -3
    \end{dcases}
    \quad \textrm{Therefore,} \quad R_n = 0
\end{equation}

As in the previous equation the probabilities are equal for all $a_k$,
\begin{equation}
    \label{eqn:r0}
    R_0 = \lim_{N\to\infty} \frac{1}{N} \sum_{k} {a_{k}}^2 \quad \textrm{With,} \quad P_{{a_{k}}^2} = 
    \begin{dcases}
        \frac{1}{2} \quad a_k^2 = 9 \\
        \frac{1}{2} \quad a_k^2 = 1
    \end{dcases}
\end{equation}

\newpage

Therefore,
\begin{equation}
    \label{eqn:r0-value}
    R_0 = 5
\end{equation}

\begin{equation}
    \label{eqn:sx}
     S_x(f) = \frac{1}{T_s}[R_0 + 2 \sum_{n=1}^{\infty}{R_n \cos 2 \pi n f T_s}]
\end{equation}

As,
\begin{equation}
    \label{eqn:ts}
    T_b = \frac{1}{R_b} \quad \textrm{Therefore with a bitrate of 1Mbps,} \quad T_b = \frac{1}{1 \times 10^6}
\end{equation}

As 4-PAM has two bits per pulse $T_s = 2 T_b$ so $T_s = 2 \mu$s Therefore,
\begin{equation}
    \label{eqn:sx-real}
    S_x(f) = \frac{1}{T_s} [5 + 0] \quad \textrm{For $T_s = 2\mu$s} \quad S_x=2500000
\end{equation}

The equation for PSD is,
\begin{equation}
    \label{eqn:PSD-4-PAM-rect}
    S_y(f) = \abs{\frac{T_s}{2} \textrm{sinc} ( \frac{f \pi T_s}{2} )}^2 S_x(f)
\end{equation}

So,
\begin{equation}
    \label{eqn:sy-value}
    S_y(f) = \abs{\frac{T_s}{2} \textrm{sinc}({\frac{f \pi T_s}{2}})}^2 (2500000)
\end{equation}
    
This results in the following PSD,
\begin{figure}[h]
    \begin{center}
        \begin{tikzpicture}
    \begin{axis}[
        axis lines = left,
        xlabel = $f$,
        ylabel = {$S_y(f)$},
    ]
    \addplot [
        domain=-0:500000000, 
        samples=2000, 
        color=red,
    ]
    {(abs(0.000001*(sin((x*pi*0.000002)/2)/((x*pi*0.000002)/2)))^2)*2500000)};
    \addlegendentry{$S_y$}

    \addplot [
        color=blue,
    ]
    coordinates {(50000000, 0.0000000005) (50000000, 0)};
    \addlegendentry{$B_f$}
    
    \end{axis}
\end{tikzpicture}
        \caption{PSD of a 4-PAM Signal with rectangular pulse shaping}
    \end{center}
\end{figure}

\subsection{Power Spectral Density of 4-PAM (Nyquist Pulse Shaping)}
    