\subsection{Real World Feasibility}

\subsubsection{Binary Polar Signalling}
Binary polar signalling is a very robust signalling method as is has a much higher noise tolerance. 
However, it has a much lower throughput compared to other options such as 4-PAM and 8-PAM. The noise
tolerance makes binary polar signalling suitable for wireless communications as this environment can have
much more noise over a much further distance.


\subsubsection{4-PAM}
4-PAM is also a reasonably robust signalling method as there is still some noise tolerance as the different levels
are reasonably well defined. This signaling method also allows double the data rate when compared to binary polar signalling.
These properties make this method well suited to wired communication as noise is less noticeable in the signal as the medium
can be controlled.

% 4-PAM is still usable in the real world, especially within a wired communication system. This is because the signal can be easily affected by noise as the amplitude
% spacing between `00' and `01' is smaller than the difference between binary polar signaling, assuming that the highest and lowest amplitudes are the same for both signals.
% However, 4-PAM provided greater data rates when compared to binary polar encoding as twice the number of bit can be sent in the same bandwidth.

\subsubsection{8-PAM}
8-PAM however, has comparatively poor noise performance, which could require significant error correction with large overhead which could negate the
improved throughput compared to other methods.

\subsubsection{Recommedation}
Overall, I would recommend using 4-PAM as it has a strong balance between data through put and noise tolerance. As 8-PAM has only 50\% more
throughput but half the separation between signals of 4-PAM. Overall the double throughput of 4-PAM to binary signalling is worth
the decreased noise resilience.

\subsection{Power Spectral Density of 4-PAM (Rectangular Pulse Shaping)}

With a rectangular pulse shaping the PSD of a 4-PAM signal is shown in () and the values for $S_x$ 
and $R_n$ can be found in x and x, x respectively.

\begin{equation}
    \label{eqn:rn}
    R_n=\lim_{N\to\infty} \frac{1}{N} \sum_{k}a_k a_{k-n}
\end{equation}

Assuming that the data is random:

\begin{equation}
    \label{eqn:ak-dist}
    P_{a_k} = 
    \begin{dcases}
        \frac{1}{4} \quad a_k = 3 \\
        \frac{1}{4} \quad a_k = 1 \\
        \frac{1}{4} \quad a_k = -1 \\
        \frac{1}{4} \quad a_k = -3
    \end{dcases}
    \quad \textrm{Therefore,} \quad R_n = 0
\end{equation}

As in the previous equation the probabilities are equal for all $a_k$,
\begin{equation}
    \label{eqn:r0}
    R_0 = \lim_{N\to\infty} \frac{1}{N} \sum_{k} {a_{k}}^2 \quad \textrm{With,} \quad P_{{a_{k}}^2} = 
    \begin{dcases}
        \frac{1}{2} \quad a_k^2 = 9 \\
        \frac{1}{2} \quad a_k^2 = 1
    \end{dcases}
\end{equation}

\newpage

Therefore,
\begin{equation}
    \label{eqn:r0-value}
    R_0 = 5
\end{equation}

\begin{equation}
    \label{eqn:sx}
     S_x(f) = \frac{1}{T_s}[R_0 + 2 \sum_{n=1}^{\infty}{R_n \cos 2 \pi n f T_s}]
\end{equation}

As,
\begin{equation}
    \label{eqn:ts}
    T_b = \frac{1}{R_b} \quad \textrm{Therefore with a bitrate of 1Mbps,} \quad T_b = \frac{1}{1 \times 10^6}
\end{equation}

As 4-PAM has two bits per pulse $T_s = 2 T_b$ so $T_s = 2 \mu$s Therefore,
\begin{equation}
    \label{eqn:sx-real}
    S_x(f) = \frac{1}{T_s} [5 + 0] \quad \textrm{For $T_s = 2\mu$s} \quad S_x=2500000
\end{equation}

The equation for PSD is,
\begin{equation}
    \label{eqn:PSD-4-PAM-rect}
    S_y(f) = \abs{\frac{T_s}{2} \textrm{sinc}( \frac{f \pi T_s}{2} )}^2 S_x(f)
\end{equation}

So,
\begin{equation}
    \label{eqn:sy-value}
    S_y(f) = \abs{\frac{T_s}{2} \textrm{sinc}({\frac{f \pi T_s}{2}})}^2 (2500000)
\end{equation}
    
This results in the following PSD,
\begin{figure}[h]
    \begin{center}
        \begin{tikzpicture}
    \begin{axis}[
        axis lines = left,
        xlabel = $f$,
        ylabel = {$S_y(f)$},
    ]
    \addplot [
        domain=-0:500000000, 
        samples=2000, 
        color=red,
    ]
    {(abs(0.000001*(sin((x*pi*0.000002)/2)/((x*pi*0.000002)/2)))^2)*2500000)};
    \addlegendentry{$S_y$}

    \addplot [
        color=blue,
    ]
    coordinates {(50000000, 0.0000000005) (50000000, 0)};
    \addlegendentry{$B_f$}
    
    \end{axis}
\end{tikzpicture}
        \caption{PSD of a 4-PAM Signal with rectangular pulse shaping}
    \end{center}
\end{figure}

\subsection{Power Spectral Density of 4-PAM (Nyquist Pulse Shaping)}
The for the design of a Niquist pulse a roll off factor of $\alpha = 1$ was used as there is adequite bandwidth
for the signal and the high roll off causes the tail oscillations of the pulse to fade more quickly, reducing the
chance of intersymbol interference (ISI)

The $S_x$ value is the same as above, using a root raised cosine pulse shaping the pulse is:

\begin{equation}
    \label{eqn:pf-w-niquist}
    P(f) =
    \begin{dcases}
        \frac{1}{4W} [1+\textrm{cos}(\frac{\pi f}{2W})] \quad \textrm{when} \quad 0<|f|<2W \\
        0 \quad \quad \textrm{when} |f| \geq 2W
    \end{dcases}
\end{equation}

Therefore $|P(f)|^2$ is,

\begin{equation}
    \label{eqn-pf2-nyquist}
    |P(f)|^2 =
    \begin{dcases}
        \frac{1}{16W^2}[1+\textrm{cos}(\frac{\pi f}{2W})]^2 \quad \textrm{when} \quad 0<|f|<2W \\
        0 \quad \textrm{when} \quad |f| \geq 2W
    \end{dcases}
\end{equation}

As $2W = \frac{1}{T_s}$,

\begin{equation}
    \label{eqn-pf2-nyquist-ts}
    |P(f)|^2 =
    \begin{dcases}
        \frac{T_s^2}{4}[1+\textrm{cos}(\pi f T_s)]^2 \quad \textrm{when} \quad 0<|f|<\frac{1}{T_s} \\
        0 \quad \textrm{when} \quad |f| \geq \frac{1}{T_s}
    \end{dcases}
\end{equation}

Therefore the power spectral density is,

\begin{equation}
    \label{eqn:psd-nyquist}
    S_y(f) = \frac{5T_s}{4}[1+\textrm{cos}(\pi f T_s)]^2
\end{equation}

This results in the following plot,

\begin{figure}[h]
    \begin{center}
        \begin{tikzpicture}
    \begin{axis}[
        axis lines = left,
        xlabel = $f$,
        ylabel = {$S_y$},
    ]
    \addplot [
        domain=-25000000:25000000,
        samples=2000,
        color=red,
    ]
    {(5*0.000002)/4*(1 + cos(pi*x*0.000002))^2};
    \addlegendentry{$S_y$}
    \end{axis}
\end{tikzpicture}

        \caption{The PSD of a RRCOS pulse shaped 4-PAM Signal}
    \end{center}
\end{figure}

\subsection{Pulse shaping a signal with a Nyquist pulse}

\begin{figure}[h]
    \begin{center}
        %% Creator: Matplotlib, PGF backend
%%
%% To include the figure in your LaTeX document, write
%%   \input{<filename>.pgf}
%%
%% Make sure the required packages are loaded in your preamble
%%   \usepackage{pgf}
%%
%% and, on pdftex
%%   \usepackage[utf8]{inputenc}\DeclareUnicodeCharacter{2212}{-}
%%
%% or, on luatex and xetex
%%   \usepackage{unicode-math}
%%
%% Figures using additional raster images can only be included by \input if
%% they are in the same directory as the main LaTeX file. For loading figures
%% from other directories you can use the `import` package
%%   \usepackage{import}
%%
%% and then include the figures with
%%   \import{<path to file>}{<filename>.pgf}
%%
%% Matplotlib used the following preamble
%%
\begingroup%
\makeatletter%
\begin{pgfpicture}%
\pgfpathrectangle{\pgfpointorigin}{\pgfqpoint{6.400000in}{4.800000in}}%
\pgfusepath{use as bounding box, clip}%
\begin{pgfscope}%
\pgfsetbuttcap%
\pgfsetmiterjoin%
\definecolor{currentfill}{rgb}{1.000000,1.000000,1.000000}%
\pgfsetfillcolor{currentfill}%
\pgfsetlinewidth{0.000000pt}%
\definecolor{currentstroke}{rgb}{1.000000,1.000000,1.000000}%
\pgfsetstrokecolor{currentstroke}%
\pgfsetdash{}{0pt}%
\pgfpathmoveto{\pgfqpoint{0.000000in}{0.000000in}}%
\pgfpathlineto{\pgfqpoint{6.400000in}{0.000000in}}%
\pgfpathlineto{\pgfqpoint{6.400000in}{4.800000in}}%
\pgfpathlineto{\pgfqpoint{0.000000in}{4.800000in}}%
\pgfpathclose%
\pgfusepath{fill}%
\end{pgfscope}%
\begin{pgfscope}%
\pgfsetbuttcap%
\pgfsetmiterjoin%
\definecolor{currentfill}{rgb}{1.000000,1.000000,1.000000}%
\pgfsetfillcolor{currentfill}%
\pgfsetlinewidth{0.000000pt}%
\definecolor{currentstroke}{rgb}{0.000000,0.000000,0.000000}%
\pgfsetstrokecolor{currentstroke}%
\pgfsetstrokeopacity{0.000000}%
\pgfsetdash{}{0pt}%
\pgfpathmoveto{\pgfqpoint{0.800000in}{0.528000in}}%
\pgfpathlineto{\pgfqpoint{5.760000in}{0.528000in}}%
\pgfpathlineto{\pgfqpoint{5.760000in}{4.224000in}}%
\pgfpathlineto{\pgfqpoint{0.800000in}{4.224000in}}%
\pgfpathclose%
\pgfusepath{fill}%
\end{pgfscope}%
\begin{pgfscope}%
\pgfsetbuttcap%
\pgfsetroundjoin%
\definecolor{currentfill}{rgb}{0.000000,0.000000,0.000000}%
\pgfsetfillcolor{currentfill}%
\pgfsetlinewidth{0.803000pt}%
\definecolor{currentstroke}{rgb}{0.000000,0.000000,0.000000}%
\pgfsetstrokecolor{currentstroke}%
\pgfsetdash{}{0pt}%
\pgfsys@defobject{currentmarker}{\pgfqpoint{0.000000in}{-0.048611in}}{\pgfqpoint{0.000000in}{0.000000in}}{%
\pgfpathmoveto{\pgfqpoint{0.000000in}{0.000000in}}%
\pgfpathlineto{\pgfqpoint{0.000000in}{-0.048611in}}%
\pgfusepath{stroke,fill}%
}%
\begin{pgfscope}%
\pgfsys@transformshift{1.025455in}{0.528000in}%
\pgfsys@useobject{currentmarker}{}%
\end{pgfscope}%
\end{pgfscope}%
\begin{pgfscope}%
\definecolor{textcolor}{rgb}{0.000000,0.000000,0.000000}%
\pgfsetstrokecolor{textcolor}%
\pgfsetfillcolor{textcolor}%
\pgftext[x=1.025455in,y=0.430778in,,top]{\color{textcolor}\rmfamily\fontsize{10.000000}{12.000000}\selectfont \(\displaystyle 0.0000\)}%
\end{pgfscope}%
\begin{pgfscope}%
\pgfsetbuttcap%
\pgfsetroundjoin%
\definecolor{currentfill}{rgb}{0.000000,0.000000,0.000000}%
\pgfsetfillcolor{currentfill}%
\pgfsetlinewidth{0.803000pt}%
\definecolor{currentstroke}{rgb}{0.000000,0.000000,0.000000}%
\pgfsetstrokecolor{currentstroke}%
\pgfsetdash{}{0pt}%
\pgfsys@defobject{currentmarker}{\pgfqpoint{0.000000in}{-0.048611in}}{\pgfqpoint{0.000000in}{0.000000in}}{%
\pgfpathmoveto{\pgfqpoint{0.000000in}{0.000000in}}%
\pgfpathlineto{\pgfqpoint{0.000000in}{-0.048611in}}%
\pgfusepath{stroke,fill}%
}%
\begin{pgfscope}%
\pgfsys@transformshift{1.596226in}{0.528000in}%
\pgfsys@useobject{currentmarker}{}%
\end{pgfscope}%
\end{pgfscope}%
\begin{pgfscope}%
\definecolor{textcolor}{rgb}{0.000000,0.000000,0.000000}%
\pgfsetstrokecolor{textcolor}%
\pgfsetfillcolor{textcolor}%
\pgftext[x=1.596226in,y=0.430778in,,top]{\color{textcolor}\rmfamily\fontsize{10.000000}{12.000000}\selectfont \(\displaystyle 0.0025\)}%
\end{pgfscope}%
\begin{pgfscope}%
\pgfsetbuttcap%
\pgfsetroundjoin%
\definecolor{currentfill}{rgb}{0.000000,0.000000,0.000000}%
\pgfsetfillcolor{currentfill}%
\pgfsetlinewidth{0.803000pt}%
\definecolor{currentstroke}{rgb}{0.000000,0.000000,0.000000}%
\pgfsetstrokecolor{currentstroke}%
\pgfsetdash{}{0pt}%
\pgfsys@defobject{currentmarker}{\pgfqpoint{0.000000in}{-0.048611in}}{\pgfqpoint{0.000000in}{0.000000in}}{%
\pgfpathmoveto{\pgfqpoint{0.000000in}{0.000000in}}%
\pgfpathlineto{\pgfqpoint{0.000000in}{-0.048611in}}%
\pgfusepath{stroke,fill}%
}%
\begin{pgfscope}%
\pgfsys@transformshift{2.166996in}{0.528000in}%
\pgfsys@useobject{currentmarker}{}%
\end{pgfscope}%
\end{pgfscope}%
\begin{pgfscope}%
\definecolor{textcolor}{rgb}{0.000000,0.000000,0.000000}%
\pgfsetstrokecolor{textcolor}%
\pgfsetfillcolor{textcolor}%
\pgftext[x=2.166996in,y=0.430778in,,top]{\color{textcolor}\rmfamily\fontsize{10.000000}{12.000000}\selectfont \(\displaystyle 0.0050\)}%
\end{pgfscope}%
\begin{pgfscope}%
\pgfsetbuttcap%
\pgfsetroundjoin%
\definecolor{currentfill}{rgb}{0.000000,0.000000,0.000000}%
\pgfsetfillcolor{currentfill}%
\pgfsetlinewidth{0.803000pt}%
\definecolor{currentstroke}{rgb}{0.000000,0.000000,0.000000}%
\pgfsetstrokecolor{currentstroke}%
\pgfsetdash{}{0pt}%
\pgfsys@defobject{currentmarker}{\pgfqpoint{0.000000in}{-0.048611in}}{\pgfqpoint{0.000000in}{0.000000in}}{%
\pgfpathmoveto{\pgfqpoint{0.000000in}{0.000000in}}%
\pgfpathlineto{\pgfqpoint{0.000000in}{-0.048611in}}%
\pgfusepath{stroke,fill}%
}%
\begin{pgfscope}%
\pgfsys@transformshift{2.737767in}{0.528000in}%
\pgfsys@useobject{currentmarker}{}%
\end{pgfscope}%
\end{pgfscope}%
\begin{pgfscope}%
\definecolor{textcolor}{rgb}{0.000000,0.000000,0.000000}%
\pgfsetstrokecolor{textcolor}%
\pgfsetfillcolor{textcolor}%
\pgftext[x=2.737767in,y=0.430778in,,top]{\color{textcolor}\rmfamily\fontsize{10.000000}{12.000000}\selectfont \(\displaystyle 0.0075\)}%
\end{pgfscope}%
\begin{pgfscope}%
\pgfsetbuttcap%
\pgfsetroundjoin%
\definecolor{currentfill}{rgb}{0.000000,0.000000,0.000000}%
\pgfsetfillcolor{currentfill}%
\pgfsetlinewidth{0.803000pt}%
\definecolor{currentstroke}{rgb}{0.000000,0.000000,0.000000}%
\pgfsetstrokecolor{currentstroke}%
\pgfsetdash{}{0pt}%
\pgfsys@defobject{currentmarker}{\pgfqpoint{0.000000in}{-0.048611in}}{\pgfqpoint{0.000000in}{0.000000in}}{%
\pgfpathmoveto{\pgfqpoint{0.000000in}{0.000000in}}%
\pgfpathlineto{\pgfqpoint{0.000000in}{-0.048611in}}%
\pgfusepath{stroke,fill}%
}%
\begin{pgfscope}%
\pgfsys@transformshift{3.308538in}{0.528000in}%
\pgfsys@useobject{currentmarker}{}%
\end{pgfscope}%
\end{pgfscope}%
\begin{pgfscope}%
\definecolor{textcolor}{rgb}{0.000000,0.000000,0.000000}%
\pgfsetstrokecolor{textcolor}%
\pgfsetfillcolor{textcolor}%
\pgftext[x=3.308538in,y=0.430778in,,top]{\color{textcolor}\rmfamily\fontsize{10.000000}{12.000000}\selectfont \(\displaystyle 0.0100\)}%
\end{pgfscope}%
\begin{pgfscope}%
\pgfsetbuttcap%
\pgfsetroundjoin%
\definecolor{currentfill}{rgb}{0.000000,0.000000,0.000000}%
\pgfsetfillcolor{currentfill}%
\pgfsetlinewidth{0.803000pt}%
\definecolor{currentstroke}{rgb}{0.000000,0.000000,0.000000}%
\pgfsetstrokecolor{currentstroke}%
\pgfsetdash{}{0pt}%
\pgfsys@defobject{currentmarker}{\pgfqpoint{0.000000in}{-0.048611in}}{\pgfqpoint{0.000000in}{0.000000in}}{%
\pgfpathmoveto{\pgfqpoint{0.000000in}{0.000000in}}%
\pgfpathlineto{\pgfqpoint{0.000000in}{-0.048611in}}%
\pgfusepath{stroke,fill}%
}%
\begin{pgfscope}%
\pgfsys@transformshift{3.879309in}{0.528000in}%
\pgfsys@useobject{currentmarker}{}%
\end{pgfscope}%
\end{pgfscope}%
\begin{pgfscope}%
\definecolor{textcolor}{rgb}{0.000000,0.000000,0.000000}%
\pgfsetstrokecolor{textcolor}%
\pgfsetfillcolor{textcolor}%
\pgftext[x=3.879309in,y=0.430778in,,top]{\color{textcolor}\rmfamily\fontsize{10.000000}{12.000000}\selectfont \(\displaystyle 0.0125\)}%
\end{pgfscope}%
\begin{pgfscope}%
\pgfsetbuttcap%
\pgfsetroundjoin%
\definecolor{currentfill}{rgb}{0.000000,0.000000,0.000000}%
\pgfsetfillcolor{currentfill}%
\pgfsetlinewidth{0.803000pt}%
\definecolor{currentstroke}{rgb}{0.000000,0.000000,0.000000}%
\pgfsetstrokecolor{currentstroke}%
\pgfsetdash{}{0pt}%
\pgfsys@defobject{currentmarker}{\pgfqpoint{0.000000in}{-0.048611in}}{\pgfqpoint{0.000000in}{0.000000in}}{%
\pgfpathmoveto{\pgfqpoint{0.000000in}{0.000000in}}%
\pgfpathlineto{\pgfqpoint{0.000000in}{-0.048611in}}%
\pgfusepath{stroke,fill}%
}%
\begin{pgfscope}%
\pgfsys@transformshift{4.450080in}{0.528000in}%
\pgfsys@useobject{currentmarker}{}%
\end{pgfscope}%
\end{pgfscope}%
\begin{pgfscope}%
\definecolor{textcolor}{rgb}{0.000000,0.000000,0.000000}%
\pgfsetstrokecolor{textcolor}%
\pgfsetfillcolor{textcolor}%
\pgftext[x=4.450080in,y=0.430778in,,top]{\color{textcolor}\rmfamily\fontsize{10.000000}{12.000000}\selectfont \(\displaystyle 0.0150\)}%
\end{pgfscope}%
\begin{pgfscope}%
\pgfsetbuttcap%
\pgfsetroundjoin%
\definecolor{currentfill}{rgb}{0.000000,0.000000,0.000000}%
\pgfsetfillcolor{currentfill}%
\pgfsetlinewidth{0.803000pt}%
\definecolor{currentstroke}{rgb}{0.000000,0.000000,0.000000}%
\pgfsetstrokecolor{currentstroke}%
\pgfsetdash{}{0pt}%
\pgfsys@defobject{currentmarker}{\pgfqpoint{0.000000in}{-0.048611in}}{\pgfqpoint{0.000000in}{0.000000in}}{%
\pgfpathmoveto{\pgfqpoint{0.000000in}{0.000000in}}%
\pgfpathlineto{\pgfqpoint{0.000000in}{-0.048611in}}%
\pgfusepath{stroke,fill}%
}%
\begin{pgfscope}%
\pgfsys@transformshift{5.020851in}{0.528000in}%
\pgfsys@useobject{currentmarker}{}%
\end{pgfscope}%
\end{pgfscope}%
\begin{pgfscope}%
\definecolor{textcolor}{rgb}{0.000000,0.000000,0.000000}%
\pgfsetstrokecolor{textcolor}%
\pgfsetfillcolor{textcolor}%
\pgftext[x=5.020851in,y=0.430778in,,top]{\color{textcolor}\rmfamily\fontsize{10.000000}{12.000000}\selectfont \(\displaystyle 0.0175\)}%
\end{pgfscope}%
\begin{pgfscope}%
\pgfsetbuttcap%
\pgfsetroundjoin%
\definecolor{currentfill}{rgb}{0.000000,0.000000,0.000000}%
\pgfsetfillcolor{currentfill}%
\pgfsetlinewidth{0.803000pt}%
\definecolor{currentstroke}{rgb}{0.000000,0.000000,0.000000}%
\pgfsetstrokecolor{currentstroke}%
\pgfsetdash{}{0pt}%
\pgfsys@defobject{currentmarker}{\pgfqpoint{0.000000in}{-0.048611in}}{\pgfqpoint{0.000000in}{0.000000in}}{%
\pgfpathmoveto{\pgfqpoint{0.000000in}{0.000000in}}%
\pgfpathlineto{\pgfqpoint{0.000000in}{-0.048611in}}%
\pgfusepath{stroke,fill}%
}%
\begin{pgfscope}%
\pgfsys@transformshift{5.591622in}{0.528000in}%
\pgfsys@useobject{currentmarker}{}%
\end{pgfscope}%
\end{pgfscope}%
\begin{pgfscope}%
\definecolor{textcolor}{rgb}{0.000000,0.000000,0.000000}%
\pgfsetstrokecolor{textcolor}%
\pgfsetfillcolor{textcolor}%
\pgftext[x=5.591622in,y=0.430778in,,top]{\color{textcolor}\rmfamily\fontsize{10.000000}{12.000000}\selectfont \(\displaystyle 0.0200\)}%
\end{pgfscope}%
\begin{pgfscope}%
\definecolor{textcolor}{rgb}{0.000000,0.000000,0.000000}%
\pgfsetstrokecolor{textcolor}%
\pgfsetfillcolor{textcolor}%
\pgftext[x=3.280000in,y=0.251766in,,top]{\color{textcolor}\rmfamily\fontsize{10.000000}{12.000000}\selectfont Time (s)}%
\end{pgfscope}%
\begin{pgfscope}%
\pgfsetbuttcap%
\pgfsetroundjoin%
\definecolor{currentfill}{rgb}{0.000000,0.000000,0.000000}%
\pgfsetfillcolor{currentfill}%
\pgfsetlinewidth{0.803000pt}%
\definecolor{currentstroke}{rgb}{0.000000,0.000000,0.000000}%
\pgfsetstrokecolor{currentstroke}%
\pgfsetdash{}{0pt}%
\pgfsys@defobject{currentmarker}{\pgfqpoint{-0.048611in}{0.000000in}}{\pgfqpoint{0.000000in}{0.000000in}}{%
\pgfpathmoveto{\pgfqpoint{0.000000in}{0.000000in}}%
\pgfpathlineto{\pgfqpoint{-0.048611in}{0.000000in}}%
\pgfusepath{stroke,fill}%
}%
\begin{pgfscope}%
\pgfsys@transformshift{0.800000in}{0.696000in}%
\pgfsys@useobject{currentmarker}{}%
\end{pgfscope}%
\end{pgfscope}%
\begin{pgfscope}%
\definecolor{textcolor}{rgb}{0.000000,0.000000,0.000000}%
\pgfsetstrokecolor{textcolor}%
\pgfsetfillcolor{textcolor}%
\pgftext[x=0.525308in, y=0.647775in, left, base]{\color{textcolor}\rmfamily\fontsize{10.000000}{12.000000}\selectfont \(\displaystyle -3\)}%
\end{pgfscope}%
\begin{pgfscope}%
\pgfsetbuttcap%
\pgfsetroundjoin%
\definecolor{currentfill}{rgb}{0.000000,0.000000,0.000000}%
\pgfsetfillcolor{currentfill}%
\pgfsetlinewidth{0.803000pt}%
\definecolor{currentstroke}{rgb}{0.000000,0.000000,0.000000}%
\pgfsetstrokecolor{currentstroke}%
\pgfsetdash{}{0pt}%
\pgfsys@defobject{currentmarker}{\pgfqpoint{-0.048611in}{0.000000in}}{\pgfqpoint{0.000000in}{0.000000in}}{%
\pgfpathmoveto{\pgfqpoint{0.000000in}{0.000000in}}%
\pgfpathlineto{\pgfqpoint{-0.048611in}{0.000000in}}%
\pgfusepath{stroke,fill}%
}%
\begin{pgfscope}%
\pgfsys@transformshift{0.800000in}{1.256000in}%
\pgfsys@useobject{currentmarker}{}%
\end{pgfscope}%
\end{pgfscope}%
\begin{pgfscope}%
\definecolor{textcolor}{rgb}{0.000000,0.000000,0.000000}%
\pgfsetstrokecolor{textcolor}%
\pgfsetfillcolor{textcolor}%
\pgftext[x=0.525308in, y=1.207775in, left, base]{\color{textcolor}\rmfamily\fontsize{10.000000}{12.000000}\selectfont \(\displaystyle -2\)}%
\end{pgfscope}%
\begin{pgfscope}%
\pgfsetbuttcap%
\pgfsetroundjoin%
\definecolor{currentfill}{rgb}{0.000000,0.000000,0.000000}%
\pgfsetfillcolor{currentfill}%
\pgfsetlinewidth{0.803000pt}%
\definecolor{currentstroke}{rgb}{0.000000,0.000000,0.000000}%
\pgfsetstrokecolor{currentstroke}%
\pgfsetdash{}{0pt}%
\pgfsys@defobject{currentmarker}{\pgfqpoint{-0.048611in}{0.000000in}}{\pgfqpoint{0.000000in}{0.000000in}}{%
\pgfpathmoveto{\pgfqpoint{0.000000in}{0.000000in}}%
\pgfpathlineto{\pgfqpoint{-0.048611in}{0.000000in}}%
\pgfusepath{stroke,fill}%
}%
\begin{pgfscope}%
\pgfsys@transformshift{0.800000in}{1.816000in}%
\pgfsys@useobject{currentmarker}{}%
\end{pgfscope}%
\end{pgfscope}%
\begin{pgfscope}%
\definecolor{textcolor}{rgb}{0.000000,0.000000,0.000000}%
\pgfsetstrokecolor{textcolor}%
\pgfsetfillcolor{textcolor}%
\pgftext[x=0.525308in, y=1.767775in, left, base]{\color{textcolor}\rmfamily\fontsize{10.000000}{12.000000}\selectfont \(\displaystyle -1\)}%
\end{pgfscope}%
\begin{pgfscope}%
\pgfsetbuttcap%
\pgfsetroundjoin%
\definecolor{currentfill}{rgb}{0.000000,0.000000,0.000000}%
\pgfsetfillcolor{currentfill}%
\pgfsetlinewidth{0.803000pt}%
\definecolor{currentstroke}{rgb}{0.000000,0.000000,0.000000}%
\pgfsetstrokecolor{currentstroke}%
\pgfsetdash{}{0pt}%
\pgfsys@defobject{currentmarker}{\pgfqpoint{-0.048611in}{0.000000in}}{\pgfqpoint{0.000000in}{0.000000in}}{%
\pgfpathmoveto{\pgfqpoint{0.000000in}{0.000000in}}%
\pgfpathlineto{\pgfqpoint{-0.048611in}{0.000000in}}%
\pgfusepath{stroke,fill}%
}%
\begin{pgfscope}%
\pgfsys@transformshift{0.800000in}{2.376000in}%
\pgfsys@useobject{currentmarker}{}%
\end{pgfscope}%
\end{pgfscope}%
\begin{pgfscope}%
\definecolor{textcolor}{rgb}{0.000000,0.000000,0.000000}%
\pgfsetstrokecolor{textcolor}%
\pgfsetfillcolor{textcolor}%
\pgftext[x=0.633333in, y=2.327775in, left, base]{\color{textcolor}\rmfamily\fontsize{10.000000}{12.000000}\selectfont \(\displaystyle 0\)}%
\end{pgfscope}%
\begin{pgfscope}%
\pgfsetbuttcap%
\pgfsetroundjoin%
\definecolor{currentfill}{rgb}{0.000000,0.000000,0.000000}%
\pgfsetfillcolor{currentfill}%
\pgfsetlinewidth{0.803000pt}%
\definecolor{currentstroke}{rgb}{0.000000,0.000000,0.000000}%
\pgfsetstrokecolor{currentstroke}%
\pgfsetdash{}{0pt}%
\pgfsys@defobject{currentmarker}{\pgfqpoint{-0.048611in}{0.000000in}}{\pgfqpoint{0.000000in}{0.000000in}}{%
\pgfpathmoveto{\pgfqpoint{0.000000in}{0.000000in}}%
\pgfpathlineto{\pgfqpoint{-0.048611in}{0.000000in}}%
\pgfusepath{stroke,fill}%
}%
\begin{pgfscope}%
\pgfsys@transformshift{0.800000in}{2.936000in}%
\pgfsys@useobject{currentmarker}{}%
\end{pgfscope}%
\end{pgfscope}%
\begin{pgfscope}%
\definecolor{textcolor}{rgb}{0.000000,0.000000,0.000000}%
\pgfsetstrokecolor{textcolor}%
\pgfsetfillcolor{textcolor}%
\pgftext[x=0.633333in, y=2.887775in, left, base]{\color{textcolor}\rmfamily\fontsize{10.000000}{12.000000}\selectfont \(\displaystyle 1\)}%
\end{pgfscope}%
\begin{pgfscope}%
\pgfsetbuttcap%
\pgfsetroundjoin%
\definecolor{currentfill}{rgb}{0.000000,0.000000,0.000000}%
\pgfsetfillcolor{currentfill}%
\pgfsetlinewidth{0.803000pt}%
\definecolor{currentstroke}{rgb}{0.000000,0.000000,0.000000}%
\pgfsetstrokecolor{currentstroke}%
\pgfsetdash{}{0pt}%
\pgfsys@defobject{currentmarker}{\pgfqpoint{-0.048611in}{0.000000in}}{\pgfqpoint{0.000000in}{0.000000in}}{%
\pgfpathmoveto{\pgfqpoint{0.000000in}{0.000000in}}%
\pgfpathlineto{\pgfqpoint{-0.048611in}{0.000000in}}%
\pgfusepath{stroke,fill}%
}%
\begin{pgfscope}%
\pgfsys@transformshift{0.800000in}{3.496000in}%
\pgfsys@useobject{currentmarker}{}%
\end{pgfscope}%
\end{pgfscope}%
\begin{pgfscope}%
\definecolor{textcolor}{rgb}{0.000000,0.000000,0.000000}%
\pgfsetstrokecolor{textcolor}%
\pgfsetfillcolor{textcolor}%
\pgftext[x=0.633333in, y=3.447775in, left, base]{\color{textcolor}\rmfamily\fontsize{10.000000}{12.000000}\selectfont \(\displaystyle 2\)}%
\end{pgfscope}%
\begin{pgfscope}%
\pgfsetbuttcap%
\pgfsetroundjoin%
\definecolor{currentfill}{rgb}{0.000000,0.000000,0.000000}%
\pgfsetfillcolor{currentfill}%
\pgfsetlinewidth{0.803000pt}%
\definecolor{currentstroke}{rgb}{0.000000,0.000000,0.000000}%
\pgfsetstrokecolor{currentstroke}%
\pgfsetdash{}{0pt}%
\pgfsys@defobject{currentmarker}{\pgfqpoint{-0.048611in}{0.000000in}}{\pgfqpoint{0.000000in}{0.000000in}}{%
\pgfpathmoveto{\pgfqpoint{0.000000in}{0.000000in}}%
\pgfpathlineto{\pgfqpoint{-0.048611in}{0.000000in}}%
\pgfusepath{stroke,fill}%
}%
\begin{pgfscope}%
\pgfsys@transformshift{0.800000in}{4.056000in}%
\pgfsys@useobject{currentmarker}{}%
\end{pgfscope}%
\end{pgfscope}%
\begin{pgfscope}%
\definecolor{textcolor}{rgb}{0.000000,0.000000,0.000000}%
\pgfsetstrokecolor{textcolor}%
\pgfsetfillcolor{textcolor}%
\pgftext[x=0.633333in, y=4.007775in, left, base]{\color{textcolor}\rmfamily\fontsize{10.000000}{12.000000}\selectfont \(\displaystyle 3\)}%
\end{pgfscope}%
\begin{pgfscope}%
\definecolor{textcolor}{rgb}{0.000000,0.000000,0.000000}%
\pgfsetstrokecolor{textcolor}%
\pgfsetfillcolor{textcolor}%
\pgftext[x=0.469752in,y=2.376000in,,bottom,rotate=90.000000]{\color{textcolor}\rmfamily\fontsize{10.000000}{12.000000}\selectfont Amplitude}%
\end{pgfscope}%
\begin{pgfscope}%
\pgfpathrectangle{\pgfqpoint{0.800000in}{0.528000in}}{\pgfqpoint{4.960000in}{3.696000in}}%
\pgfusepath{clip}%
\pgfsetrectcap%
\pgfsetroundjoin%
\pgfsetlinewidth{1.505625pt}%
\definecolor{currentstroke}{rgb}{0.121569,0.466667,0.705882}%
\pgfsetstrokecolor{currentstroke}%
\pgfsetdash{}{0pt}%
\pgfpathmoveto{\pgfqpoint{1.025455in}{2.377844in}}%
\pgfpathlineto{\pgfqpoint{1.082532in}{2.381239in}}%
\pgfpathlineto{\pgfqpoint{1.139609in}{2.382225in}}%
\pgfpathlineto{\pgfqpoint{1.196686in}{2.379098in}}%
\pgfpathlineto{\pgfqpoint{1.253763in}{2.372232in}}%
\pgfpathlineto{\pgfqpoint{1.310840in}{2.364774in}}%
\pgfpathlineto{\pgfqpoint{1.367917in}{2.361912in}}%
\pgfpathlineto{\pgfqpoint{1.424994in}{2.368529in}}%
\pgfpathlineto{\pgfqpoint{1.482071in}{2.385182in}}%
\pgfpathlineto{\pgfqpoint{1.539148in}{2.406205in}}%
\pgfpathlineto{\pgfqpoint{1.596226in}{2.418751in}}%
\pgfpathlineto{\pgfqpoint{1.653303in}{2.402360in}}%
\pgfpathlineto{\pgfqpoint{1.710380in}{2.333870in}}%
\pgfpathlineto{\pgfqpoint{1.767457in}{2.194426in}}%
\pgfpathlineto{\pgfqpoint{1.824534in}{1.977258in}}%
\pgfpathlineto{\pgfqpoint{1.881611in}{1.693668in}}%
\pgfpathlineto{\pgfqpoint{1.938688in}{1.373007in}}%
\pgfpathlineto{\pgfqpoint{1.995765in}{1.062901in}}%
\pgfpathlineto{\pgfqpoint{2.052842in}{0.821767in}}%
\pgfpathlineto{\pgfqpoint{2.109919in}{0.700633in}}%
\pgfpathlineto{\pgfqpoint{2.166996in}{0.731091in}}%
\pgfpathlineto{\pgfqpoint{2.224074in}{0.915932in}}%
\pgfpathlineto{\pgfqpoint{2.281151in}{1.227358in}}%
\pgfpathlineto{\pgfqpoint{2.338228in}{1.613862in}}%
\pgfpathlineto{\pgfqpoint{2.395305in}{2.014624in}}%
\pgfpathlineto{\pgfqpoint{2.452382in}{2.374760in}}%
\pgfpathlineto{\pgfqpoint{2.509459in}{2.658955in}}%
\pgfpathlineto{\pgfqpoint{2.566536in}{2.860633in}}%
\pgfpathlineto{\pgfqpoint{2.623613in}{2.999903in}}%
\pgfpathlineto{\pgfqpoint{2.680690in}{3.113422in}}%
\pgfpathlineto{\pgfqpoint{2.737767in}{3.239096in}}%
\pgfpathlineto{\pgfqpoint{2.794845in}{3.400756in}}%
\pgfpathlineto{\pgfqpoint{2.851922in}{3.596108in}}%
\pgfpathlineto{\pgfqpoint{2.908999in}{3.799012in}}%
\pgfpathlineto{\pgfqpoint{2.966076in}{3.968329in}}%
\pgfpathlineto{\pgfqpoint{3.023153in}{4.055866in}}%
\pgfpathlineto{\pgfqpoint{3.080230in}{4.022411in}}%
\pgfpathlineto{\pgfqpoint{3.137307in}{3.849803in}}%
\pgfpathlineto{\pgfqpoint{3.194384in}{3.547407in}}%
\pgfpathlineto{\pgfqpoint{3.251461in}{3.151658in}}%
\pgfpathlineto{\pgfqpoint{3.308538in}{2.720887in}}%
\pgfpathlineto{\pgfqpoint{3.365616in}{2.319780in}}%
\pgfpathlineto{\pgfqpoint{3.422693in}{2.009209in}}%
\pgfpathlineto{\pgfqpoint{3.479770in}{1.838870in}}%
\pgfpathlineto{\pgfqpoint{3.536847in}{1.836581in}}%
\pgfpathlineto{\pgfqpoint{3.593924in}{2.003392in}}%
\pgfpathlineto{\pgfqpoint{3.651001in}{2.313338in}}%
\pgfpathlineto{\pgfqpoint{3.708078in}{2.717783in}}%
\pgfpathlineto{\pgfqpoint{3.765155in}{3.153080in}}%
\pgfpathlineto{\pgfqpoint{3.822232in}{3.552818in}}%
\pgfpathlineto{\pgfqpoint{3.879309in}{3.858520in}}%
\pgfpathlineto{\pgfqpoint{3.936386in}{4.028445in}}%
\pgfpathlineto{\pgfqpoint{3.993464in}{4.045172in}}%
\pgfpathlineto{\pgfqpoint{4.050541in}{3.917783in}}%
\pgfpathlineto{\pgfqpoint{4.107618in}{3.678621in}}%
\pgfpathlineto{\pgfqpoint{4.164695in}{3.375305in}}%
\pgfpathlineto{\pgfqpoint{4.221772in}{3.058332in}}%
\pgfpathlineto{\pgfqpoint{4.278849in}{2.774742in}}%
\pgfpathlineto{\pgfqpoint{4.335926in}{2.557574in}}%
\pgfpathlineto{\pgfqpoint{4.393003in}{2.418130in}}%
\pgfpathlineto{\pgfqpoint{4.450080in}{2.349640in}}%
\pgfpathlineto{\pgfqpoint{4.507158in}{2.333249in}}%
\pgfpathlineto{\pgfqpoint{4.564235in}{2.345795in}}%
\pgfpathlineto{\pgfqpoint{4.621312in}{2.366818in}}%
\pgfpathlineto{\pgfqpoint{4.678389in}{2.383471in}}%
\pgfpathlineto{\pgfqpoint{4.735466in}{2.390088in}}%
\pgfpathlineto{\pgfqpoint{4.792543in}{2.387226in}}%
\pgfpathlineto{\pgfqpoint{4.849620in}{2.379768in}}%
\pgfpathlineto{\pgfqpoint{4.906697in}{2.372902in}}%
\pgfpathlineto{\pgfqpoint{4.963774in}{2.369775in}}%
\pgfpathlineto{\pgfqpoint{5.020851in}{2.370761in}}%
\pgfpathlineto{\pgfqpoint{5.077929in}{2.374156in}}%
\pgfpathlineto{\pgfqpoint{5.135005in}{2.376000in}}%
\pgfpathlineto{\pgfqpoint{5.192083in}{2.376000in}}%
\pgfpathlineto{\pgfqpoint{5.249160in}{2.376000in}}%
\pgfpathlineto{\pgfqpoint{5.306237in}{2.376000in}}%
\pgfpathlineto{\pgfqpoint{5.363314in}{2.376000in}}%
\pgfpathlineto{\pgfqpoint{5.420391in}{2.376000in}}%
\pgfpathlineto{\pgfqpoint{5.477468in}{2.376000in}}%
\pgfpathlineto{\pgfqpoint{5.534545in}{2.376000in}}%
\pgfusepath{stroke}%
\end{pgfscope}%
\begin{pgfscope}%
\pgfpathrectangle{\pgfqpoint{0.800000in}{0.528000in}}{\pgfqpoint{4.960000in}{3.696000in}}%
\pgfusepath{clip}%
\pgfsetrectcap%
\pgfsetroundjoin%
\pgfsetlinewidth{1.505625pt}%
\definecolor{currentstroke}{rgb}{0.000000,0.000000,1.000000}%
\pgfsetstrokecolor{currentstroke}%
\pgfsetdash{}{0pt}%
\pgfpathmoveto{\pgfqpoint{1.025455in}{2.377844in}}%
\pgfpathlineto{\pgfqpoint{1.082532in}{2.381239in}}%
\pgfpathlineto{\pgfqpoint{1.139609in}{2.382225in}}%
\pgfpathlineto{\pgfqpoint{1.196686in}{2.379098in}}%
\pgfpathlineto{\pgfqpoint{1.253763in}{2.372232in}}%
\pgfpathlineto{\pgfqpoint{1.310840in}{2.364774in}}%
\pgfpathlineto{\pgfqpoint{1.367917in}{2.361912in}}%
\pgfpathlineto{\pgfqpoint{1.424994in}{2.368529in}}%
\pgfpathlineto{\pgfqpoint{1.482071in}{2.385182in}}%
\pgfpathlineto{\pgfqpoint{1.539148in}{2.406205in}}%
\pgfpathlineto{\pgfqpoint{1.596226in}{2.418751in}}%
\pgfpathlineto{\pgfqpoint{1.653303in}{2.402360in}}%
\pgfpathlineto{\pgfqpoint{1.710380in}{2.333870in}}%
\pgfpathlineto{\pgfqpoint{1.767457in}{2.194426in}}%
\pgfpathlineto{\pgfqpoint{1.824534in}{1.977258in}}%
\pgfpathlineto{\pgfqpoint{1.881611in}{1.693668in}}%
\pgfpathlineto{\pgfqpoint{1.938688in}{1.373007in}}%
\pgfpathlineto{\pgfqpoint{1.995765in}{1.062901in}}%
\pgfpathlineto{\pgfqpoint{2.052842in}{0.821767in}}%
\pgfpathlineto{\pgfqpoint{2.109919in}{0.700633in}}%
\pgfpathlineto{\pgfqpoint{2.166996in}{0.731091in}}%
\pgfpathlineto{\pgfqpoint{2.224074in}{0.915932in}}%
\pgfpathlineto{\pgfqpoint{2.281151in}{1.227358in}}%
\pgfpathlineto{\pgfqpoint{2.338228in}{1.613862in}}%
\pgfpathlineto{\pgfqpoint{2.395305in}{2.014624in}}%
\pgfpathlineto{\pgfqpoint{2.452382in}{2.374760in}}%
\pgfpathlineto{\pgfqpoint{2.509459in}{2.658955in}}%
\pgfpathlineto{\pgfqpoint{2.566536in}{2.860633in}}%
\pgfpathlineto{\pgfqpoint{2.623613in}{2.999903in}}%
\pgfpathlineto{\pgfqpoint{2.680690in}{3.113422in}}%
\pgfpathlineto{\pgfqpoint{2.737767in}{3.239096in}}%
\pgfpathlineto{\pgfqpoint{2.794845in}{3.400756in}}%
\pgfpathlineto{\pgfqpoint{2.851922in}{3.596108in}}%
\pgfpathlineto{\pgfqpoint{2.908999in}{3.799012in}}%
\pgfpathlineto{\pgfqpoint{2.966076in}{3.968329in}}%
\pgfpathlineto{\pgfqpoint{3.023153in}{4.055866in}}%
\pgfpathlineto{\pgfqpoint{3.080230in}{4.022411in}}%
\pgfpathlineto{\pgfqpoint{3.137307in}{3.849803in}}%
\pgfpathlineto{\pgfqpoint{3.194384in}{3.547407in}}%
\pgfpathlineto{\pgfqpoint{3.251461in}{3.151658in}}%
\pgfpathlineto{\pgfqpoint{3.308538in}{2.720887in}}%
\pgfpathlineto{\pgfqpoint{3.365616in}{2.319780in}}%
\pgfpathlineto{\pgfqpoint{3.422693in}{2.009209in}}%
\pgfpathlineto{\pgfqpoint{3.479770in}{1.838870in}}%
\pgfpathlineto{\pgfqpoint{3.536847in}{1.836581in}}%
\pgfpathlineto{\pgfqpoint{3.593924in}{2.003392in}}%
\pgfpathlineto{\pgfqpoint{3.651001in}{2.313338in}}%
\pgfpathlineto{\pgfqpoint{3.708078in}{2.717783in}}%
\pgfpathlineto{\pgfqpoint{3.765155in}{3.153080in}}%
\pgfpathlineto{\pgfqpoint{3.822232in}{3.552818in}}%
\pgfpathlineto{\pgfqpoint{3.879309in}{3.858520in}}%
\pgfpathlineto{\pgfqpoint{3.936386in}{4.028445in}}%
\pgfpathlineto{\pgfqpoint{3.993464in}{4.045172in}}%
\pgfpathlineto{\pgfqpoint{4.050541in}{3.917783in}}%
\pgfpathlineto{\pgfqpoint{4.107618in}{3.678621in}}%
\pgfpathlineto{\pgfqpoint{4.164695in}{3.375305in}}%
\pgfpathlineto{\pgfqpoint{4.221772in}{3.058332in}}%
\pgfpathlineto{\pgfqpoint{4.278849in}{2.774742in}}%
\pgfpathlineto{\pgfqpoint{4.335926in}{2.557574in}}%
\pgfpathlineto{\pgfqpoint{4.393003in}{2.418130in}}%
\pgfpathlineto{\pgfqpoint{4.450080in}{2.349640in}}%
\pgfpathlineto{\pgfqpoint{4.507158in}{2.333249in}}%
\pgfpathlineto{\pgfqpoint{4.564235in}{2.345795in}}%
\pgfpathlineto{\pgfqpoint{4.621312in}{2.366818in}}%
\pgfpathlineto{\pgfqpoint{4.678389in}{2.383471in}}%
\pgfpathlineto{\pgfqpoint{4.735466in}{2.390088in}}%
\pgfpathlineto{\pgfqpoint{4.792543in}{2.387226in}}%
\pgfpathlineto{\pgfqpoint{4.849620in}{2.379768in}}%
\pgfpathlineto{\pgfqpoint{4.906697in}{2.372902in}}%
\pgfpathlineto{\pgfqpoint{4.963774in}{2.369775in}}%
\pgfpathlineto{\pgfqpoint{5.020851in}{2.370761in}}%
\pgfpathlineto{\pgfqpoint{5.077929in}{2.374156in}}%
\pgfpathlineto{\pgfqpoint{5.135005in}{2.376000in}}%
\pgfpathlineto{\pgfqpoint{5.192083in}{2.376000in}}%
\pgfpathlineto{\pgfqpoint{5.249160in}{2.376000in}}%
\pgfpathlineto{\pgfqpoint{5.306237in}{2.376000in}}%
\pgfpathlineto{\pgfqpoint{5.363314in}{2.376000in}}%
\pgfpathlineto{\pgfqpoint{5.420391in}{2.376000in}}%
\pgfpathlineto{\pgfqpoint{5.477468in}{2.376000in}}%
\pgfpathlineto{\pgfqpoint{5.534545in}{2.376000in}}%
\pgfusepath{stroke}%
\end{pgfscope}%
\begin{pgfscope}%
\pgfpathrectangle{\pgfqpoint{0.800000in}{0.528000in}}{\pgfqpoint{4.960000in}{3.696000in}}%
\pgfusepath{clip}%
\pgfsetbuttcap%
\pgfsetroundjoin%
\pgfsetlinewidth{1.505625pt}%
\definecolor{currentstroke}{rgb}{1.000000,0.000000,0.000000}%
\pgfsetstrokecolor{currentstroke}%
\pgfsetdash{{5.550000pt}{2.400000pt}}{0.000000pt}%
\pgfpathmoveto{\pgfqpoint{1.025455in}{2.376000in}}%
\pgfpathlineto{\pgfqpoint{1.082532in}{2.376000in}}%
\pgfpathlineto{\pgfqpoint{1.139609in}{2.376000in}}%
\pgfpathlineto{\pgfqpoint{1.196686in}{2.376000in}}%
\pgfpathlineto{\pgfqpoint{1.253763in}{2.376000in}}%
\pgfpathlineto{\pgfqpoint{1.310840in}{2.376000in}}%
\pgfpathlineto{\pgfqpoint{1.367917in}{2.376000in}}%
\pgfpathlineto{\pgfqpoint{1.424994in}{2.376000in}}%
\pgfpathlineto{\pgfqpoint{1.482071in}{2.376000in}}%
\pgfpathlineto{\pgfqpoint{1.539148in}{2.376000in}}%
\pgfpathlineto{\pgfqpoint{1.596226in}{2.376000in}}%
\pgfpathlineto{\pgfqpoint{1.653303in}{2.376000in}}%
\pgfpathlineto{\pgfqpoint{1.710380in}{2.376000in}}%
\pgfpathlineto{\pgfqpoint{1.767457in}{2.376000in}}%
\pgfpathlineto{\pgfqpoint{1.824534in}{2.376000in}}%
\pgfpathlineto{\pgfqpoint{1.881611in}{2.376000in}}%
\pgfpathlineto{\pgfqpoint{1.938688in}{2.376000in}}%
\pgfpathlineto{\pgfqpoint{1.995765in}{2.376000in}}%
\pgfpathlineto{\pgfqpoint{2.052842in}{2.376000in}}%
\pgfpathlineto{\pgfqpoint{2.109919in}{2.376000in}}%
\pgfpathlineto{\pgfqpoint{2.166996in}{0.696000in}}%
\pgfpathlineto{\pgfqpoint{2.224074in}{0.696000in}}%
\pgfpathlineto{\pgfqpoint{2.281151in}{0.696000in}}%
\pgfpathlineto{\pgfqpoint{2.338228in}{0.696000in}}%
\pgfpathlineto{\pgfqpoint{2.395305in}{0.696000in}}%
\pgfpathlineto{\pgfqpoint{2.452382in}{0.696000in}}%
\pgfpathlineto{\pgfqpoint{2.509459in}{0.696000in}}%
\pgfpathlineto{\pgfqpoint{2.566536in}{0.696000in}}%
\pgfpathlineto{\pgfqpoint{2.623613in}{2.936000in}}%
\pgfpathlineto{\pgfqpoint{2.680690in}{2.936000in}}%
\pgfpathlineto{\pgfqpoint{2.737767in}{2.936000in}}%
\pgfpathlineto{\pgfqpoint{2.794845in}{2.936000in}}%
\pgfpathlineto{\pgfqpoint{2.851922in}{2.936000in}}%
\pgfpathlineto{\pgfqpoint{2.908999in}{2.936000in}}%
\pgfpathlineto{\pgfqpoint{2.966076in}{2.936000in}}%
\pgfpathlineto{\pgfqpoint{3.023153in}{2.936000in}}%
\pgfpathlineto{\pgfqpoint{3.080230in}{4.056000in}}%
\pgfpathlineto{\pgfqpoint{3.137307in}{4.056000in}}%
\pgfpathlineto{\pgfqpoint{3.194384in}{4.056000in}}%
\pgfpathlineto{\pgfqpoint{3.251461in}{4.056000in}}%
\pgfpathlineto{\pgfqpoint{3.308538in}{4.056000in}}%
\pgfpathlineto{\pgfqpoint{3.365616in}{4.056000in}}%
\pgfpathlineto{\pgfqpoint{3.422693in}{4.056000in}}%
\pgfpathlineto{\pgfqpoint{3.479770in}{4.056000in}}%
\pgfpathlineto{\pgfqpoint{3.536847in}{1.816000in}}%
\pgfpathlineto{\pgfqpoint{3.593924in}{1.816000in}}%
\pgfpathlineto{\pgfqpoint{3.651001in}{1.816000in}}%
\pgfpathlineto{\pgfqpoint{3.708078in}{1.816000in}}%
\pgfpathlineto{\pgfqpoint{3.765155in}{1.816000in}}%
\pgfpathlineto{\pgfqpoint{3.822232in}{1.816000in}}%
\pgfpathlineto{\pgfqpoint{3.879309in}{1.816000in}}%
\pgfpathlineto{\pgfqpoint{3.936386in}{1.816000in}}%
\pgfpathlineto{\pgfqpoint{3.993464in}{4.056000in}}%
\pgfpathlineto{\pgfqpoint{4.050541in}{4.056000in}}%
\pgfpathlineto{\pgfqpoint{4.107618in}{4.056000in}}%
\pgfpathlineto{\pgfqpoint{4.164695in}{4.056000in}}%
\pgfpathlineto{\pgfqpoint{4.221772in}{4.056000in}}%
\pgfpathlineto{\pgfqpoint{4.278849in}{4.056000in}}%
\pgfpathlineto{\pgfqpoint{4.335926in}{4.056000in}}%
\pgfpathlineto{\pgfqpoint{4.393003in}{4.056000in}}%
\pgfpathlineto{\pgfqpoint{4.450080in}{2.376000in}}%
\pgfpathlineto{\pgfqpoint{4.507158in}{2.376000in}}%
\pgfpathlineto{\pgfqpoint{4.564235in}{2.376000in}}%
\pgfpathlineto{\pgfqpoint{4.621312in}{2.376000in}}%
\pgfpathlineto{\pgfqpoint{4.678389in}{2.376000in}}%
\pgfpathlineto{\pgfqpoint{4.735466in}{2.376000in}}%
\pgfpathlineto{\pgfqpoint{4.792543in}{2.376000in}}%
\pgfpathlineto{\pgfqpoint{4.849620in}{2.376000in}}%
\pgfpathlineto{\pgfqpoint{4.906697in}{2.376000in}}%
\pgfpathlineto{\pgfqpoint{4.963774in}{2.376000in}}%
\pgfpathlineto{\pgfqpoint{5.020851in}{2.376000in}}%
\pgfpathlineto{\pgfqpoint{5.077929in}{2.376000in}}%
\pgfpathlineto{\pgfqpoint{5.135005in}{2.376000in}}%
\pgfpathlineto{\pgfqpoint{5.192083in}{2.376000in}}%
\pgfpathlineto{\pgfqpoint{5.249160in}{2.376000in}}%
\pgfpathlineto{\pgfqpoint{5.306237in}{2.376000in}}%
\pgfpathlineto{\pgfqpoint{5.363314in}{2.376000in}}%
\pgfpathlineto{\pgfqpoint{5.420391in}{2.376000in}}%
\pgfpathlineto{\pgfqpoint{5.477468in}{2.376000in}}%
\pgfpathlineto{\pgfqpoint{5.534545in}{2.376000in}}%
\pgfusepath{stroke}%
\end{pgfscope}%
\begin{pgfscope}%
\pgfsetrectcap%
\pgfsetmiterjoin%
\pgfsetlinewidth{0.803000pt}%
\definecolor{currentstroke}{rgb}{0.000000,0.000000,0.000000}%
\pgfsetstrokecolor{currentstroke}%
\pgfsetdash{}{0pt}%
\pgfpathmoveto{\pgfqpoint{0.800000in}{0.528000in}}%
\pgfpathlineto{\pgfqpoint{0.800000in}{4.224000in}}%
\pgfusepath{stroke}%
\end{pgfscope}%
\begin{pgfscope}%
\pgfsetrectcap%
\pgfsetmiterjoin%
\pgfsetlinewidth{0.803000pt}%
\definecolor{currentstroke}{rgb}{0.000000,0.000000,0.000000}%
\pgfsetstrokecolor{currentstroke}%
\pgfsetdash{}{0pt}%
\pgfpathmoveto{\pgfqpoint{5.760000in}{0.528000in}}%
\pgfpathlineto{\pgfqpoint{5.760000in}{4.224000in}}%
\pgfusepath{stroke}%
\end{pgfscope}%
\begin{pgfscope}%
\pgfsetrectcap%
\pgfsetmiterjoin%
\pgfsetlinewidth{0.803000pt}%
\definecolor{currentstroke}{rgb}{0.000000,0.000000,0.000000}%
\pgfsetstrokecolor{currentstroke}%
\pgfsetdash{}{0pt}%
\pgfpathmoveto{\pgfqpoint{0.800000in}{0.528000in}}%
\pgfpathlineto{\pgfqpoint{5.760000in}{0.528000in}}%
\pgfusepath{stroke}%
\end{pgfscope}%
\begin{pgfscope}%
\pgfsetrectcap%
\pgfsetmiterjoin%
\pgfsetlinewidth{0.803000pt}%
\definecolor{currentstroke}{rgb}{0.000000,0.000000,0.000000}%
\pgfsetstrokecolor{currentstroke}%
\pgfsetdash{}{0pt}%
\pgfpathmoveto{\pgfqpoint{0.800000in}{4.224000in}}%
\pgfpathlineto{\pgfqpoint{5.760000in}{4.224000in}}%
\pgfusepath{stroke}%
\end{pgfscope}%
\begin{pgfscope}%
\definecolor{textcolor}{rgb}{0.000000,0.000000,0.000000}%
\pgfsetstrokecolor{textcolor}%
\pgfsetfillcolor{textcolor}%
\pgftext[x=3.280000in,y=4.307333in,,base]{\color{textcolor}\rmfamily\fontsize{12.000000}{14.400000}\selectfont Nyquist Pulse shaped Signal}%
\end{pgfscope}%
\end{pgfpicture}%
\makeatother%
\endgroup%

    \end{center}
    \caption{A Nyquist Pulse Shaped Signal carrying the data `0010110111'}
\end{figure}


    
