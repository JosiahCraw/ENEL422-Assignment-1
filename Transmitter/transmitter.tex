\subsection{Real World Implementation}

To meet the specifications the system must have a data rate of 1Mbps and a bandwidth of no greater then
500kHz. This results in a bit period ($T_s$) of $1\mu$s, As Nyquist pulse shaping is used the Nyquist
bandwidth is $B_w / 2 = 500\textrm{kHz}$. Using Nyquist pulse shaping allows data transfer over a
bandwidth limited channel such as the one given in the specifications. For real world use the ideal
Nyquist pulse cannot be used as it is not physically realisable due to the infinite frequencies 
contained within the sinc pulse produced in the frequency domain. This can be solved by not using
an ideal Nyquist pulse such as a root raised cosine or a raised cosine pulse as these have the same characteristics
as the Nyquist pulse where they are zeros at all time periods other than at `0', this stops a past symbol
from adding to the current signal .

\subsubsection{Binary Polar Signalling}
Binary polar signalling is a very robust signalling method as is has a high noise tolerance. This is
due to the large distance between the values for `1' and `0' this is advantageous when transmitting
over a noisy medium as the signal is less likely to have bit errors when it is received.
However, it has a much lower throughput compared to other options such as 4-PAM and 8-PAM. This transmission type also
will introduce distortion due to the bandwidth limitations of the channel as there will be tail frequencies that are
removed. The noise tolerance makes binary polar signalling suitable for wireless communications as this environment 
can have much more noise over a much further distance.


\subsubsection{4-PAM}
PAM signalling has the advantage of being able to transfer more data per signal period, in the case of 4-PAM
two data bits are transferred per symbol period. These bit are encoded in the amplitude of the pulse
the encoding is found in Table \ref{table:4-pam-coding}. This encoding allows 4-PAM to have double the transmission
rate of binary polar signaling. This extra bandwidth allows the Nyquist pulse to use more bandwidth to reduce the intersymbol
interference (ISI) of the system by using a Nyquist pulse with a higher roll off, reducing the impact of ISI for future bits
if the receiver samples incorrectly. 4-PAM however, has a lower noise tolerance when compared to binary polar signalling as
the distance between the symbols is less as there a more values period making the signal less noise tolerant.

\begin{table}[h]
    \label{table:4-pam-coding}
    \caption{4-PAM Encoding}
    \begin{center}
        \begin{tabular} { | m{1cm} || m{1cm} | }
           \hline
           {`00'} & {-3} \\
           \hline
           {`01'} & {-1} \\
           \hline
           {`10'} & {1} \\
           \hline
           {`11'} & {3} \\
           \hline
        \end{tabular}
    \end{center}
\end{table}
% 4-PAM is also a reasonably robust signalling method as there is still some noise tolerance as the different levels
% are reasonably well defined. This signaling method also allows double the data rate when compared to binary polar signalling.
% These properties make this method well suited to wired communication as noise is less noticeable in the signal as the medium
% can be controlled.

% 4-PAM is still usable in the real world, especially within a wired communication system. This is because the signal can be easily affected by noise as the amplitude
% spacing between `00' and `01' is smaller than the difference between binary polar signaling, assuming that the highest and lowest amplitudes are the same for both signals.
% However, 4-PAM provided greater data rates when compared to binary polar encoding as twice the number of bit can be sent in the same bandwidth.

\subsubsection{8-PAM}
8-PAM has the same advantages as 4-PAM in regards to throughput with double the levels of 4-PAM, 8-PAM is able to transmit 3
bits per symbol. This however, comes with worse noise tolerance than 4-PAM as the different levels are even closer together
this causes the signal to become much less noise tolerant than either 4-PAM or binary polar signalling.

\subsubsection{Recommedation}
Overall, 4-PAM best matches the requirements as it would have low distortion due to the excess bandwidth available to use a Nyquist pulse
with a high roll off factor. It also has higher noise tolerance when compared to 8-PAM and as the extra spectral efficiency gained through
8-PAM is not required by the specifications 4-PAM is the most suitable choice.

\subsection{Power Spectral Density of 4-PAM (Rectangular Pulse Shaping)}

The equation for the power spectral density (PSD) of a line code signal is shown in (\ref{eqn:psd-formula}) where $|P(f)|^2$ is
the energy spectral density of the pulse and $S_Y$ is the power spectral density of the data.

\begin{equation}
    \label{eqn:psd-formula}
    S_Y(f) = |P(f)|^2 S_X(f)
\end{equation}

$S_X$ is defined as (\ref{eqn:sx}) where $R_n$ is the average of average power of all of the signal data levels as shown in
(~\ref{eqn:rn}).

\begin{equation}
    \label{eqn:sx}
    S_X = \frac{1}{T_s}[R_0 + 2 \sum_{n=1}^{\infty}{R_n \cos 2 \pi n f T_s}]
\end{equation}

\begin{equation}
    \label{eqn:rn}
    R_n=\lim_{N\to\infty} \frac{1}{N} \sum_{k}a_k a_{k-n}
\end{equation}

Assuming that the data is random the probability of each data level is equal.

\begin{equation}
    \label{eqn:ak-dist}
    P_{a_k} = 
    \begin{dcases}
        \frac{1}{4} \quad a_k = 3 \\
        \frac{1}{4} \quad a_k = 1 \\
        \frac{1}{4} \quad a_k = -1 \\
        \frac{1}{4} \quad a_k = -3
    \end{dcases}
\end{equation}

Therefore as $R_n$ is the average power of all data and each of the symbols is equally likely the average is
$R_n=0$ \\

As in the (\ref{eqn:ak-dist} the probabilities are equal for all $a_k$ therefore $a_k^2$ results in two equally likely 
options as shown in (~\ref{eqn:r0}),
\begin{equation}
    \label{eqn:r0}
    R_0 = \lim_{N\to\infty} \frac{1}{N} \sum_{k} {a_{k}}^2 \quad \textrm{With,} \quad P_{{a_{k}}^2} = 
    \begin{dcases}
        \frac{1}{2} \quad a_k^2 = 9 \\
        \frac{1}{2} \quad a_k^2 = 1
    \end{dcases}
\end{equation}

Therefore as $a_k^2=9 \quad \textrm{or} \quad a_k^2=1 $ the average comes to,
\begin{equation}
    \label{eqn:r0-value}
    R_0 = 5
\end{equation}

The required bit period can be derived from the required data rate.
\begin{equation}
    \label{eqn:ts}
    T_b = \frac{1}{R_b} \quad \textrm{Therefore with a bitrate of 1Mbps,} \quad T_b = \frac{1}{1 \times 10^6}
\end{equation}

As 4-PAM has two bits per pulse $T_s = 2 T_b$ so $T_s = 2 \mu$s Therefore,
\begin{equation}
    \label{eqn:sx-real}
    S_X(f) = \frac{1}{T_s} [5 + 0] \quad \textrm{For $T_s = 2\mu$s} \quad S_x=2500000
\end{equation}
\\
The rectangular pulse is defined as,

\begin{equation}
    \label{eqn:pt}
    p(t) = \textrm{rect}(\frac{t}{T_s})
\end{equation}

\begin{equation}
    \label{eqn:pf}
    P(f) = T_s \textrm{sinc}(\pi f T_s)
\end{equation}

The equation for PSD is,
\begin{equation}
    \label{eqn:PSD-4-PAM-rect}
    S_Y(f) = \abs{T_s \textrm{sinc}(f \pi T_s)}^2 S_x(f)
\end{equation}

So,
\begin{equation}
    \label{eqn:sy-value}
    S_Y(f) = \abs{T_s \textrm{sinc}(f \pi T_s)}^2 (2500000)
\end{equation}
    
As shown in the PSD in Figure \ref{fig:psd-rect} the rectangular pulse signal has frequencies present outside
the bandlimits, the removal of these would result in a distorted signal. This shows that this signal does not meet
bandwidth specification.

\begin{figure}[h]
    \begin{center}
        \begin{tikzpicture}
    \begin{axis}[
        axis lines = left,
        xlabel = $f$,
        ylabel = {$S_y(f)$},
    ]
    \addplot [
        domain=-0:500000000, 
        samples=2000, 
        color=red,
    ]
    {(abs(0.000001*(sin((x*pi*0.000002)/2)/((x*pi*0.000002)/2)))^2)*2500000)};
    \addlegendentry{$S_y$}

    \addplot [
        color=blue,
    ]
    coordinates {(50000000, 0.0000000005) (50000000, 0)};
    \addlegendentry{$B_f$}
    
    \end{axis}
\end{tikzpicture}
        \label{fig:psd-rect}
        \caption{PSD of a 4-PAM Signal with rectangular pulse shaping}
    \end{center}
\end{figure}

The simulated PSD of a RCOS pulse is shown in Figure \ref{fig:sim-psd}.

\begin{figure}[H]
    \begin{center}
        %% Creator: Matplotlib, PGF backend
%%
%% To include the figure in your LaTeX document, write
%%   \input{<filename>.pgf}
%%
%% Make sure the required packages are loaded in your preamble
%%   \usepackage{pgf}
%%
%% and, on pdftex
%%   \usepackage[utf8]{inputenc}\DeclareUnicodeCharacter{2212}{-}
%%
%% or, on luatex and xetex
%%   \usepackage{unicode-math}
%%
%% Figures using additional raster images can only be included by \input if
%% they are in the same directory as the main LaTeX file. For loading figures
%% from other directories you can use the `import` package
%%   \usepackage{import}
%%
%% and then include the figures with
%%   \import{<path to file>}{<filename>.pgf}
%%
%% Matplotlib used the following preamble
%%
\begingroup%
\makeatletter%
\begin{pgfpicture}%
\pgfpathrectangle{\pgfpointorigin}{\pgfqpoint{6.400000in}{4.800000in}}%
\pgfusepath{use as bounding box, clip}%
\begin{pgfscope}%
\pgfsetbuttcap%
\pgfsetmiterjoin%
\definecolor{currentfill}{rgb}{1.000000,1.000000,1.000000}%
\pgfsetfillcolor{currentfill}%
\pgfsetlinewidth{0.000000pt}%
\definecolor{currentstroke}{rgb}{1.000000,1.000000,1.000000}%
\pgfsetstrokecolor{currentstroke}%
\pgfsetdash{}{0pt}%
\pgfpathmoveto{\pgfqpoint{0.000000in}{0.000000in}}%
\pgfpathlineto{\pgfqpoint{6.400000in}{0.000000in}}%
\pgfpathlineto{\pgfqpoint{6.400000in}{4.800000in}}%
\pgfpathlineto{\pgfqpoint{0.000000in}{4.800000in}}%
\pgfpathclose%
\pgfusepath{fill}%
\end{pgfscope}%
\begin{pgfscope}%
\pgfsetbuttcap%
\pgfsetmiterjoin%
\definecolor{currentfill}{rgb}{1.000000,1.000000,1.000000}%
\pgfsetfillcolor{currentfill}%
\pgfsetlinewidth{0.000000pt}%
\definecolor{currentstroke}{rgb}{0.000000,0.000000,0.000000}%
\pgfsetstrokecolor{currentstroke}%
\pgfsetstrokeopacity{0.000000}%
\pgfsetdash{}{0pt}%
\pgfpathmoveto{\pgfqpoint{0.649696in}{2.852538in}}%
\pgfpathlineto{\pgfqpoint{6.358330in}{2.852538in}}%
\pgfpathlineto{\pgfqpoint{6.358330in}{4.559256in}}%
\pgfpathlineto{\pgfqpoint{0.649696in}{4.559256in}}%
\pgfpathclose%
\pgfusepath{fill}%
\end{pgfscope}%
\begin{pgfscope}%
\pgfsetbuttcap%
\pgfsetroundjoin%
\definecolor{currentfill}{rgb}{0.000000,0.000000,0.000000}%
\pgfsetfillcolor{currentfill}%
\pgfsetlinewidth{0.803000pt}%
\definecolor{currentstroke}{rgb}{0.000000,0.000000,0.000000}%
\pgfsetstrokecolor{currentstroke}%
\pgfsetdash{}{0pt}%
\pgfsys@defobject{currentmarker}{\pgfqpoint{0.000000in}{-0.048611in}}{\pgfqpoint{0.000000in}{0.000000in}}{%
\pgfpathmoveto{\pgfqpoint{0.000000in}{0.000000in}}%
\pgfpathlineto{\pgfqpoint{0.000000in}{-0.048611in}}%
\pgfusepath{stroke,fill}%
}%
\begin{pgfscope}%
\pgfsys@transformshift{0.909179in}{2.852538in}%
\pgfsys@useobject{currentmarker}{}%
\end{pgfscope}%
\end{pgfscope}%
\begin{pgfscope}%
\definecolor{textcolor}{rgb}{0.000000,0.000000,0.000000}%
\pgfsetstrokecolor{textcolor}%
\pgfsetfillcolor{textcolor}%
\pgftext[x=0.909179in,y=2.755316in,,top]{\color{textcolor}\rmfamily\fontsize{10.000000}{12.000000}\selectfont \(\displaystyle 0.000\)}%
\end{pgfscope}%
\begin{pgfscope}%
\pgfsetbuttcap%
\pgfsetroundjoin%
\definecolor{currentfill}{rgb}{0.000000,0.000000,0.000000}%
\pgfsetfillcolor{currentfill}%
\pgfsetlinewidth{0.803000pt}%
\definecolor{currentstroke}{rgb}{0.000000,0.000000,0.000000}%
\pgfsetstrokecolor{currentstroke}%
\pgfsetdash{}{0pt}%
\pgfsys@defobject{currentmarker}{\pgfqpoint{0.000000in}{-0.048611in}}{\pgfqpoint{0.000000in}{0.000000in}}{%
\pgfpathmoveto{\pgfqpoint{0.000000in}{0.000000in}}%
\pgfpathlineto{\pgfqpoint{0.000000in}{-0.048611in}}%
\pgfusepath{stroke,fill}%
}%
\begin{pgfscope}%
\pgfsys@transformshift{1.557928in}{2.852538in}%
\pgfsys@useobject{currentmarker}{}%
\end{pgfscope}%
\end{pgfscope}%
\begin{pgfscope}%
\definecolor{textcolor}{rgb}{0.000000,0.000000,0.000000}%
\pgfsetstrokecolor{textcolor}%
\pgfsetfillcolor{textcolor}%
\pgftext[x=1.557928in,y=2.755316in,,top]{\color{textcolor}\rmfamily\fontsize{10.000000}{12.000000}\selectfont \(\displaystyle 0.001\)}%
\end{pgfscope}%
\begin{pgfscope}%
\pgfsetbuttcap%
\pgfsetroundjoin%
\definecolor{currentfill}{rgb}{0.000000,0.000000,0.000000}%
\pgfsetfillcolor{currentfill}%
\pgfsetlinewidth{0.803000pt}%
\definecolor{currentstroke}{rgb}{0.000000,0.000000,0.000000}%
\pgfsetstrokecolor{currentstroke}%
\pgfsetdash{}{0pt}%
\pgfsys@defobject{currentmarker}{\pgfqpoint{0.000000in}{-0.048611in}}{\pgfqpoint{0.000000in}{0.000000in}}{%
\pgfpathmoveto{\pgfqpoint{0.000000in}{0.000000in}}%
\pgfpathlineto{\pgfqpoint{0.000000in}{-0.048611in}}%
\pgfusepath{stroke,fill}%
}%
\begin{pgfscope}%
\pgfsys@transformshift{2.206677in}{2.852538in}%
\pgfsys@useobject{currentmarker}{}%
\end{pgfscope}%
\end{pgfscope}%
\begin{pgfscope}%
\definecolor{textcolor}{rgb}{0.000000,0.000000,0.000000}%
\pgfsetstrokecolor{textcolor}%
\pgfsetfillcolor{textcolor}%
\pgftext[x=2.206677in,y=2.755316in,,top]{\color{textcolor}\rmfamily\fontsize{10.000000}{12.000000}\selectfont \(\displaystyle 0.002\)}%
\end{pgfscope}%
\begin{pgfscope}%
\pgfsetbuttcap%
\pgfsetroundjoin%
\definecolor{currentfill}{rgb}{0.000000,0.000000,0.000000}%
\pgfsetfillcolor{currentfill}%
\pgfsetlinewidth{0.803000pt}%
\definecolor{currentstroke}{rgb}{0.000000,0.000000,0.000000}%
\pgfsetstrokecolor{currentstroke}%
\pgfsetdash{}{0pt}%
\pgfsys@defobject{currentmarker}{\pgfqpoint{0.000000in}{-0.048611in}}{\pgfqpoint{0.000000in}{0.000000in}}{%
\pgfpathmoveto{\pgfqpoint{0.000000in}{0.000000in}}%
\pgfpathlineto{\pgfqpoint{0.000000in}{-0.048611in}}%
\pgfusepath{stroke,fill}%
}%
\begin{pgfscope}%
\pgfsys@transformshift{2.855426in}{2.852538in}%
\pgfsys@useobject{currentmarker}{}%
\end{pgfscope}%
\end{pgfscope}%
\begin{pgfscope}%
\definecolor{textcolor}{rgb}{0.000000,0.000000,0.000000}%
\pgfsetstrokecolor{textcolor}%
\pgfsetfillcolor{textcolor}%
\pgftext[x=2.855426in,y=2.755316in,,top]{\color{textcolor}\rmfamily\fontsize{10.000000}{12.000000}\selectfont \(\displaystyle 0.003\)}%
\end{pgfscope}%
\begin{pgfscope}%
\pgfsetbuttcap%
\pgfsetroundjoin%
\definecolor{currentfill}{rgb}{0.000000,0.000000,0.000000}%
\pgfsetfillcolor{currentfill}%
\pgfsetlinewidth{0.803000pt}%
\definecolor{currentstroke}{rgb}{0.000000,0.000000,0.000000}%
\pgfsetstrokecolor{currentstroke}%
\pgfsetdash{}{0pt}%
\pgfsys@defobject{currentmarker}{\pgfqpoint{0.000000in}{-0.048611in}}{\pgfqpoint{0.000000in}{0.000000in}}{%
\pgfpathmoveto{\pgfqpoint{0.000000in}{0.000000in}}%
\pgfpathlineto{\pgfqpoint{0.000000in}{-0.048611in}}%
\pgfusepath{stroke,fill}%
}%
\begin{pgfscope}%
\pgfsys@transformshift{3.504175in}{2.852538in}%
\pgfsys@useobject{currentmarker}{}%
\end{pgfscope}%
\end{pgfscope}%
\begin{pgfscope}%
\definecolor{textcolor}{rgb}{0.000000,0.000000,0.000000}%
\pgfsetstrokecolor{textcolor}%
\pgfsetfillcolor{textcolor}%
\pgftext[x=3.504175in,y=2.755316in,,top]{\color{textcolor}\rmfamily\fontsize{10.000000}{12.000000}\selectfont \(\displaystyle 0.004\)}%
\end{pgfscope}%
\begin{pgfscope}%
\pgfsetbuttcap%
\pgfsetroundjoin%
\definecolor{currentfill}{rgb}{0.000000,0.000000,0.000000}%
\pgfsetfillcolor{currentfill}%
\pgfsetlinewidth{0.803000pt}%
\definecolor{currentstroke}{rgb}{0.000000,0.000000,0.000000}%
\pgfsetstrokecolor{currentstroke}%
\pgfsetdash{}{0pt}%
\pgfsys@defobject{currentmarker}{\pgfqpoint{0.000000in}{-0.048611in}}{\pgfqpoint{0.000000in}{0.000000in}}{%
\pgfpathmoveto{\pgfqpoint{0.000000in}{0.000000in}}%
\pgfpathlineto{\pgfqpoint{0.000000in}{-0.048611in}}%
\pgfusepath{stroke,fill}%
}%
\begin{pgfscope}%
\pgfsys@transformshift{4.152924in}{2.852538in}%
\pgfsys@useobject{currentmarker}{}%
\end{pgfscope}%
\end{pgfscope}%
\begin{pgfscope}%
\definecolor{textcolor}{rgb}{0.000000,0.000000,0.000000}%
\pgfsetstrokecolor{textcolor}%
\pgfsetfillcolor{textcolor}%
\pgftext[x=4.152924in,y=2.755316in,,top]{\color{textcolor}\rmfamily\fontsize{10.000000}{12.000000}\selectfont \(\displaystyle 0.005\)}%
\end{pgfscope}%
\begin{pgfscope}%
\pgfsetbuttcap%
\pgfsetroundjoin%
\definecolor{currentfill}{rgb}{0.000000,0.000000,0.000000}%
\pgfsetfillcolor{currentfill}%
\pgfsetlinewidth{0.803000pt}%
\definecolor{currentstroke}{rgb}{0.000000,0.000000,0.000000}%
\pgfsetstrokecolor{currentstroke}%
\pgfsetdash{}{0pt}%
\pgfsys@defobject{currentmarker}{\pgfqpoint{0.000000in}{-0.048611in}}{\pgfqpoint{0.000000in}{0.000000in}}{%
\pgfpathmoveto{\pgfqpoint{0.000000in}{0.000000in}}%
\pgfpathlineto{\pgfqpoint{0.000000in}{-0.048611in}}%
\pgfusepath{stroke,fill}%
}%
\begin{pgfscope}%
\pgfsys@transformshift{4.801673in}{2.852538in}%
\pgfsys@useobject{currentmarker}{}%
\end{pgfscope}%
\end{pgfscope}%
\begin{pgfscope}%
\definecolor{textcolor}{rgb}{0.000000,0.000000,0.000000}%
\pgfsetstrokecolor{textcolor}%
\pgfsetfillcolor{textcolor}%
\pgftext[x=4.801673in,y=2.755316in,,top]{\color{textcolor}\rmfamily\fontsize{10.000000}{12.000000}\selectfont \(\displaystyle 0.006\)}%
\end{pgfscope}%
\begin{pgfscope}%
\pgfsetbuttcap%
\pgfsetroundjoin%
\definecolor{currentfill}{rgb}{0.000000,0.000000,0.000000}%
\pgfsetfillcolor{currentfill}%
\pgfsetlinewidth{0.803000pt}%
\definecolor{currentstroke}{rgb}{0.000000,0.000000,0.000000}%
\pgfsetstrokecolor{currentstroke}%
\pgfsetdash{}{0pt}%
\pgfsys@defobject{currentmarker}{\pgfqpoint{0.000000in}{-0.048611in}}{\pgfqpoint{0.000000in}{0.000000in}}{%
\pgfpathmoveto{\pgfqpoint{0.000000in}{0.000000in}}%
\pgfpathlineto{\pgfqpoint{0.000000in}{-0.048611in}}%
\pgfusepath{stroke,fill}%
}%
\begin{pgfscope}%
\pgfsys@transformshift{5.450422in}{2.852538in}%
\pgfsys@useobject{currentmarker}{}%
\end{pgfscope}%
\end{pgfscope}%
\begin{pgfscope}%
\definecolor{textcolor}{rgb}{0.000000,0.000000,0.000000}%
\pgfsetstrokecolor{textcolor}%
\pgfsetfillcolor{textcolor}%
\pgftext[x=5.450422in,y=2.755316in,,top]{\color{textcolor}\rmfamily\fontsize{10.000000}{12.000000}\selectfont \(\displaystyle 0.007\)}%
\end{pgfscope}%
\begin{pgfscope}%
\pgfsetbuttcap%
\pgfsetroundjoin%
\definecolor{currentfill}{rgb}{0.000000,0.000000,0.000000}%
\pgfsetfillcolor{currentfill}%
\pgfsetlinewidth{0.803000pt}%
\definecolor{currentstroke}{rgb}{0.000000,0.000000,0.000000}%
\pgfsetstrokecolor{currentstroke}%
\pgfsetdash{}{0pt}%
\pgfsys@defobject{currentmarker}{\pgfqpoint{0.000000in}{-0.048611in}}{\pgfqpoint{0.000000in}{0.000000in}}{%
\pgfpathmoveto{\pgfqpoint{0.000000in}{0.000000in}}%
\pgfpathlineto{\pgfqpoint{0.000000in}{-0.048611in}}%
\pgfusepath{stroke,fill}%
}%
\begin{pgfscope}%
\pgfsys@transformshift{6.099171in}{2.852538in}%
\pgfsys@useobject{currentmarker}{}%
\end{pgfscope}%
\end{pgfscope}%
\begin{pgfscope}%
\definecolor{textcolor}{rgb}{0.000000,0.000000,0.000000}%
\pgfsetstrokecolor{textcolor}%
\pgfsetfillcolor{textcolor}%
\pgftext[x=6.099171in,y=2.755316in,,top]{\color{textcolor}\rmfamily\fontsize{10.000000}{12.000000}\selectfont \(\displaystyle 0.008\)}%
\end{pgfscope}%
\begin{pgfscope}%
\definecolor{textcolor}{rgb}{0.000000,0.000000,0.000000}%
\pgfsetstrokecolor{textcolor}%
\pgfsetfillcolor{textcolor}%
\pgftext[x=3.504013in,y=2.576303in,,top]{\color{textcolor}\rmfamily\fontsize{10.000000}{12.000000}\selectfont Time (s)}%
\end{pgfscope}%
\begin{pgfscope}%
\pgfsetbuttcap%
\pgfsetroundjoin%
\definecolor{currentfill}{rgb}{0.000000,0.000000,0.000000}%
\pgfsetfillcolor{currentfill}%
\pgfsetlinewidth{0.803000pt}%
\definecolor{currentstroke}{rgb}{0.000000,0.000000,0.000000}%
\pgfsetstrokecolor{currentstroke}%
\pgfsetdash{}{0pt}%
\pgfsys@defobject{currentmarker}{\pgfqpoint{-0.048611in}{0.000000in}}{\pgfqpoint{0.000000in}{0.000000in}}{%
\pgfpathmoveto{\pgfqpoint{0.000000in}{0.000000in}}%
\pgfpathlineto{\pgfqpoint{-0.048611in}{0.000000in}}%
\pgfusepath{stroke,fill}%
}%
\begin{pgfscope}%
\pgfsys@transformshift{0.649696in}{3.188710in}%
\pgfsys@useobject{currentmarker}{}%
\end{pgfscope}%
\end{pgfscope}%
\begin{pgfscope}%
\definecolor{textcolor}{rgb}{0.000000,0.000000,0.000000}%
\pgfsetstrokecolor{textcolor}%
\pgfsetfillcolor{textcolor}%
\pgftext[x=0.375004in, y=3.140484in, left, base]{\color{textcolor}\rmfamily\fontsize{10.000000}{12.000000}\selectfont \(\displaystyle -2\)}%
\end{pgfscope}%
\begin{pgfscope}%
\pgfsetbuttcap%
\pgfsetroundjoin%
\definecolor{currentfill}{rgb}{0.000000,0.000000,0.000000}%
\pgfsetfillcolor{currentfill}%
\pgfsetlinewidth{0.803000pt}%
\definecolor{currentstroke}{rgb}{0.000000,0.000000,0.000000}%
\pgfsetstrokecolor{currentstroke}%
\pgfsetdash{}{0pt}%
\pgfsys@defobject{currentmarker}{\pgfqpoint{-0.048611in}{0.000000in}}{\pgfqpoint{0.000000in}{0.000000in}}{%
\pgfpathmoveto{\pgfqpoint{0.000000in}{0.000000in}}%
\pgfpathlineto{\pgfqpoint{-0.048611in}{0.000000in}}%
\pgfusepath{stroke,fill}%
}%
\begin{pgfscope}%
\pgfsys@transformshift{0.649696in}{3.705897in}%
\pgfsys@useobject{currentmarker}{}%
\end{pgfscope}%
\end{pgfscope}%
\begin{pgfscope}%
\definecolor{textcolor}{rgb}{0.000000,0.000000,0.000000}%
\pgfsetstrokecolor{textcolor}%
\pgfsetfillcolor{textcolor}%
\pgftext[x=0.483029in, y=3.657672in, left, base]{\color{textcolor}\rmfamily\fontsize{10.000000}{12.000000}\selectfont \(\displaystyle 0\)}%
\end{pgfscope}%
\begin{pgfscope}%
\pgfsetbuttcap%
\pgfsetroundjoin%
\definecolor{currentfill}{rgb}{0.000000,0.000000,0.000000}%
\pgfsetfillcolor{currentfill}%
\pgfsetlinewidth{0.803000pt}%
\definecolor{currentstroke}{rgb}{0.000000,0.000000,0.000000}%
\pgfsetstrokecolor{currentstroke}%
\pgfsetdash{}{0pt}%
\pgfsys@defobject{currentmarker}{\pgfqpoint{-0.048611in}{0.000000in}}{\pgfqpoint{0.000000in}{0.000000in}}{%
\pgfpathmoveto{\pgfqpoint{0.000000in}{0.000000in}}%
\pgfpathlineto{\pgfqpoint{-0.048611in}{0.000000in}}%
\pgfusepath{stroke,fill}%
}%
\begin{pgfscope}%
\pgfsys@transformshift{0.649696in}{4.223084in}%
\pgfsys@useobject{currentmarker}{}%
\end{pgfscope}%
\end{pgfscope}%
\begin{pgfscope}%
\definecolor{textcolor}{rgb}{0.000000,0.000000,0.000000}%
\pgfsetstrokecolor{textcolor}%
\pgfsetfillcolor{textcolor}%
\pgftext[x=0.483029in, y=4.174859in, left, base]{\color{textcolor}\rmfamily\fontsize{10.000000}{12.000000}\selectfont \(\displaystyle 2\)}%
\end{pgfscope}%
\begin{pgfscope}%
\definecolor{textcolor}{rgb}{0.000000,0.000000,0.000000}%
\pgfsetstrokecolor{textcolor}%
\pgfsetfillcolor{textcolor}%
\pgftext[x=0.319448in,y=3.705897in,,bottom,rotate=90.000000]{\color{textcolor}\rmfamily\fontsize{10.000000}{12.000000}\selectfont Amplitude}%
\end{pgfscope}%
\begin{pgfscope}%
\pgfpathrectangle{\pgfqpoint{0.649696in}{2.852538in}}{\pgfqpoint{5.708634in}{1.706718in}}%
\pgfusepath{clip}%
\pgfsetrectcap%
\pgfsetroundjoin%
\pgfsetlinewidth{1.505625pt}%
\definecolor{currentstroke}{rgb}{0.000000,0.000000,1.000000}%
\pgfsetstrokecolor{currentstroke}%
\pgfsetdash{}{0pt}%
\pgfpathmoveto{\pgfqpoint{0.909179in}{3.705897in}}%
\pgfpathlineto{\pgfqpoint{2.205055in}{3.705897in}}%
\pgfpathlineto{\pgfqpoint{2.206353in}{2.930116in}}%
\pgfpathlineto{\pgfqpoint{2.207650in}{2.930116in}}%
\pgfpathlineto{\pgfqpoint{2.208948in}{3.964491in}}%
\pgfpathlineto{\pgfqpoint{2.210245in}{3.964491in}}%
\pgfpathlineto{\pgfqpoint{2.211543in}{3.447303in}}%
\pgfpathlineto{\pgfqpoint{2.212840in}{3.447303in}}%
\pgfpathlineto{\pgfqpoint{2.214138in}{2.930116in}}%
\pgfpathlineto{\pgfqpoint{2.215435in}{2.930116in}}%
\pgfpathlineto{\pgfqpoint{2.216733in}{3.964491in}}%
\pgfpathlineto{\pgfqpoint{2.218030in}{3.964491in}}%
\pgfpathlineto{\pgfqpoint{2.219328in}{3.447303in}}%
\pgfpathlineto{\pgfqpoint{2.220625in}{3.447303in}}%
\pgfpathlineto{\pgfqpoint{2.221923in}{2.930116in}}%
\pgfpathlineto{\pgfqpoint{2.223220in}{2.930116in}}%
\pgfpathlineto{\pgfqpoint{2.224518in}{3.964491in}}%
\pgfpathlineto{\pgfqpoint{2.225815in}{3.964491in}}%
\pgfpathlineto{\pgfqpoint{2.227113in}{3.447303in}}%
\pgfpathlineto{\pgfqpoint{2.228410in}{3.447303in}}%
\pgfpathlineto{\pgfqpoint{2.229708in}{4.481678in}}%
\pgfpathlineto{\pgfqpoint{2.231005in}{4.481678in}}%
\pgfpathlineto{\pgfqpoint{2.232303in}{2.930116in}}%
\pgfpathlineto{\pgfqpoint{2.233600in}{2.930116in}}%
\pgfpathlineto{\pgfqpoint{2.234898in}{4.481678in}}%
\pgfpathlineto{\pgfqpoint{2.236195in}{4.481678in}}%
\pgfpathlineto{\pgfqpoint{2.237493in}{3.964491in}}%
\pgfpathlineto{\pgfqpoint{2.238790in}{3.964491in}}%
\pgfpathlineto{\pgfqpoint{2.240088in}{2.930116in}}%
\pgfpathlineto{\pgfqpoint{2.243980in}{2.930116in}}%
\pgfpathlineto{\pgfqpoint{2.245278in}{3.447303in}}%
\pgfpathlineto{\pgfqpoint{2.249170in}{3.447303in}}%
\pgfpathlineto{\pgfqpoint{2.250468in}{2.930116in}}%
\pgfpathlineto{\pgfqpoint{2.251765in}{2.930116in}}%
\pgfpathlineto{\pgfqpoint{2.253063in}{4.481678in}}%
\pgfpathlineto{\pgfqpoint{2.254360in}{4.481678in}}%
\pgfpathlineto{\pgfqpoint{2.255658in}{3.447303in}}%
\pgfpathlineto{\pgfqpoint{2.256955in}{3.447303in}}%
\pgfpathlineto{\pgfqpoint{2.258252in}{3.964491in}}%
\pgfpathlineto{\pgfqpoint{2.259550in}{3.964491in}}%
\pgfpathlineto{\pgfqpoint{2.260847in}{3.447303in}}%
\pgfpathlineto{\pgfqpoint{2.264740in}{3.447303in}}%
\pgfpathlineto{\pgfqpoint{2.266038in}{2.930116in}}%
\pgfpathlineto{\pgfqpoint{2.269930in}{2.930116in}}%
\pgfpathlineto{\pgfqpoint{2.271228in}{3.964491in}}%
\pgfpathlineto{\pgfqpoint{2.277715in}{3.964491in}}%
\pgfpathlineto{\pgfqpoint{2.279012in}{2.930116in}}%
\pgfpathlineto{\pgfqpoint{2.280310in}{2.930116in}}%
\pgfpathlineto{\pgfqpoint{2.281607in}{3.964491in}}%
\pgfpathlineto{\pgfqpoint{2.282905in}{3.964491in}}%
\pgfpathlineto{\pgfqpoint{2.284202in}{3.447303in}}%
\pgfpathlineto{\pgfqpoint{2.288095in}{3.447303in}}%
\pgfpathlineto{\pgfqpoint{2.289393in}{4.481678in}}%
\pgfpathlineto{\pgfqpoint{2.293285in}{4.481678in}}%
\pgfpathlineto{\pgfqpoint{2.294583in}{3.964491in}}%
\pgfpathlineto{\pgfqpoint{2.295880in}{3.964491in}}%
\pgfpathlineto{\pgfqpoint{2.297178in}{3.447303in}}%
\pgfpathlineto{\pgfqpoint{2.298475in}{3.447303in}}%
\pgfpathlineto{\pgfqpoint{2.299772in}{4.481678in}}%
\pgfpathlineto{\pgfqpoint{2.301070in}{4.481678in}}%
\pgfpathlineto{\pgfqpoint{2.302367in}{3.964491in}}%
\pgfpathlineto{\pgfqpoint{2.303665in}{3.964491in}}%
\pgfpathlineto{\pgfqpoint{2.304962in}{2.930116in}}%
\pgfpathlineto{\pgfqpoint{2.306260in}{2.930116in}}%
\pgfpathlineto{\pgfqpoint{2.307557in}{3.964491in}}%
\pgfpathlineto{\pgfqpoint{2.314045in}{3.964491in}}%
\pgfpathlineto{\pgfqpoint{2.315343in}{2.930116in}}%
\pgfpathlineto{\pgfqpoint{2.316640in}{2.930116in}}%
\pgfpathlineto{\pgfqpoint{2.317937in}{4.481678in}}%
\pgfpathlineto{\pgfqpoint{2.319235in}{4.481678in}}%
\pgfpathlineto{\pgfqpoint{2.320532in}{3.447303in}}%
\pgfpathlineto{\pgfqpoint{2.321830in}{3.447303in}}%
\pgfpathlineto{\pgfqpoint{2.323127in}{4.481678in}}%
\pgfpathlineto{\pgfqpoint{2.324425in}{4.481678in}}%
\pgfpathlineto{\pgfqpoint{2.325722in}{2.930116in}}%
\pgfpathlineto{\pgfqpoint{2.327020in}{2.930116in}}%
\pgfpathlineto{\pgfqpoint{2.328317in}{4.481678in}}%
\pgfpathlineto{\pgfqpoint{2.332210in}{4.481678in}}%
\pgfpathlineto{\pgfqpoint{2.333507in}{3.447303in}}%
\pgfpathlineto{\pgfqpoint{2.334805in}{3.447303in}}%
\pgfpathlineto{\pgfqpoint{2.336102in}{4.481678in}}%
\pgfpathlineto{\pgfqpoint{2.337400in}{4.481678in}}%
\pgfpathlineto{\pgfqpoint{2.338697in}{3.447303in}}%
\pgfpathlineto{\pgfqpoint{2.339995in}{3.447303in}}%
\pgfpathlineto{\pgfqpoint{2.341292in}{2.930116in}}%
\pgfpathlineto{\pgfqpoint{2.342590in}{2.930116in}}%
\pgfpathlineto{\pgfqpoint{2.343887in}{3.964491in}}%
\pgfpathlineto{\pgfqpoint{2.345185in}{3.964491in}}%
\pgfpathlineto{\pgfqpoint{2.346482in}{4.481678in}}%
\pgfpathlineto{\pgfqpoint{2.347780in}{4.481678in}}%
\pgfpathlineto{\pgfqpoint{2.349077in}{2.930116in}}%
\pgfpathlineto{\pgfqpoint{2.350375in}{2.930116in}}%
\pgfpathlineto{\pgfqpoint{2.351672in}{3.964491in}}%
\pgfpathlineto{\pgfqpoint{2.352970in}{3.964491in}}%
\pgfpathlineto{\pgfqpoint{2.354267in}{3.447303in}}%
\pgfpathlineto{\pgfqpoint{2.355565in}{3.447303in}}%
\pgfpathlineto{\pgfqpoint{2.356862in}{2.930116in}}%
\pgfpathlineto{\pgfqpoint{2.358160in}{2.930116in}}%
\pgfpathlineto{\pgfqpoint{2.359457in}{3.964491in}}%
\pgfpathlineto{\pgfqpoint{2.363350in}{3.964491in}}%
\pgfpathlineto{\pgfqpoint{2.364647in}{4.481678in}}%
\pgfpathlineto{\pgfqpoint{2.368540in}{4.481678in}}%
\pgfpathlineto{\pgfqpoint{2.369837in}{3.964491in}}%
\pgfpathlineto{\pgfqpoint{2.371135in}{3.964491in}}%
\pgfpathlineto{\pgfqpoint{2.372432in}{2.930116in}}%
\pgfpathlineto{\pgfqpoint{2.373730in}{2.930116in}}%
\pgfpathlineto{\pgfqpoint{2.375027in}{3.447303in}}%
\pgfpathlineto{\pgfqpoint{2.378920in}{3.447303in}}%
\pgfpathlineto{\pgfqpoint{2.380217in}{2.930116in}}%
\pgfpathlineto{\pgfqpoint{2.381515in}{2.930116in}}%
\pgfpathlineto{\pgfqpoint{2.382812in}{4.481678in}}%
\pgfpathlineto{\pgfqpoint{2.384110in}{4.481678in}}%
\pgfpathlineto{\pgfqpoint{2.385407in}{3.964491in}}%
\pgfpathlineto{\pgfqpoint{2.386705in}{3.964491in}}%
\pgfpathlineto{\pgfqpoint{2.388002in}{3.447303in}}%
\pgfpathlineto{\pgfqpoint{2.389300in}{3.447303in}}%
\pgfpathlineto{\pgfqpoint{2.390597in}{2.930116in}}%
\pgfpathlineto{\pgfqpoint{2.391895in}{2.930116in}}%
\pgfpathlineto{\pgfqpoint{2.393192in}{4.481678in}}%
\pgfpathlineto{\pgfqpoint{2.394490in}{4.481678in}}%
\pgfpathlineto{\pgfqpoint{2.395787in}{3.964491in}}%
\pgfpathlineto{\pgfqpoint{2.397085in}{3.964491in}}%
\pgfpathlineto{\pgfqpoint{2.398382in}{4.481678in}}%
\pgfpathlineto{\pgfqpoint{2.399680in}{4.481678in}}%
\pgfpathlineto{\pgfqpoint{2.400977in}{3.447303in}}%
\pgfpathlineto{\pgfqpoint{2.402275in}{3.447303in}}%
\pgfpathlineto{\pgfqpoint{2.403572in}{4.481678in}}%
\pgfpathlineto{\pgfqpoint{2.404870in}{4.481678in}}%
\pgfpathlineto{\pgfqpoint{2.406167in}{3.447303in}}%
\pgfpathlineto{\pgfqpoint{2.407465in}{3.447303in}}%
\pgfpathlineto{\pgfqpoint{2.408762in}{3.964491in}}%
\pgfpathlineto{\pgfqpoint{2.410060in}{3.964491in}}%
\pgfpathlineto{\pgfqpoint{2.411357in}{2.930116in}}%
\pgfpathlineto{\pgfqpoint{2.412655in}{2.930116in}}%
\pgfpathlineto{\pgfqpoint{2.413952in}{3.964491in}}%
\pgfpathlineto{\pgfqpoint{2.415250in}{3.964491in}}%
\pgfpathlineto{\pgfqpoint{2.416547in}{3.447303in}}%
\pgfpathlineto{\pgfqpoint{2.417845in}{3.447303in}}%
\pgfpathlineto{\pgfqpoint{2.419142in}{3.964491in}}%
\pgfpathlineto{\pgfqpoint{2.420440in}{3.964491in}}%
\pgfpathlineto{\pgfqpoint{2.421737in}{3.447303in}}%
\pgfpathlineto{\pgfqpoint{2.425630in}{3.447303in}}%
\pgfpathlineto{\pgfqpoint{2.426927in}{2.930116in}}%
\pgfpathlineto{\pgfqpoint{2.430820in}{2.930116in}}%
\pgfpathlineto{\pgfqpoint{2.432117in}{3.964491in}}%
\pgfpathlineto{\pgfqpoint{2.436010in}{3.964491in}}%
\pgfpathlineto{\pgfqpoint{2.437307in}{4.481678in}}%
\pgfpathlineto{\pgfqpoint{2.438605in}{4.481678in}}%
\pgfpathlineto{\pgfqpoint{2.439902in}{3.447303in}}%
\pgfpathlineto{\pgfqpoint{2.441200in}{3.447303in}}%
\pgfpathlineto{\pgfqpoint{2.442497in}{3.964491in}}%
\pgfpathlineto{\pgfqpoint{2.443795in}{3.964491in}}%
\pgfpathlineto{\pgfqpoint{2.445092in}{4.481678in}}%
\pgfpathlineto{\pgfqpoint{2.446390in}{4.481678in}}%
\pgfpathlineto{\pgfqpoint{2.447687in}{3.964491in}}%
\pgfpathlineto{\pgfqpoint{2.448985in}{3.964491in}}%
\pgfpathlineto{\pgfqpoint{2.450282in}{3.447303in}}%
\pgfpathlineto{\pgfqpoint{2.451580in}{3.447303in}}%
\pgfpathlineto{\pgfqpoint{2.452877in}{2.930116in}}%
\pgfpathlineto{\pgfqpoint{2.454175in}{2.930116in}}%
\pgfpathlineto{\pgfqpoint{2.455472in}{3.447303in}}%
\pgfpathlineto{\pgfqpoint{2.456770in}{3.447303in}}%
\pgfpathlineto{\pgfqpoint{2.458067in}{3.964491in}}%
\pgfpathlineto{\pgfqpoint{2.459365in}{3.964491in}}%
\pgfpathlineto{\pgfqpoint{2.460662in}{4.481678in}}%
\pgfpathlineto{\pgfqpoint{2.464555in}{4.481678in}}%
\pgfpathlineto{\pgfqpoint{2.465852in}{3.447303in}}%
\pgfpathlineto{\pgfqpoint{2.467150in}{3.447303in}}%
\pgfpathlineto{\pgfqpoint{2.468447in}{2.930116in}}%
\pgfpathlineto{\pgfqpoint{2.469745in}{2.930116in}}%
\pgfpathlineto{\pgfqpoint{2.471042in}{3.447303in}}%
\pgfpathlineto{\pgfqpoint{2.472340in}{3.447303in}}%
\pgfpathlineto{\pgfqpoint{2.473637in}{2.930116in}}%
\pgfpathlineto{\pgfqpoint{2.474935in}{2.930116in}}%
\pgfpathlineto{\pgfqpoint{2.476232in}{3.964491in}}%
\pgfpathlineto{\pgfqpoint{2.477530in}{3.964491in}}%
\pgfpathlineto{\pgfqpoint{2.478827in}{4.481678in}}%
\pgfpathlineto{\pgfqpoint{2.480125in}{4.481678in}}%
\pgfpathlineto{\pgfqpoint{2.481422in}{3.447303in}}%
\pgfpathlineto{\pgfqpoint{2.482720in}{3.447303in}}%
\pgfpathlineto{\pgfqpoint{2.484017in}{2.930116in}}%
\pgfpathlineto{\pgfqpoint{2.487910in}{2.930116in}}%
\pgfpathlineto{\pgfqpoint{2.489207in}{4.481678in}}%
\pgfpathlineto{\pgfqpoint{2.490505in}{4.481678in}}%
\pgfpathlineto{\pgfqpoint{2.491802in}{3.447303in}}%
\pgfpathlineto{\pgfqpoint{2.495695in}{3.447303in}}%
\pgfpathlineto{\pgfqpoint{2.496992in}{4.481678in}}%
\pgfpathlineto{\pgfqpoint{2.498290in}{4.481678in}}%
\pgfpathlineto{\pgfqpoint{2.499587in}{3.447303in}}%
\pgfpathlineto{\pgfqpoint{2.500885in}{3.447303in}}%
\pgfpathlineto{\pgfqpoint{2.502182in}{3.964491in}}%
\pgfpathlineto{\pgfqpoint{2.503480in}{3.964491in}}%
\pgfpathlineto{\pgfqpoint{2.504777in}{4.481678in}}%
\pgfpathlineto{\pgfqpoint{2.506075in}{4.481678in}}%
\pgfpathlineto{\pgfqpoint{2.507372in}{3.964491in}}%
\pgfpathlineto{\pgfqpoint{2.508670in}{3.964491in}}%
\pgfpathlineto{\pgfqpoint{2.509967in}{4.481678in}}%
\pgfpathlineto{\pgfqpoint{2.513860in}{4.481678in}}%
\pgfpathlineto{\pgfqpoint{2.515157in}{2.930116in}}%
\pgfpathlineto{\pgfqpoint{2.516455in}{2.930116in}}%
\pgfpathlineto{\pgfqpoint{2.517752in}{3.964491in}}%
\pgfpathlineto{\pgfqpoint{2.521645in}{3.964491in}}%
\pgfpathlineto{\pgfqpoint{2.522942in}{3.447303in}}%
\pgfpathlineto{\pgfqpoint{2.526835in}{3.447303in}}%
\pgfpathlineto{\pgfqpoint{2.528132in}{3.964491in}}%
\pgfpathlineto{\pgfqpoint{2.529430in}{3.964491in}}%
\pgfpathlineto{\pgfqpoint{2.530727in}{3.447303in}}%
\pgfpathlineto{\pgfqpoint{2.532025in}{3.447303in}}%
\pgfpathlineto{\pgfqpoint{2.533322in}{4.481678in}}%
\pgfpathlineto{\pgfqpoint{2.534620in}{4.481678in}}%
\pgfpathlineto{\pgfqpoint{2.535917in}{3.447303in}}%
\pgfpathlineto{\pgfqpoint{2.537215in}{3.447303in}}%
\pgfpathlineto{\pgfqpoint{2.538512in}{4.481678in}}%
\pgfpathlineto{\pgfqpoint{2.539810in}{4.481678in}}%
\pgfpathlineto{\pgfqpoint{2.541107in}{2.930116in}}%
\pgfpathlineto{\pgfqpoint{2.542405in}{2.930116in}}%
\pgfpathlineto{\pgfqpoint{2.543702in}{3.447303in}}%
\pgfpathlineto{\pgfqpoint{2.550190in}{3.447303in}}%
\pgfpathlineto{\pgfqpoint{2.551487in}{3.964491in}}%
\pgfpathlineto{\pgfqpoint{2.552785in}{3.964491in}}%
\pgfpathlineto{\pgfqpoint{2.554082in}{2.930116in}}%
\pgfpathlineto{\pgfqpoint{2.560570in}{2.930116in}}%
\pgfpathlineto{\pgfqpoint{2.561867in}{3.964491in}}%
\pgfpathlineto{\pgfqpoint{2.563165in}{3.964491in}}%
\pgfpathlineto{\pgfqpoint{2.564462in}{2.930116in}}%
\pgfpathlineto{\pgfqpoint{2.565760in}{2.930116in}}%
\pgfpathlineto{\pgfqpoint{2.567057in}{4.481678in}}%
\pgfpathlineto{\pgfqpoint{2.568355in}{4.481678in}}%
\pgfpathlineto{\pgfqpoint{2.569652in}{2.930116in}}%
\pgfpathlineto{\pgfqpoint{2.570950in}{2.930116in}}%
\pgfpathlineto{\pgfqpoint{2.572247in}{4.481678in}}%
\pgfpathlineto{\pgfqpoint{2.573545in}{4.481678in}}%
\pgfpathlineto{\pgfqpoint{2.574842in}{2.930116in}}%
\pgfpathlineto{\pgfqpoint{2.576139in}{2.930116in}}%
\pgfpathlineto{\pgfqpoint{2.577437in}{3.447303in}}%
\pgfpathlineto{\pgfqpoint{2.578734in}{3.447303in}}%
\pgfpathlineto{\pgfqpoint{2.580032in}{3.964491in}}%
\pgfpathlineto{\pgfqpoint{2.581330in}{3.964491in}}%
\pgfpathlineto{\pgfqpoint{2.582627in}{2.930116in}}%
\pgfpathlineto{\pgfqpoint{2.586520in}{2.930116in}}%
\pgfpathlineto{\pgfqpoint{2.587817in}{4.481678in}}%
\pgfpathlineto{\pgfqpoint{2.589115in}{4.481678in}}%
\pgfpathlineto{\pgfqpoint{2.590412in}{3.447303in}}%
\pgfpathlineto{\pgfqpoint{2.591710in}{3.447303in}}%
\pgfpathlineto{\pgfqpoint{2.593007in}{3.964491in}}%
\pgfpathlineto{\pgfqpoint{2.594305in}{3.964491in}}%
\pgfpathlineto{\pgfqpoint{2.595602in}{4.481678in}}%
\pgfpathlineto{\pgfqpoint{2.599494in}{4.481678in}}%
\pgfpathlineto{\pgfqpoint{2.600792in}{3.447303in}}%
\pgfpathlineto{\pgfqpoint{2.602089in}{3.447303in}}%
\pgfpathlineto{\pgfqpoint{2.603387in}{4.481678in}}%
\pgfpathlineto{\pgfqpoint{2.604684in}{4.481678in}}%
\pgfpathlineto{\pgfqpoint{2.605982in}{2.930116in}}%
\pgfpathlineto{\pgfqpoint{2.609875in}{2.930116in}}%
\pgfpathlineto{\pgfqpoint{2.611172in}{4.481678in}}%
\pgfpathlineto{\pgfqpoint{2.612470in}{4.481678in}}%
\pgfpathlineto{\pgfqpoint{2.613767in}{2.930116in}}%
\pgfpathlineto{\pgfqpoint{2.615065in}{2.930116in}}%
\pgfpathlineto{\pgfqpoint{2.616362in}{3.447303in}}%
\pgfpathlineto{\pgfqpoint{2.620254in}{3.447303in}}%
\pgfpathlineto{\pgfqpoint{2.621552in}{3.964491in}}%
\pgfpathlineto{\pgfqpoint{2.622849in}{3.964491in}}%
\pgfpathlineto{\pgfqpoint{2.624147in}{2.930116in}}%
\pgfpathlineto{\pgfqpoint{2.630635in}{2.930116in}}%
\pgfpathlineto{\pgfqpoint{2.631932in}{3.447303in}}%
\pgfpathlineto{\pgfqpoint{2.633230in}{3.447303in}}%
\pgfpathlineto{\pgfqpoint{2.634527in}{3.964491in}}%
\pgfpathlineto{\pgfqpoint{2.638419in}{3.964491in}}%
\pgfpathlineto{\pgfqpoint{2.639717in}{3.447303in}}%
\pgfpathlineto{\pgfqpoint{2.643609in}{3.447303in}}%
\pgfpathlineto{\pgfqpoint{2.644907in}{4.481678in}}%
\pgfpathlineto{\pgfqpoint{2.646204in}{4.481678in}}%
\pgfpathlineto{\pgfqpoint{2.647502in}{2.930116in}}%
\pgfpathlineto{\pgfqpoint{2.648799in}{2.930116in}}%
\pgfpathlineto{\pgfqpoint{2.650097in}{4.481678in}}%
\pgfpathlineto{\pgfqpoint{2.653989in}{4.481678in}}%
\pgfpathlineto{\pgfqpoint{2.655287in}{3.964491in}}%
\pgfpathlineto{\pgfqpoint{2.656584in}{3.964491in}}%
\pgfpathlineto{\pgfqpoint{2.657882in}{2.930116in}}%
\pgfpathlineto{\pgfqpoint{2.661774in}{2.930116in}}%
\pgfpathlineto{\pgfqpoint{2.663072in}{3.447303in}}%
\pgfpathlineto{\pgfqpoint{2.664369in}{3.447303in}}%
\pgfpathlineto{\pgfqpoint{2.665667in}{4.481678in}}%
\pgfpathlineto{\pgfqpoint{2.666964in}{4.481678in}}%
\pgfpathlineto{\pgfqpoint{2.668262in}{3.447303in}}%
\pgfpathlineto{\pgfqpoint{2.669559in}{3.447303in}}%
\pgfpathlineto{\pgfqpoint{2.670857in}{3.964491in}}%
\pgfpathlineto{\pgfqpoint{2.677344in}{3.964491in}}%
\pgfpathlineto{\pgfqpoint{2.678642in}{2.930116in}}%
\pgfpathlineto{\pgfqpoint{2.679939in}{2.930116in}}%
\pgfpathlineto{\pgfqpoint{2.681237in}{3.964491in}}%
\pgfpathlineto{\pgfqpoint{2.682534in}{3.964491in}}%
\pgfpathlineto{\pgfqpoint{2.683832in}{4.481678in}}%
\pgfpathlineto{\pgfqpoint{2.685129in}{4.481678in}}%
\pgfpathlineto{\pgfqpoint{2.686427in}{3.964491in}}%
\pgfpathlineto{\pgfqpoint{2.687724in}{3.964491in}}%
\pgfpathlineto{\pgfqpoint{2.689022in}{2.930116in}}%
\pgfpathlineto{\pgfqpoint{2.690319in}{2.930116in}}%
\pgfpathlineto{\pgfqpoint{2.691617in}{4.481678in}}%
\pgfpathlineto{\pgfqpoint{2.692914in}{4.481678in}}%
\pgfpathlineto{\pgfqpoint{2.694212in}{3.964491in}}%
\pgfpathlineto{\pgfqpoint{2.695509in}{3.964491in}}%
\pgfpathlineto{\pgfqpoint{2.696807in}{2.930116in}}%
\pgfpathlineto{\pgfqpoint{2.698104in}{2.930116in}}%
\pgfpathlineto{\pgfqpoint{2.699402in}{4.481678in}}%
\pgfpathlineto{\pgfqpoint{2.700699in}{4.481678in}}%
\pgfpathlineto{\pgfqpoint{2.701997in}{3.964491in}}%
\pgfpathlineto{\pgfqpoint{2.705889in}{3.964491in}}%
\pgfpathlineto{\pgfqpoint{2.707187in}{3.447303in}}%
\pgfpathlineto{\pgfqpoint{2.708484in}{3.447303in}}%
\pgfpathlineto{\pgfqpoint{2.709782in}{4.481678in}}%
\pgfpathlineto{\pgfqpoint{2.711079in}{4.481678in}}%
\pgfpathlineto{\pgfqpoint{2.712377in}{3.447303in}}%
\pgfpathlineto{\pgfqpoint{2.713674in}{3.447303in}}%
\pgfpathlineto{\pgfqpoint{2.714972in}{2.930116in}}%
\pgfpathlineto{\pgfqpoint{2.716269in}{2.930116in}}%
\pgfpathlineto{\pgfqpoint{2.717567in}{3.447303in}}%
\pgfpathlineto{\pgfqpoint{2.718864in}{3.447303in}}%
\pgfpathlineto{\pgfqpoint{2.720162in}{2.930116in}}%
\pgfpathlineto{\pgfqpoint{2.721459in}{2.930116in}}%
\pgfpathlineto{\pgfqpoint{2.722757in}{3.964491in}}%
\pgfpathlineto{\pgfqpoint{2.724054in}{3.964491in}}%
\pgfpathlineto{\pgfqpoint{2.725352in}{3.447303in}}%
\pgfpathlineto{\pgfqpoint{2.726649in}{3.447303in}}%
\pgfpathlineto{\pgfqpoint{2.727947in}{4.481678in}}%
\pgfpathlineto{\pgfqpoint{2.729244in}{4.481678in}}%
\pgfpathlineto{\pgfqpoint{2.730542in}{3.447303in}}%
\pgfpathlineto{\pgfqpoint{2.731839in}{3.447303in}}%
\pgfpathlineto{\pgfqpoint{2.733137in}{2.930116in}}%
\pgfpathlineto{\pgfqpoint{2.734434in}{2.930116in}}%
\pgfpathlineto{\pgfqpoint{2.735732in}{4.481678in}}%
\pgfpathlineto{\pgfqpoint{2.737029in}{4.481678in}}%
\pgfpathlineto{\pgfqpoint{2.738327in}{3.964491in}}%
\pgfpathlineto{\pgfqpoint{2.739624in}{3.964491in}}%
\pgfpathlineto{\pgfqpoint{2.740922in}{2.930116in}}%
\pgfpathlineto{\pgfqpoint{2.742219in}{2.930116in}}%
\pgfpathlineto{\pgfqpoint{2.743517in}{3.964491in}}%
\pgfpathlineto{\pgfqpoint{2.744814in}{3.964491in}}%
\pgfpathlineto{\pgfqpoint{2.746112in}{4.481678in}}%
\pgfpathlineto{\pgfqpoint{2.752599in}{4.481678in}}%
\pgfpathlineto{\pgfqpoint{2.753897in}{3.447303in}}%
\pgfpathlineto{\pgfqpoint{2.757789in}{3.447303in}}%
\pgfpathlineto{\pgfqpoint{2.759087in}{2.930116in}}%
\pgfpathlineto{\pgfqpoint{2.762979in}{2.930116in}}%
\pgfpathlineto{\pgfqpoint{2.764277in}{3.964491in}}%
\pgfpathlineto{\pgfqpoint{2.765574in}{3.964491in}}%
\pgfpathlineto{\pgfqpoint{2.766872in}{2.930116in}}%
\pgfpathlineto{\pgfqpoint{2.770764in}{2.930116in}}%
\pgfpathlineto{\pgfqpoint{2.772062in}{3.447303in}}%
\pgfpathlineto{\pgfqpoint{2.773359in}{3.447303in}}%
\pgfpathlineto{\pgfqpoint{2.774657in}{3.964491in}}%
\pgfpathlineto{\pgfqpoint{2.778549in}{3.964491in}}%
\pgfpathlineto{\pgfqpoint{2.779847in}{3.447303in}}%
\pgfpathlineto{\pgfqpoint{2.783739in}{3.447303in}}%
\pgfpathlineto{\pgfqpoint{2.785037in}{2.930116in}}%
\pgfpathlineto{\pgfqpoint{2.786334in}{2.930116in}}%
\pgfpathlineto{\pgfqpoint{2.787632in}{3.964491in}}%
\pgfpathlineto{\pgfqpoint{2.788929in}{3.964491in}}%
\pgfpathlineto{\pgfqpoint{2.790227in}{4.481678in}}%
\pgfpathlineto{\pgfqpoint{2.791524in}{4.481678in}}%
\pgfpathlineto{\pgfqpoint{2.792822in}{2.930116in}}%
\pgfpathlineto{\pgfqpoint{2.796714in}{2.930116in}}%
\pgfpathlineto{\pgfqpoint{2.798012in}{4.481678in}}%
\pgfpathlineto{\pgfqpoint{2.801904in}{4.481678in}}%
\pgfpathlineto{\pgfqpoint{2.803202in}{3.447303in}}%
\pgfpathlineto{\pgfqpoint{2.804499in}{3.447303in}}%
\pgfpathlineto{\pgfqpoint{2.805797in}{4.481678in}}%
\pgfpathlineto{\pgfqpoint{2.807094in}{4.481678in}}%
\pgfpathlineto{\pgfqpoint{2.808392in}{3.447303in}}%
\pgfpathlineto{\pgfqpoint{2.809689in}{3.447303in}}%
\pgfpathlineto{\pgfqpoint{2.810987in}{3.964491in}}%
\pgfpathlineto{\pgfqpoint{2.812284in}{3.964491in}}%
\pgfpathlineto{\pgfqpoint{2.813582in}{3.447303in}}%
\pgfpathlineto{\pgfqpoint{2.820069in}{3.447303in}}%
\pgfpathlineto{\pgfqpoint{2.821367in}{4.481678in}}%
\pgfpathlineto{\pgfqpoint{2.822664in}{4.481678in}}%
\pgfpathlineto{\pgfqpoint{2.823962in}{3.964491in}}%
\pgfpathlineto{\pgfqpoint{2.825259in}{3.964491in}}%
\pgfpathlineto{\pgfqpoint{2.826557in}{2.930116in}}%
\pgfpathlineto{\pgfqpoint{2.830449in}{2.930116in}}%
\pgfpathlineto{\pgfqpoint{2.831747in}{4.481678in}}%
\pgfpathlineto{\pgfqpoint{2.833044in}{4.481678in}}%
\pgfpathlineto{\pgfqpoint{2.834342in}{3.447303in}}%
\pgfpathlineto{\pgfqpoint{2.835639in}{3.447303in}}%
\pgfpathlineto{\pgfqpoint{2.836937in}{3.964491in}}%
\pgfpathlineto{\pgfqpoint{2.840829in}{3.964491in}}%
\pgfpathlineto{\pgfqpoint{2.842127in}{2.930116in}}%
\pgfpathlineto{\pgfqpoint{2.843424in}{2.930116in}}%
\pgfpathlineto{\pgfqpoint{2.844722in}{4.481678in}}%
\pgfpathlineto{\pgfqpoint{2.848614in}{4.481678in}}%
\pgfpathlineto{\pgfqpoint{2.849912in}{3.964491in}}%
\pgfpathlineto{\pgfqpoint{2.851209in}{3.964491in}}%
\pgfpathlineto{\pgfqpoint{2.852507in}{3.447303in}}%
\pgfpathlineto{\pgfqpoint{2.853804in}{3.447303in}}%
\pgfpathlineto{\pgfqpoint{2.855102in}{3.964491in}}%
\pgfpathlineto{\pgfqpoint{2.858994in}{3.964491in}}%
\pgfpathlineto{\pgfqpoint{2.860292in}{2.930116in}}%
\pgfpathlineto{\pgfqpoint{2.866779in}{2.930116in}}%
\pgfpathlineto{\pgfqpoint{2.868077in}{4.481678in}}%
\pgfpathlineto{\pgfqpoint{2.869374in}{4.481678in}}%
\pgfpathlineto{\pgfqpoint{2.870672in}{3.964491in}}%
\pgfpathlineto{\pgfqpoint{2.871969in}{3.964491in}}%
\pgfpathlineto{\pgfqpoint{2.873267in}{4.481678in}}%
\pgfpathlineto{\pgfqpoint{2.877159in}{4.481678in}}%
\pgfpathlineto{\pgfqpoint{2.878457in}{2.930116in}}%
\pgfpathlineto{\pgfqpoint{2.884944in}{2.930116in}}%
\pgfpathlineto{\pgfqpoint{2.886242in}{3.447303in}}%
\pgfpathlineto{\pgfqpoint{2.890134in}{3.447303in}}%
\pgfpathlineto{\pgfqpoint{2.891432in}{2.930116in}}%
\pgfpathlineto{\pgfqpoint{2.892729in}{2.930116in}}%
\pgfpathlineto{\pgfqpoint{2.894026in}{4.481678in}}%
\pgfpathlineto{\pgfqpoint{2.895324in}{4.481678in}}%
\pgfpathlineto{\pgfqpoint{2.896621in}{2.930116in}}%
\pgfpathlineto{\pgfqpoint{2.897919in}{2.930116in}}%
\pgfpathlineto{\pgfqpoint{2.899217in}{3.447303in}}%
\pgfpathlineto{\pgfqpoint{2.900514in}{3.447303in}}%
\pgfpathlineto{\pgfqpoint{2.901812in}{3.964491in}}%
\pgfpathlineto{\pgfqpoint{2.903109in}{3.964491in}}%
\pgfpathlineto{\pgfqpoint{2.904407in}{2.930116in}}%
\pgfpathlineto{\pgfqpoint{2.905704in}{2.930116in}}%
\pgfpathlineto{\pgfqpoint{2.907002in}{4.481678in}}%
\pgfpathlineto{\pgfqpoint{2.908299in}{4.481678in}}%
\pgfpathlineto{\pgfqpoint{2.909597in}{2.930116in}}%
\pgfpathlineto{\pgfqpoint{2.913489in}{2.930116in}}%
\pgfpathlineto{\pgfqpoint{2.914786in}{4.481678in}}%
\pgfpathlineto{\pgfqpoint{2.916084in}{4.481678in}}%
\pgfpathlineto{\pgfqpoint{2.917381in}{2.930116in}}%
\pgfpathlineto{\pgfqpoint{2.918679in}{2.930116in}}%
\pgfpathlineto{\pgfqpoint{2.919976in}{3.447303in}}%
\pgfpathlineto{\pgfqpoint{2.926464in}{3.447303in}}%
\pgfpathlineto{\pgfqpoint{2.927762in}{4.481678in}}%
\pgfpathlineto{\pgfqpoint{2.929059in}{4.481678in}}%
\pgfpathlineto{\pgfqpoint{2.930357in}{3.964491in}}%
\pgfpathlineto{\pgfqpoint{2.931654in}{3.964491in}}%
\pgfpathlineto{\pgfqpoint{2.932951in}{2.930116in}}%
\pgfpathlineto{\pgfqpoint{2.934249in}{2.930116in}}%
\pgfpathlineto{\pgfqpoint{2.935546in}{3.447303in}}%
\pgfpathlineto{\pgfqpoint{2.936844in}{3.447303in}}%
\pgfpathlineto{\pgfqpoint{2.938141in}{2.930116in}}%
\pgfpathlineto{\pgfqpoint{2.939439in}{2.930116in}}%
\pgfpathlineto{\pgfqpoint{2.940736in}{4.481678in}}%
\pgfpathlineto{\pgfqpoint{2.942034in}{4.481678in}}%
\pgfpathlineto{\pgfqpoint{2.943331in}{2.930116in}}%
\pgfpathlineto{\pgfqpoint{2.944629in}{2.930116in}}%
\pgfpathlineto{\pgfqpoint{2.945926in}{3.964491in}}%
\pgfpathlineto{\pgfqpoint{2.947224in}{3.964491in}}%
\pgfpathlineto{\pgfqpoint{2.948522in}{2.930116in}}%
\pgfpathlineto{\pgfqpoint{2.949819in}{2.930116in}}%
\pgfpathlineto{\pgfqpoint{2.951117in}{3.964491in}}%
\pgfpathlineto{\pgfqpoint{2.952414in}{3.964491in}}%
\pgfpathlineto{\pgfqpoint{2.953711in}{3.447303in}}%
\pgfpathlineto{\pgfqpoint{2.957604in}{3.447303in}}%
\pgfpathlineto{\pgfqpoint{2.958901in}{3.964491in}}%
\pgfpathlineto{\pgfqpoint{2.960199in}{3.964491in}}%
\pgfpathlineto{\pgfqpoint{2.961496in}{2.930116in}}%
\pgfpathlineto{\pgfqpoint{2.965389in}{2.930116in}}%
\pgfpathlineto{\pgfqpoint{2.966686in}{3.447303in}}%
\pgfpathlineto{\pgfqpoint{2.967984in}{3.447303in}}%
\pgfpathlineto{\pgfqpoint{2.969281in}{2.930116in}}%
\pgfpathlineto{\pgfqpoint{2.970579in}{2.930116in}}%
\pgfpathlineto{\pgfqpoint{2.971876in}{3.447303in}}%
\pgfpathlineto{\pgfqpoint{2.973174in}{3.447303in}}%
\pgfpathlineto{\pgfqpoint{2.974471in}{2.930116in}}%
\pgfpathlineto{\pgfqpoint{2.975769in}{2.930116in}}%
\pgfpathlineto{\pgfqpoint{2.977066in}{3.447303in}}%
\pgfpathlineto{\pgfqpoint{2.986149in}{3.447303in}}%
\pgfpathlineto{\pgfqpoint{2.987446in}{4.481678in}}%
\pgfpathlineto{\pgfqpoint{2.988744in}{4.481678in}}%
\pgfpathlineto{\pgfqpoint{2.990041in}{3.447303in}}%
\pgfpathlineto{\pgfqpoint{2.993934in}{3.447303in}}%
\pgfpathlineto{\pgfqpoint{2.995231in}{3.964491in}}%
\pgfpathlineto{\pgfqpoint{2.996529in}{3.964491in}}%
\pgfpathlineto{\pgfqpoint{2.997826in}{4.481678in}}%
\pgfpathlineto{\pgfqpoint{3.006909in}{4.481678in}}%
\pgfpathlineto{\pgfqpoint{3.008206in}{2.930116in}}%
\pgfpathlineto{\pgfqpoint{3.012099in}{2.930116in}}%
\pgfpathlineto{\pgfqpoint{3.013396in}{3.964491in}}%
\pgfpathlineto{\pgfqpoint{3.014694in}{3.964491in}}%
\pgfpathlineto{\pgfqpoint{3.015991in}{3.447303in}}%
\pgfpathlineto{\pgfqpoint{3.017289in}{3.447303in}}%
\pgfpathlineto{\pgfqpoint{3.018586in}{3.964491in}}%
\pgfpathlineto{\pgfqpoint{3.022479in}{3.964491in}}%
\pgfpathlineto{\pgfqpoint{3.023776in}{2.930116in}}%
\pgfpathlineto{\pgfqpoint{3.025074in}{2.930116in}}%
\pgfpathlineto{\pgfqpoint{3.026371in}{4.481678in}}%
\pgfpathlineto{\pgfqpoint{3.027669in}{4.481678in}}%
\pgfpathlineto{\pgfqpoint{3.028966in}{2.930116in}}%
\pgfpathlineto{\pgfqpoint{3.030264in}{2.930116in}}%
\pgfpathlineto{\pgfqpoint{3.031561in}{4.481678in}}%
\pgfpathlineto{\pgfqpoint{3.032859in}{4.481678in}}%
\pgfpathlineto{\pgfqpoint{3.034156in}{3.447303in}}%
\pgfpathlineto{\pgfqpoint{3.035454in}{3.447303in}}%
\pgfpathlineto{\pgfqpoint{3.036751in}{2.930116in}}%
\pgfpathlineto{\pgfqpoint{3.040644in}{2.930116in}}%
\pgfpathlineto{\pgfqpoint{3.041941in}{3.447303in}}%
\pgfpathlineto{\pgfqpoint{3.048429in}{3.447303in}}%
\pgfpathlineto{\pgfqpoint{3.049726in}{4.481678in}}%
\pgfpathlineto{\pgfqpoint{3.051024in}{4.481678in}}%
\pgfpathlineto{\pgfqpoint{3.052321in}{3.964491in}}%
\pgfpathlineto{\pgfqpoint{3.053619in}{3.964491in}}%
\pgfpathlineto{\pgfqpoint{3.054916in}{3.447303in}}%
\pgfpathlineto{\pgfqpoint{3.061404in}{3.447303in}}%
\pgfpathlineto{\pgfqpoint{3.062701in}{3.964491in}}%
\pgfpathlineto{\pgfqpoint{3.066594in}{3.964491in}}%
\pgfpathlineto{\pgfqpoint{3.067891in}{3.447303in}}%
\pgfpathlineto{\pgfqpoint{3.069189in}{3.447303in}}%
\pgfpathlineto{\pgfqpoint{3.070486in}{2.930116in}}%
\pgfpathlineto{\pgfqpoint{3.071784in}{2.930116in}}%
\pgfpathlineto{\pgfqpoint{3.073081in}{3.447303in}}%
\pgfpathlineto{\pgfqpoint{3.074379in}{3.447303in}}%
\pgfpathlineto{\pgfqpoint{3.075676in}{4.481678in}}%
\pgfpathlineto{\pgfqpoint{3.076974in}{4.481678in}}%
\pgfpathlineto{\pgfqpoint{3.078271in}{3.447303in}}%
\pgfpathlineto{\pgfqpoint{3.079569in}{3.447303in}}%
\pgfpathlineto{\pgfqpoint{3.080866in}{2.930116in}}%
\pgfpathlineto{\pgfqpoint{3.084759in}{2.930116in}}%
\pgfpathlineto{\pgfqpoint{3.086056in}{3.447303in}}%
\pgfpathlineto{\pgfqpoint{3.087354in}{3.447303in}}%
\pgfpathlineto{\pgfqpoint{3.088651in}{3.964491in}}%
\pgfpathlineto{\pgfqpoint{3.089949in}{3.964491in}}%
\pgfpathlineto{\pgfqpoint{3.091246in}{3.447303in}}%
\pgfpathlineto{\pgfqpoint{3.092544in}{3.447303in}}%
\pgfpathlineto{\pgfqpoint{3.093841in}{2.930116in}}%
\pgfpathlineto{\pgfqpoint{3.095139in}{2.930116in}}%
\pgfpathlineto{\pgfqpoint{3.096436in}{4.481678in}}%
\pgfpathlineto{\pgfqpoint{3.097734in}{4.481678in}}%
\pgfpathlineto{\pgfqpoint{3.099031in}{3.447303in}}%
\pgfpathlineto{\pgfqpoint{3.100329in}{3.447303in}}%
\pgfpathlineto{\pgfqpoint{3.101626in}{4.481678in}}%
\pgfpathlineto{\pgfqpoint{3.102924in}{4.481678in}}%
\pgfpathlineto{\pgfqpoint{3.104221in}{3.964491in}}%
\pgfpathlineto{\pgfqpoint{3.105519in}{3.964491in}}%
\pgfpathlineto{\pgfqpoint{3.106816in}{4.481678in}}%
\pgfpathlineto{\pgfqpoint{3.108114in}{4.481678in}}%
\pgfpathlineto{\pgfqpoint{3.109411in}{3.447303in}}%
\pgfpathlineto{\pgfqpoint{3.113304in}{3.447303in}}%
\pgfpathlineto{\pgfqpoint{3.114601in}{2.930116in}}%
\pgfpathlineto{\pgfqpoint{3.115899in}{2.930116in}}%
\pgfpathlineto{\pgfqpoint{3.117196in}{3.964491in}}%
\pgfpathlineto{\pgfqpoint{3.118494in}{3.964491in}}%
\pgfpathlineto{\pgfqpoint{3.119791in}{4.481678in}}%
\pgfpathlineto{\pgfqpoint{3.121089in}{4.481678in}}%
\pgfpathlineto{\pgfqpoint{3.122386in}{3.447303in}}%
\pgfpathlineto{\pgfqpoint{3.126279in}{3.447303in}}%
\pgfpathlineto{\pgfqpoint{3.127576in}{3.964491in}}%
\pgfpathlineto{\pgfqpoint{3.128874in}{3.964491in}}%
\pgfpathlineto{\pgfqpoint{3.130171in}{4.481678in}}%
\pgfpathlineto{\pgfqpoint{3.134064in}{4.481678in}}%
\pgfpathlineto{\pgfqpoint{3.135361in}{3.964491in}}%
\pgfpathlineto{\pgfqpoint{3.139254in}{3.964491in}}%
\pgfpathlineto{\pgfqpoint{3.140551in}{3.447303in}}%
\pgfpathlineto{\pgfqpoint{3.144444in}{3.447303in}}%
\pgfpathlineto{\pgfqpoint{3.145741in}{2.930116in}}%
\pgfpathlineto{\pgfqpoint{3.147039in}{2.930116in}}%
\pgfpathlineto{\pgfqpoint{3.148336in}{4.481678in}}%
\pgfpathlineto{\pgfqpoint{3.152229in}{4.481678in}}%
\pgfpathlineto{\pgfqpoint{3.153526in}{3.447303in}}%
\pgfpathlineto{\pgfqpoint{3.154824in}{3.447303in}}%
\pgfpathlineto{\pgfqpoint{3.156121in}{3.964491in}}%
\pgfpathlineto{\pgfqpoint{3.157419in}{3.964491in}}%
\pgfpathlineto{\pgfqpoint{3.158716in}{3.447303in}}%
\pgfpathlineto{\pgfqpoint{3.165204in}{3.447303in}}%
\pgfpathlineto{\pgfqpoint{3.166501in}{4.481678in}}%
\pgfpathlineto{\pgfqpoint{3.170394in}{4.481678in}}%
\pgfpathlineto{\pgfqpoint{3.171691in}{2.930116in}}%
\pgfpathlineto{\pgfqpoint{3.175584in}{2.930116in}}%
\pgfpathlineto{\pgfqpoint{3.176881in}{3.964491in}}%
\pgfpathlineto{\pgfqpoint{3.178179in}{3.964491in}}%
\pgfpathlineto{\pgfqpoint{3.179476in}{4.481678in}}%
\pgfpathlineto{\pgfqpoint{3.180774in}{4.481678in}}%
\pgfpathlineto{\pgfqpoint{3.182071in}{3.447303in}}%
\pgfpathlineto{\pgfqpoint{3.183369in}{3.447303in}}%
\pgfpathlineto{\pgfqpoint{3.184666in}{2.930116in}}%
\pgfpathlineto{\pgfqpoint{3.185964in}{2.930116in}}%
\pgfpathlineto{\pgfqpoint{3.187261in}{3.964491in}}%
\pgfpathlineto{\pgfqpoint{3.191154in}{3.964491in}}%
\pgfpathlineto{\pgfqpoint{3.192451in}{3.447303in}}%
\pgfpathlineto{\pgfqpoint{3.196344in}{3.447303in}}%
\pgfpathlineto{\pgfqpoint{3.197641in}{2.930116in}}%
\pgfpathlineto{\pgfqpoint{3.198939in}{2.930116in}}%
\pgfpathlineto{\pgfqpoint{3.200236in}{3.964491in}}%
\pgfpathlineto{\pgfqpoint{3.201534in}{3.964491in}}%
\pgfpathlineto{\pgfqpoint{3.202831in}{4.481678in}}%
\pgfpathlineto{\pgfqpoint{3.204129in}{4.481678in}}%
\pgfpathlineto{\pgfqpoint{3.205426in}{3.447303in}}%
\pgfpathlineto{\pgfqpoint{3.206724in}{3.447303in}}%
\pgfpathlineto{\pgfqpoint{3.208021in}{3.964491in}}%
\pgfpathlineto{\pgfqpoint{3.214508in}{3.964491in}}%
\pgfpathlineto{\pgfqpoint{3.215806in}{2.930116in}}%
\pgfpathlineto{\pgfqpoint{3.217104in}{2.930116in}}%
\pgfpathlineto{\pgfqpoint{3.218401in}{3.964491in}}%
\pgfpathlineto{\pgfqpoint{3.222294in}{3.964491in}}%
\pgfpathlineto{\pgfqpoint{3.223591in}{2.930116in}}%
\pgfpathlineto{\pgfqpoint{3.224889in}{2.930116in}}%
\pgfpathlineto{\pgfqpoint{3.226186in}{3.964491in}}%
\pgfpathlineto{\pgfqpoint{3.227484in}{3.964491in}}%
\pgfpathlineto{\pgfqpoint{3.228781in}{4.481678in}}%
\pgfpathlineto{\pgfqpoint{3.230079in}{4.481678in}}%
\pgfpathlineto{\pgfqpoint{3.231376in}{2.930116in}}%
\pgfpathlineto{\pgfqpoint{3.232673in}{2.930116in}}%
\pgfpathlineto{\pgfqpoint{3.233971in}{3.964491in}}%
\pgfpathlineto{\pgfqpoint{3.235268in}{3.964491in}}%
\pgfpathlineto{\pgfqpoint{3.236566in}{2.930116in}}%
\pgfpathlineto{\pgfqpoint{3.237863in}{2.930116in}}%
\pgfpathlineto{\pgfqpoint{3.239161in}{3.964491in}}%
\pgfpathlineto{\pgfqpoint{3.240458in}{3.964491in}}%
\pgfpathlineto{\pgfqpoint{3.241756in}{3.447303in}}%
\pgfpathlineto{\pgfqpoint{3.248244in}{3.447303in}}%
\pgfpathlineto{\pgfqpoint{3.249541in}{3.964491in}}%
\pgfpathlineto{\pgfqpoint{3.253433in}{3.964491in}}%
\pgfpathlineto{\pgfqpoint{3.254731in}{4.481678in}}%
\pgfpathlineto{\pgfqpoint{3.256028in}{4.481678in}}%
\pgfpathlineto{\pgfqpoint{3.257326in}{3.964491in}}%
\pgfpathlineto{\pgfqpoint{3.258623in}{3.964491in}}%
\pgfpathlineto{\pgfqpoint{3.259921in}{4.481678in}}%
\pgfpathlineto{\pgfqpoint{3.261218in}{4.481678in}}%
\pgfpathlineto{\pgfqpoint{3.262516in}{3.447303in}}%
\pgfpathlineto{\pgfqpoint{3.263813in}{3.447303in}}%
\pgfpathlineto{\pgfqpoint{3.265111in}{3.964491in}}%
\pgfpathlineto{\pgfqpoint{3.266409in}{3.964491in}}%
\pgfpathlineto{\pgfqpoint{3.267706in}{4.481678in}}%
\pgfpathlineto{\pgfqpoint{3.271598in}{4.481678in}}%
\pgfpathlineto{\pgfqpoint{3.272896in}{3.964491in}}%
\pgfpathlineto{\pgfqpoint{3.276788in}{3.964491in}}%
\pgfpathlineto{\pgfqpoint{3.278086in}{3.447303in}}%
\pgfpathlineto{\pgfqpoint{3.279383in}{3.447303in}}%
\pgfpathlineto{\pgfqpoint{3.280681in}{4.481678in}}%
\pgfpathlineto{\pgfqpoint{3.289763in}{4.481678in}}%
\pgfpathlineto{\pgfqpoint{3.291061in}{3.447303in}}%
\pgfpathlineto{\pgfqpoint{3.297548in}{3.447303in}}%
\pgfpathlineto{\pgfqpoint{3.298846in}{3.964491in}}%
\pgfpathlineto{\pgfqpoint{3.302738in}{3.964491in}}%
\pgfpathlineto{\pgfqpoint{3.304036in}{3.447303in}}%
\pgfpathlineto{\pgfqpoint{3.307928in}{3.447303in}}%
\pgfpathlineto{\pgfqpoint{3.309226in}{2.930116in}}%
\pgfpathlineto{\pgfqpoint{3.310523in}{2.930116in}}%
\pgfpathlineto{\pgfqpoint{3.311821in}{3.964491in}}%
\pgfpathlineto{\pgfqpoint{3.313118in}{3.964491in}}%
\pgfpathlineto{\pgfqpoint{3.314416in}{4.481678in}}%
\pgfpathlineto{\pgfqpoint{3.315713in}{4.481678in}}%
\pgfpathlineto{\pgfqpoint{3.317011in}{3.964491in}}%
\pgfpathlineto{\pgfqpoint{3.318308in}{3.964491in}}%
\pgfpathlineto{\pgfqpoint{3.319606in}{3.447303in}}%
\pgfpathlineto{\pgfqpoint{3.323498in}{3.447303in}}%
\pgfpathlineto{\pgfqpoint{3.324796in}{4.481678in}}%
\pgfpathlineto{\pgfqpoint{3.328688in}{4.481678in}}%
\pgfpathlineto{\pgfqpoint{3.329986in}{3.447303in}}%
\pgfpathlineto{\pgfqpoint{3.333878in}{3.447303in}}%
\pgfpathlineto{\pgfqpoint{3.335176in}{3.964491in}}%
\pgfpathlineto{\pgfqpoint{3.336473in}{3.964491in}}%
\pgfpathlineto{\pgfqpoint{3.337771in}{4.481678in}}%
\pgfpathlineto{\pgfqpoint{3.339068in}{4.481678in}}%
\pgfpathlineto{\pgfqpoint{3.340366in}{3.447303in}}%
\pgfpathlineto{\pgfqpoint{3.341663in}{3.447303in}}%
\pgfpathlineto{\pgfqpoint{3.342961in}{4.481678in}}%
\pgfpathlineto{\pgfqpoint{3.344258in}{4.481678in}}%
\pgfpathlineto{\pgfqpoint{3.345556in}{3.964491in}}%
\pgfpathlineto{\pgfqpoint{3.346853in}{3.964491in}}%
\pgfpathlineto{\pgfqpoint{3.348151in}{2.930116in}}%
\pgfpathlineto{\pgfqpoint{3.349448in}{2.930116in}}%
\pgfpathlineto{\pgfqpoint{3.350746in}{3.964491in}}%
\pgfpathlineto{\pgfqpoint{3.357233in}{3.964491in}}%
\pgfpathlineto{\pgfqpoint{3.358531in}{3.447303in}}%
\pgfpathlineto{\pgfqpoint{3.362423in}{3.447303in}}%
\pgfpathlineto{\pgfqpoint{3.363721in}{2.930116in}}%
\pgfpathlineto{\pgfqpoint{3.367613in}{2.930116in}}%
\pgfpathlineto{\pgfqpoint{3.368911in}{4.481678in}}%
\pgfpathlineto{\pgfqpoint{3.370208in}{4.481678in}}%
\pgfpathlineto{\pgfqpoint{3.371506in}{3.964491in}}%
\pgfpathlineto{\pgfqpoint{3.372803in}{3.964491in}}%
\pgfpathlineto{\pgfqpoint{3.374101in}{2.930116in}}%
\pgfpathlineto{\pgfqpoint{3.375398in}{2.930116in}}%
\pgfpathlineto{\pgfqpoint{3.376696in}{3.964491in}}%
\pgfpathlineto{\pgfqpoint{3.385778in}{3.964491in}}%
\pgfpathlineto{\pgfqpoint{3.387076in}{2.930116in}}%
\pgfpathlineto{\pgfqpoint{3.388373in}{2.930116in}}%
\pgfpathlineto{\pgfqpoint{3.389671in}{3.447303in}}%
\pgfpathlineto{\pgfqpoint{3.390968in}{3.447303in}}%
\pgfpathlineto{\pgfqpoint{3.392266in}{3.964491in}}%
\pgfpathlineto{\pgfqpoint{3.393563in}{3.964491in}}%
\pgfpathlineto{\pgfqpoint{3.394861in}{4.481678in}}%
\pgfpathlineto{\pgfqpoint{3.396158in}{4.481678in}}%
\pgfpathlineto{\pgfqpoint{3.397456in}{3.964491in}}%
\pgfpathlineto{\pgfqpoint{3.398753in}{3.964491in}}%
\pgfpathlineto{\pgfqpoint{3.400051in}{4.481678in}}%
\pgfpathlineto{\pgfqpoint{3.401348in}{4.481678in}}%
\pgfpathlineto{\pgfqpoint{3.402646in}{2.930116in}}%
\pgfpathlineto{\pgfqpoint{3.403943in}{2.930116in}}%
\pgfpathlineto{\pgfqpoint{3.405241in}{3.447303in}}%
\pgfpathlineto{\pgfqpoint{3.406538in}{3.447303in}}%
\pgfpathlineto{\pgfqpoint{3.407836in}{3.964491in}}%
\pgfpathlineto{\pgfqpoint{3.409133in}{3.964491in}}%
\pgfpathlineto{\pgfqpoint{3.410431in}{4.481678in}}%
\pgfpathlineto{\pgfqpoint{3.411728in}{4.481678in}}%
\pgfpathlineto{\pgfqpoint{3.413026in}{2.930116in}}%
\pgfpathlineto{\pgfqpoint{3.416918in}{2.930116in}}%
\pgfpathlineto{\pgfqpoint{3.418216in}{4.481678in}}%
\pgfpathlineto{\pgfqpoint{3.419513in}{4.481678in}}%
\pgfpathlineto{\pgfqpoint{3.420811in}{3.964491in}}%
\pgfpathlineto{\pgfqpoint{3.424703in}{3.964491in}}%
\pgfpathlineto{\pgfqpoint{3.426001in}{2.930116in}}%
\pgfpathlineto{\pgfqpoint{3.427298in}{2.930116in}}%
\pgfpathlineto{\pgfqpoint{3.428596in}{4.481678in}}%
\pgfpathlineto{\pgfqpoint{3.432488in}{4.481678in}}%
\pgfpathlineto{\pgfqpoint{3.433786in}{3.447303in}}%
\pgfpathlineto{\pgfqpoint{3.435083in}{3.447303in}}%
\pgfpathlineto{\pgfqpoint{3.436381in}{4.481678in}}%
\pgfpathlineto{\pgfqpoint{3.437678in}{4.481678in}}%
\pgfpathlineto{\pgfqpoint{3.438976in}{3.447303in}}%
\pgfpathlineto{\pgfqpoint{3.440273in}{3.447303in}}%
\pgfpathlineto{\pgfqpoint{3.441571in}{2.930116in}}%
\pgfpathlineto{\pgfqpoint{3.442868in}{2.930116in}}%
\pgfpathlineto{\pgfqpoint{3.444166in}{3.447303in}}%
\pgfpathlineto{\pgfqpoint{3.448058in}{3.447303in}}%
\pgfpathlineto{\pgfqpoint{3.449356in}{3.964491in}}%
\pgfpathlineto{\pgfqpoint{3.450653in}{3.964491in}}%
\pgfpathlineto{\pgfqpoint{3.451951in}{2.930116in}}%
\pgfpathlineto{\pgfqpoint{3.453248in}{2.930116in}}%
\pgfpathlineto{\pgfqpoint{3.454546in}{4.481678in}}%
\pgfpathlineto{\pgfqpoint{3.455843in}{4.481678in}}%
\pgfpathlineto{\pgfqpoint{3.457141in}{3.447303in}}%
\pgfpathlineto{\pgfqpoint{3.458438in}{3.447303in}}%
\pgfpathlineto{\pgfqpoint{3.459736in}{4.481678in}}%
\pgfpathlineto{\pgfqpoint{3.461033in}{4.481678in}}%
\pgfpathlineto{\pgfqpoint{3.462331in}{3.447303in}}%
\pgfpathlineto{\pgfqpoint{3.463628in}{3.447303in}}%
\pgfpathlineto{\pgfqpoint{3.464926in}{4.481678in}}%
\pgfpathlineto{\pgfqpoint{3.466223in}{4.481678in}}%
\pgfpathlineto{\pgfqpoint{3.467521in}{3.447303in}}%
\pgfpathlineto{\pgfqpoint{3.471413in}{3.447303in}}%
\pgfpathlineto{\pgfqpoint{3.472711in}{4.481678in}}%
\pgfpathlineto{\pgfqpoint{3.474008in}{4.481678in}}%
\pgfpathlineto{\pgfqpoint{3.475306in}{3.964491in}}%
\pgfpathlineto{\pgfqpoint{3.476603in}{3.964491in}}%
\pgfpathlineto{\pgfqpoint{3.477901in}{2.930116in}}%
\pgfpathlineto{\pgfqpoint{3.484388in}{2.930116in}}%
\pgfpathlineto{\pgfqpoint{3.485686in}{4.481678in}}%
\pgfpathlineto{\pgfqpoint{3.492173in}{4.481678in}}%
\pgfpathlineto{\pgfqpoint{3.493471in}{3.447303in}}%
\pgfpathlineto{\pgfqpoint{3.502553in}{3.447303in}}%
\pgfpathlineto{\pgfqpoint{3.503851in}{4.481678in}}%
\pgfpathlineto{\pgfqpoint{3.507743in}{4.481678in}}%
\pgfpathlineto{\pgfqpoint{3.509041in}{3.447303in}}%
\pgfpathlineto{\pgfqpoint{3.510338in}{3.447303in}}%
\pgfpathlineto{\pgfqpoint{3.511636in}{4.481678in}}%
\pgfpathlineto{\pgfqpoint{3.512933in}{4.481678in}}%
\pgfpathlineto{\pgfqpoint{3.514231in}{2.930116in}}%
\pgfpathlineto{\pgfqpoint{3.520718in}{2.930116in}}%
\pgfpathlineto{\pgfqpoint{3.522015in}{3.964491in}}%
\pgfpathlineto{\pgfqpoint{3.523313in}{3.964491in}}%
\pgfpathlineto{\pgfqpoint{3.524610in}{3.447303in}}%
\pgfpathlineto{\pgfqpoint{3.525908in}{3.447303in}}%
\pgfpathlineto{\pgfqpoint{3.527205in}{4.481678in}}%
\pgfpathlineto{\pgfqpoint{3.528503in}{4.481678in}}%
\pgfpathlineto{\pgfqpoint{3.529800in}{3.964491in}}%
\pgfpathlineto{\pgfqpoint{3.531098in}{3.964491in}}%
\pgfpathlineto{\pgfqpoint{3.532395in}{3.447303in}}%
\pgfpathlineto{\pgfqpoint{3.533693in}{3.447303in}}%
\pgfpathlineto{\pgfqpoint{3.534991in}{4.481678in}}%
\pgfpathlineto{\pgfqpoint{3.536288in}{4.481678in}}%
\pgfpathlineto{\pgfqpoint{3.537586in}{3.964491in}}%
\pgfpathlineto{\pgfqpoint{3.538883in}{3.964491in}}%
\pgfpathlineto{\pgfqpoint{3.540181in}{2.930116in}}%
\pgfpathlineto{\pgfqpoint{3.541478in}{2.930116in}}%
\pgfpathlineto{\pgfqpoint{3.542776in}{4.481678in}}%
\pgfpathlineto{\pgfqpoint{3.544073in}{4.481678in}}%
\pgfpathlineto{\pgfqpoint{3.545371in}{3.964491in}}%
\pgfpathlineto{\pgfqpoint{3.549263in}{3.964491in}}%
\pgfpathlineto{\pgfqpoint{3.550561in}{4.481678in}}%
\pgfpathlineto{\pgfqpoint{3.551858in}{4.481678in}}%
\pgfpathlineto{\pgfqpoint{3.553156in}{3.964491in}}%
\pgfpathlineto{\pgfqpoint{3.554453in}{3.964491in}}%
\pgfpathlineto{\pgfqpoint{3.555751in}{3.447303in}}%
\pgfpathlineto{\pgfqpoint{3.562238in}{3.447303in}}%
\pgfpathlineto{\pgfqpoint{3.563535in}{3.964491in}}%
\pgfpathlineto{\pgfqpoint{3.567428in}{3.964491in}}%
\pgfpathlineto{\pgfqpoint{3.568725in}{4.481678in}}%
\pgfpathlineto{\pgfqpoint{3.570023in}{4.481678in}}%
\pgfpathlineto{\pgfqpoint{3.571320in}{3.964491in}}%
\pgfpathlineto{\pgfqpoint{3.572618in}{3.964491in}}%
\pgfpathlineto{\pgfqpoint{3.573915in}{3.447303in}}%
\pgfpathlineto{\pgfqpoint{3.575213in}{3.447303in}}%
\pgfpathlineto{\pgfqpoint{3.576510in}{2.930116in}}%
\pgfpathlineto{\pgfqpoint{3.580403in}{2.930116in}}%
\pgfpathlineto{\pgfqpoint{3.581700in}{4.481678in}}%
\pgfpathlineto{\pgfqpoint{3.582998in}{4.481678in}}%
\pgfpathlineto{\pgfqpoint{3.584296in}{3.447303in}}%
\pgfpathlineto{\pgfqpoint{3.585593in}{3.447303in}}%
\pgfpathlineto{\pgfqpoint{3.586891in}{3.964491in}}%
\pgfpathlineto{\pgfqpoint{3.588188in}{3.964491in}}%
\pgfpathlineto{\pgfqpoint{3.589486in}{2.930116in}}%
\pgfpathlineto{\pgfqpoint{3.593378in}{2.930116in}}%
\pgfpathlineto{\pgfqpoint{3.594676in}{4.481678in}}%
\pgfpathlineto{\pgfqpoint{3.595973in}{4.481678in}}%
\pgfpathlineto{\pgfqpoint{3.597270in}{3.447303in}}%
\pgfpathlineto{\pgfqpoint{3.598568in}{3.447303in}}%
\pgfpathlineto{\pgfqpoint{3.599865in}{3.964491in}}%
\pgfpathlineto{\pgfqpoint{3.601163in}{3.964491in}}%
\pgfpathlineto{\pgfqpoint{3.602460in}{4.481678in}}%
\pgfpathlineto{\pgfqpoint{3.603758in}{4.481678in}}%
\pgfpathlineto{\pgfqpoint{3.605055in}{3.447303in}}%
\pgfpathlineto{\pgfqpoint{3.606353in}{3.447303in}}%
\pgfpathlineto{\pgfqpoint{3.607650in}{4.481678in}}%
\pgfpathlineto{\pgfqpoint{3.608948in}{4.481678in}}%
\pgfpathlineto{\pgfqpoint{3.610245in}{2.930116in}}%
\pgfpathlineto{\pgfqpoint{3.611543in}{2.930116in}}%
\pgfpathlineto{\pgfqpoint{3.612840in}{3.964491in}}%
\pgfpathlineto{\pgfqpoint{3.614138in}{3.964491in}}%
\pgfpathlineto{\pgfqpoint{3.615435in}{2.930116in}}%
\pgfpathlineto{\pgfqpoint{3.616733in}{2.930116in}}%
\pgfpathlineto{\pgfqpoint{3.618030in}{4.481678in}}%
\pgfpathlineto{\pgfqpoint{3.619328in}{4.481678in}}%
\pgfpathlineto{\pgfqpoint{3.620625in}{3.964491in}}%
\pgfpathlineto{\pgfqpoint{3.621923in}{3.964491in}}%
\pgfpathlineto{\pgfqpoint{3.623220in}{4.481678in}}%
\pgfpathlineto{\pgfqpoint{3.627113in}{4.481678in}}%
\pgfpathlineto{\pgfqpoint{3.628410in}{3.964491in}}%
\pgfpathlineto{\pgfqpoint{3.629708in}{3.964491in}}%
\pgfpathlineto{\pgfqpoint{3.631005in}{2.930116in}}%
\pgfpathlineto{\pgfqpoint{3.632303in}{2.930116in}}%
\pgfpathlineto{\pgfqpoint{3.633600in}{3.447303in}}%
\pgfpathlineto{\pgfqpoint{3.634898in}{3.447303in}}%
\pgfpathlineto{\pgfqpoint{3.636196in}{3.964491in}}%
\pgfpathlineto{\pgfqpoint{3.637493in}{3.964491in}}%
\pgfpathlineto{\pgfqpoint{3.638790in}{3.447303in}}%
\pgfpathlineto{\pgfqpoint{3.640088in}{3.447303in}}%
\pgfpathlineto{\pgfqpoint{3.641385in}{3.964491in}}%
\pgfpathlineto{\pgfqpoint{3.645278in}{3.964491in}}%
\pgfpathlineto{\pgfqpoint{3.646575in}{2.930116in}}%
\pgfpathlineto{\pgfqpoint{3.650468in}{2.930116in}}%
\pgfpathlineto{\pgfqpoint{3.651765in}{3.964491in}}%
\pgfpathlineto{\pgfqpoint{3.653063in}{3.964491in}}%
\pgfpathlineto{\pgfqpoint{3.654360in}{4.481678in}}%
\pgfpathlineto{\pgfqpoint{3.658253in}{4.481678in}}%
\pgfpathlineto{\pgfqpoint{3.659550in}{3.447303in}}%
\pgfpathlineto{\pgfqpoint{3.660848in}{3.447303in}}%
\pgfpathlineto{\pgfqpoint{3.662145in}{2.930116in}}%
\pgfpathlineto{\pgfqpoint{3.663443in}{2.930116in}}%
\pgfpathlineto{\pgfqpoint{3.664740in}{4.481678in}}%
\pgfpathlineto{\pgfqpoint{3.666038in}{4.481678in}}%
\pgfpathlineto{\pgfqpoint{3.667335in}{2.930116in}}%
\pgfpathlineto{\pgfqpoint{3.668633in}{2.930116in}}%
\pgfpathlineto{\pgfqpoint{3.669930in}{4.481678in}}%
\pgfpathlineto{\pgfqpoint{3.673823in}{4.481678in}}%
\pgfpathlineto{\pgfqpoint{3.675120in}{2.930116in}}%
\pgfpathlineto{\pgfqpoint{3.679013in}{2.930116in}}%
\pgfpathlineto{\pgfqpoint{3.680310in}{3.964491in}}%
\pgfpathlineto{\pgfqpoint{3.681608in}{3.964491in}}%
\pgfpathlineto{\pgfqpoint{3.682905in}{4.481678in}}%
\pgfpathlineto{\pgfqpoint{3.684203in}{4.481678in}}%
\pgfpathlineto{\pgfqpoint{3.685500in}{3.964491in}}%
\pgfpathlineto{\pgfqpoint{3.686798in}{3.964491in}}%
\pgfpathlineto{\pgfqpoint{3.688095in}{4.481678in}}%
\pgfpathlineto{\pgfqpoint{3.689393in}{4.481678in}}%
\pgfpathlineto{\pgfqpoint{3.690690in}{2.930116in}}%
\pgfpathlineto{\pgfqpoint{3.691988in}{2.930116in}}%
\pgfpathlineto{\pgfqpoint{3.693285in}{3.447303in}}%
\pgfpathlineto{\pgfqpoint{3.694583in}{3.447303in}}%
\pgfpathlineto{\pgfqpoint{3.695880in}{2.930116in}}%
\pgfpathlineto{\pgfqpoint{3.697178in}{2.930116in}}%
\pgfpathlineto{\pgfqpoint{3.698475in}{4.481678in}}%
\pgfpathlineto{\pgfqpoint{3.702368in}{4.481678in}}%
\pgfpathlineto{\pgfqpoint{3.703665in}{3.964491in}}%
\pgfpathlineto{\pgfqpoint{3.704963in}{3.964491in}}%
\pgfpathlineto{\pgfqpoint{3.706260in}{2.930116in}}%
\pgfpathlineto{\pgfqpoint{3.707558in}{2.930116in}}%
\pgfpathlineto{\pgfqpoint{3.708855in}{3.964491in}}%
\pgfpathlineto{\pgfqpoint{3.715343in}{3.964491in}}%
\pgfpathlineto{\pgfqpoint{3.716640in}{3.447303in}}%
\pgfpathlineto{\pgfqpoint{3.717938in}{3.447303in}}%
\pgfpathlineto{\pgfqpoint{3.719235in}{3.964491in}}%
\pgfpathlineto{\pgfqpoint{3.720533in}{3.964491in}}%
\pgfpathlineto{\pgfqpoint{3.721830in}{2.930116in}}%
\pgfpathlineto{\pgfqpoint{3.723128in}{2.930116in}}%
\pgfpathlineto{\pgfqpoint{3.724425in}{4.481678in}}%
\pgfpathlineto{\pgfqpoint{3.725723in}{4.481678in}}%
\pgfpathlineto{\pgfqpoint{3.727020in}{2.930116in}}%
\pgfpathlineto{\pgfqpoint{3.730913in}{2.930116in}}%
\pgfpathlineto{\pgfqpoint{3.732210in}{3.447303in}}%
\pgfpathlineto{\pgfqpoint{3.733508in}{3.447303in}}%
\pgfpathlineto{\pgfqpoint{3.734805in}{2.930116in}}%
\pgfpathlineto{\pgfqpoint{3.736103in}{2.930116in}}%
\pgfpathlineto{\pgfqpoint{3.737400in}{3.964491in}}%
\pgfpathlineto{\pgfqpoint{3.743888in}{3.964491in}}%
\pgfpathlineto{\pgfqpoint{3.745185in}{3.447303in}}%
\pgfpathlineto{\pgfqpoint{3.746483in}{3.447303in}}%
\pgfpathlineto{\pgfqpoint{3.747780in}{4.481678in}}%
\pgfpathlineto{\pgfqpoint{3.749078in}{4.481678in}}%
\pgfpathlineto{\pgfqpoint{3.750375in}{3.964491in}}%
\pgfpathlineto{\pgfqpoint{3.751673in}{3.964491in}}%
\pgfpathlineto{\pgfqpoint{3.752970in}{4.481678in}}%
\pgfpathlineto{\pgfqpoint{3.756863in}{4.481678in}}%
\pgfpathlineto{\pgfqpoint{3.758160in}{3.964491in}}%
\pgfpathlineto{\pgfqpoint{3.762053in}{3.964491in}}%
\pgfpathlineto{\pgfqpoint{3.763350in}{4.481678in}}%
\pgfpathlineto{\pgfqpoint{3.767243in}{4.481678in}}%
\pgfpathlineto{\pgfqpoint{3.768540in}{3.447303in}}%
\pgfpathlineto{\pgfqpoint{3.769838in}{3.447303in}}%
\pgfpathlineto{\pgfqpoint{3.771135in}{3.964491in}}%
\pgfpathlineto{\pgfqpoint{3.775028in}{3.964491in}}%
\pgfpathlineto{\pgfqpoint{3.776325in}{4.481678in}}%
\pgfpathlineto{\pgfqpoint{3.777623in}{4.481678in}}%
\pgfpathlineto{\pgfqpoint{3.778920in}{3.964491in}}%
\pgfpathlineto{\pgfqpoint{3.780218in}{3.964491in}}%
\pgfpathlineto{\pgfqpoint{3.781515in}{4.481678in}}%
\pgfpathlineto{\pgfqpoint{3.782813in}{4.481678in}}%
\pgfpathlineto{\pgfqpoint{3.784110in}{3.964491in}}%
\pgfpathlineto{\pgfqpoint{3.785408in}{3.964491in}}%
\pgfpathlineto{\pgfqpoint{3.786705in}{2.930116in}}%
\pgfpathlineto{\pgfqpoint{3.788003in}{2.930116in}}%
\pgfpathlineto{\pgfqpoint{3.789300in}{3.447303in}}%
\pgfpathlineto{\pgfqpoint{3.790598in}{3.447303in}}%
\pgfpathlineto{\pgfqpoint{3.791895in}{3.964491in}}%
\pgfpathlineto{\pgfqpoint{3.793193in}{3.964491in}}%
\pgfpathlineto{\pgfqpoint{3.794490in}{2.930116in}}%
\pgfpathlineto{\pgfqpoint{3.795787in}{2.930116in}}%
\pgfpathlineto{\pgfqpoint{3.797085in}{3.964491in}}%
\pgfpathlineto{\pgfqpoint{3.798382in}{3.964491in}}%
\pgfpathlineto{\pgfqpoint{3.799680in}{3.447303in}}%
\pgfpathlineto{\pgfqpoint{3.800977in}{3.447303in}}%
\pgfpathlineto{\pgfqpoint{3.802275in}{3.964491in}}%
\pgfpathlineto{\pgfqpoint{3.803573in}{3.964491in}}%
\pgfpathlineto{\pgfqpoint{3.804870in}{4.481678in}}%
\pgfpathlineto{\pgfqpoint{3.808763in}{4.481678in}}%
\pgfpathlineto{\pgfqpoint{3.810060in}{3.447303in}}%
\pgfpathlineto{\pgfqpoint{3.811358in}{3.447303in}}%
\pgfpathlineto{\pgfqpoint{3.812655in}{4.481678in}}%
\pgfpathlineto{\pgfqpoint{3.813953in}{4.481678in}}%
\pgfpathlineto{\pgfqpoint{3.815250in}{3.964491in}}%
\pgfpathlineto{\pgfqpoint{3.816548in}{3.964491in}}%
\pgfpathlineto{\pgfqpoint{3.817845in}{3.447303in}}%
\pgfpathlineto{\pgfqpoint{3.819143in}{3.447303in}}%
\pgfpathlineto{\pgfqpoint{3.820440in}{2.930116in}}%
\pgfpathlineto{\pgfqpoint{3.821738in}{2.930116in}}%
\pgfpathlineto{\pgfqpoint{3.823035in}{3.964491in}}%
\pgfpathlineto{\pgfqpoint{3.824333in}{3.964491in}}%
\pgfpathlineto{\pgfqpoint{3.825630in}{3.447303in}}%
\pgfpathlineto{\pgfqpoint{3.826928in}{3.447303in}}%
\pgfpathlineto{\pgfqpoint{3.828225in}{3.964491in}}%
\pgfpathlineto{\pgfqpoint{3.829523in}{3.964491in}}%
\pgfpathlineto{\pgfqpoint{3.830820in}{2.930116in}}%
\pgfpathlineto{\pgfqpoint{3.839902in}{2.930116in}}%
\pgfpathlineto{\pgfqpoint{3.841200in}{3.964491in}}%
\pgfpathlineto{\pgfqpoint{3.845092in}{3.964491in}}%
\pgfpathlineto{\pgfqpoint{3.846390in}{2.930116in}}%
\pgfpathlineto{\pgfqpoint{3.847687in}{2.930116in}}%
\pgfpathlineto{\pgfqpoint{3.848985in}{4.481678in}}%
\pgfpathlineto{\pgfqpoint{3.850282in}{4.481678in}}%
\pgfpathlineto{\pgfqpoint{3.851580in}{2.930116in}}%
\pgfpathlineto{\pgfqpoint{3.852878in}{2.930116in}}%
\pgfpathlineto{\pgfqpoint{3.854175in}{3.964491in}}%
\pgfpathlineto{\pgfqpoint{3.855473in}{3.964491in}}%
\pgfpathlineto{\pgfqpoint{3.856770in}{2.930116in}}%
\pgfpathlineto{\pgfqpoint{3.858068in}{2.930116in}}%
\pgfpathlineto{\pgfqpoint{3.859365in}{3.447303in}}%
\pgfpathlineto{\pgfqpoint{3.860663in}{3.447303in}}%
\pgfpathlineto{\pgfqpoint{3.861960in}{2.930116in}}%
\pgfpathlineto{\pgfqpoint{3.865853in}{2.930116in}}%
\pgfpathlineto{\pgfqpoint{3.867150in}{3.964491in}}%
\pgfpathlineto{\pgfqpoint{3.868448in}{3.964491in}}%
\pgfpathlineto{\pgfqpoint{3.869745in}{3.447303in}}%
\pgfpathlineto{\pgfqpoint{3.873638in}{3.447303in}}%
\pgfpathlineto{\pgfqpoint{3.874935in}{2.930116in}}%
\pgfpathlineto{\pgfqpoint{3.876232in}{2.930116in}}%
\pgfpathlineto{\pgfqpoint{3.877530in}{3.964491in}}%
\pgfpathlineto{\pgfqpoint{3.881422in}{3.964491in}}%
\pgfpathlineto{\pgfqpoint{3.882720in}{2.930116in}}%
\pgfpathlineto{\pgfqpoint{3.884017in}{2.930116in}}%
\pgfpathlineto{\pgfqpoint{3.885315in}{3.964491in}}%
\pgfpathlineto{\pgfqpoint{3.889207in}{3.964491in}}%
\pgfpathlineto{\pgfqpoint{3.890505in}{3.447303in}}%
\pgfpathlineto{\pgfqpoint{3.891802in}{3.447303in}}%
\pgfpathlineto{\pgfqpoint{3.893100in}{4.481678in}}%
\pgfpathlineto{\pgfqpoint{3.894397in}{4.481678in}}%
\pgfpathlineto{\pgfqpoint{3.895695in}{2.930116in}}%
\pgfpathlineto{\pgfqpoint{3.896992in}{2.930116in}}%
\pgfpathlineto{\pgfqpoint{3.898290in}{3.964491in}}%
\pgfpathlineto{\pgfqpoint{3.899587in}{3.964491in}}%
\pgfpathlineto{\pgfqpoint{3.900885in}{3.447303in}}%
\pgfpathlineto{\pgfqpoint{3.902183in}{3.447303in}}%
\pgfpathlineto{\pgfqpoint{3.903480in}{4.481678in}}%
\pgfpathlineto{\pgfqpoint{3.904778in}{4.481678in}}%
\pgfpathlineto{\pgfqpoint{3.906075in}{2.930116in}}%
\pgfpathlineto{\pgfqpoint{3.907373in}{2.930116in}}%
\pgfpathlineto{\pgfqpoint{3.908670in}{3.447303in}}%
\pgfpathlineto{\pgfqpoint{3.909968in}{3.447303in}}%
\pgfpathlineto{\pgfqpoint{3.911265in}{4.481678in}}%
\pgfpathlineto{\pgfqpoint{3.912563in}{4.481678in}}%
\pgfpathlineto{\pgfqpoint{3.913860in}{3.964491in}}%
\pgfpathlineto{\pgfqpoint{3.920347in}{3.964491in}}%
\pgfpathlineto{\pgfqpoint{3.921645in}{4.481678in}}%
\pgfpathlineto{\pgfqpoint{3.928132in}{4.481678in}}%
\pgfpathlineto{\pgfqpoint{3.929430in}{2.930116in}}%
\pgfpathlineto{\pgfqpoint{3.930727in}{2.930116in}}%
\pgfpathlineto{\pgfqpoint{3.932025in}{4.481678in}}%
\pgfpathlineto{\pgfqpoint{3.933322in}{4.481678in}}%
\pgfpathlineto{\pgfqpoint{3.934620in}{3.447303in}}%
\pgfpathlineto{\pgfqpoint{3.935917in}{3.447303in}}%
\pgfpathlineto{\pgfqpoint{3.937215in}{2.930116in}}%
\pgfpathlineto{\pgfqpoint{3.938512in}{2.930116in}}%
\pgfpathlineto{\pgfqpoint{3.939810in}{3.447303in}}%
\pgfpathlineto{\pgfqpoint{3.941107in}{3.447303in}}%
\pgfpathlineto{\pgfqpoint{3.942405in}{4.481678in}}%
\pgfpathlineto{\pgfqpoint{3.943702in}{4.481678in}}%
\pgfpathlineto{\pgfqpoint{3.945000in}{3.964491in}}%
\pgfpathlineto{\pgfqpoint{3.948892in}{3.964491in}}%
\pgfpathlineto{\pgfqpoint{3.950190in}{3.447303in}}%
\pgfpathlineto{\pgfqpoint{3.951487in}{3.447303in}}%
\pgfpathlineto{\pgfqpoint{3.952785in}{3.964491in}}%
\pgfpathlineto{\pgfqpoint{3.954083in}{3.964491in}}%
\pgfpathlineto{\pgfqpoint{3.955380in}{3.447303in}}%
\pgfpathlineto{\pgfqpoint{3.956677in}{3.447303in}}%
\pgfpathlineto{\pgfqpoint{3.957975in}{4.481678in}}%
\pgfpathlineto{\pgfqpoint{3.959272in}{4.481678in}}%
\pgfpathlineto{\pgfqpoint{3.960570in}{3.447303in}}%
\pgfpathlineto{\pgfqpoint{3.961867in}{3.447303in}}%
\pgfpathlineto{\pgfqpoint{3.963165in}{3.964491in}}%
\pgfpathlineto{\pgfqpoint{3.964462in}{3.964491in}}%
\pgfpathlineto{\pgfqpoint{3.965760in}{4.481678in}}%
\pgfpathlineto{\pgfqpoint{3.974842in}{4.481678in}}%
\pgfpathlineto{\pgfqpoint{3.976140in}{3.964491in}}%
\pgfpathlineto{\pgfqpoint{3.977437in}{3.964491in}}%
\pgfpathlineto{\pgfqpoint{3.978735in}{4.481678in}}%
\pgfpathlineto{\pgfqpoint{3.980032in}{4.481678in}}%
\pgfpathlineto{\pgfqpoint{3.981330in}{3.447303in}}%
\pgfpathlineto{\pgfqpoint{3.982627in}{3.447303in}}%
\pgfpathlineto{\pgfqpoint{3.983925in}{4.481678in}}%
\pgfpathlineto{\pgfqpoint{3.985222in}{4.481678in}}%
\pgfpathlineto{\pgfqpoint{3.986520in}{3.447303in}}%
\pgfpathlineto{\pgfqpoint{3.987817in}{3.447303in}}%
\pgfpathlineto{\pgfqpoint{3.989115in}{3.964491in}}%
\pgfpathlineto{\pgfqpoint{3.990412in}{3.964491in}}%
\pgfpathlineto{\pgfqpoint{3.991710in}{2.930116in}}%
\pgfpathlineto{\pgfqpoint{3.993007in}{2.930116in}}%
\pgfpathlineto{\pgfqpoint{3.994305in}{3.447303in}}%
\pgfpathlineto{\pgfqpoint{3.995602in}{3.447303in}}%
\pgfpathlineto{\pgfqpoint{3.996900in}{4.481678in}}%
\pgfpathlineto{\pgfqpoint{3.998197in}{4.481678in}}%
\pgfpathlineto{\pgfqpoint{3.999495in}{3.964491in}}%
\pgfpathlineto{\pgfqpoint{4.000792in}{3.964491in}}%
\pgfpathlineto{\pgfqpoint{4.002090in}{3.447303in}}%
\pgfpathlineto{\pgfqpoint{4.003387in}{3.447303in}}%
\pgfpathlineto{\pgfqpoint{4.004685in}{4.481678in}}%
\pgfpathlineto{\pgfqpoint{4.005982in}{4.481678in}}%
\pgfpathlineto{\pgfqpoint{4.007280in}{3.964491in}}%
\pgfpathlineto{\pgfqpoint{4.011172in}{3.964491in}}%
\pgfpathlineto{\pgfqpoint{4.012470in}{2.930116in}}%
\pgfpathlineto{\pgfqpoint{4.013767in}{2.930116in}}%
\pgfpathlineto{\pgfqpoint{4.015065in}{3.447303in}}%
\pgfpathlineto{\pgfqpoint{4.016362in}{3.447303in}}%
\pgfpathlineto{\pgfqpoint{4.017660in}{3.964491in}}%
\pgfpathlineto{\pgfqpoint{4.031932in}{3.964491in}}%
\pgfpathlineto{\pgfqpoint{4.033230in}{2.930116in}}%
\pgfpathlineto{\pgfqpoint{4.037122in}{2.930116in}}%
\pgfpathlineto{\pgfqpoint{4.038420in}{4.481678in}}%
\pgfpathlineto{\pgfqpoint{4.039717in}{4.481678in}}%
\pgfpathlineto{\pgfqpoint{4.041015in}{3.964491in}}%
\pgfpathlineto{\pgfqpoint{4.042312in}{3.964491in}}%
\pgfpathlineto{\pgfqpoint{4.043610in}{4.481678in}}%
\pgfpathlineto{\pgfqpoint{4.044907in}{4.481678in}}%
\pgfpathlineto{\pgfqpoint{4.046205in}{3.447303in}}%
\pgfpathlineto{\pgfqpoint{4.052692in}{3.447303in}}%
\pgfpathlineto{\pgfqpoint{4.053990in}{2.930116in}}%
\pgfpathlineto{\pgfqpoint{4.055287in}{2.930116in}}%
\pgfpathlineto{\pgfqpoint{4.056585in}{3.964491in}}%
\pgfpathlineto{\pgfqpoint{4.057882in}{3.964491in}}%
\pgfpathlineto{\pgfqpoint{4.059180in}{3.447303in}}%
\pgfpathlineto{\pgfqpoint{4.060477in}{3.447303in}}%
\pgfpathlineto{\pgfqpoint{4.061775in}{3.964491in}}%
\pgfpathlineto{\pgfqpoint{4.063072in}{3.964491in}}%
\pgfpathlineto{\pgfqpoint{4.064370in}{3.447303in}}%
\pgfpathlineto{\pgfqpoint{4.065667in}{3.447303in}}%
\pgfpathlineto{\pgfqpoint{4.066965in}{2.930116in}}%
\pgfpathlineto{\pgfqpoint{4.070857in}{2.930116in}}%
\pgfpathlineto{\pgfqpoint{4.072155in}{3.964491in}}%
\pgfpathlineto{\pgfqpoint{4.073452in}{3.964491in}}%
\pgfpathlineto{\pgfqpoint{4.074750in}{4.481678in}}%
\pgfpathlineto{\pgfqpoint{4.076047in}{4.481678in}}%
\pgfpathlineto{\pgfqpoint{4.077345in}{3.964491in}}%
\pgfpathlineto{\pgfqpoint{4.081237in}{3.964491in}}%
\pgfpathlineto{\pgfqpoint{4.082535in}{4.481678in}}%
\pgfpathlineto{\pgfqpoint{4.083832in}{4.481678in}}%
\pgfpathlineto{\pgfqpoint{4.085130in}{3.964491in}}%
\pgfpathlineto{\pgfqpoint{4.086427in}{3.964491in}}%
\pgfpathlineto{\pgfqpoint{4.087725in}{4.481678in}}%
\pgfpathlineto{\pgfqpoint{4.089022in}{4.481678in}}%
\pgfpathlineto{\pgfqpoint{4.090320in}{2.930116in}}%
\pgfpathlineto{\pgfqpoint{4.094212in}{2.930116in}}%
\pgfpathlineto{\pgfqpoint{4.095510in}{3.964491in}}%
\pgfpathlineto{\pgfqpoint{4.096807in}{3.964491in}}%
\pgfpathlineto{\pgfqpoint{4.098105in}{3.447303in}}%
\pgfpathlineto{\pgfqpoint{4.099402in}{3.447303in}}%
\pgfpathlineto{\pgfqpoint{4.100700in}{2.930116in}}%
\pgfpathlineto{\pgfqpoint{4.101997in}{2.930116in}}%
\pgfpathlineto{\pgfqpoint{4.103295in}{3.964491in}}%
\pgfpathlineto{\pgfqpoint{4.104592in}{3.964491in}}%
\pgfpathlineto{\pgfqpoint{4.105890in}{4.481678in}}%
\pgfpathlineto{\pgfqpoint{4.109782in}{4.481678in}}%
\pgfpathlineto{\pgfqpoint{4.111080in}{3.447303in}}%
\pgfpathlineto{\pgfqpoint{4.112377in}{3.447303in}}%
\pgfpathlineto{\pgfqpoint{4.113674in}{2.930116in}}%
\pgfpathlineto{\pgfqpoint{4.114972in}{2.930116in}}%
\pgfpathlineto{\pgfqpoint{4.116269in}{3.964491in}}%
\pgfpathlineto{\pgfqpoint{4.120162in}{3.964491in}}%
\pgfpathlineto{\pgfqpoint{4.121460in}{4.481678in}}%
\pgfpathlineto{\pgfqpoint{4.125352in}{4.481678in}}%
\pgfpathlineto{\pgfqpoint{4.126650in}{3.964491in}}%
\pgfpathlineto{\pgfqpoint{4.127947in}{3.964491in}}%
\pgfpathlineto{\pgfqpoint{4.129245in}{2.930116in}}%
\pgfpathlineto{\pgfqpoint{4.130542in}{2.930116in}}%
\pgfpathlineto{\pgfqpoint{4.131840in}{3.447303in}}%
\pgfpathlineto{\pgfqpoint{4.133137in}{3.447303in}}%
\pgfpathlineto{\pgfqpoint{4.134435in}{3.964491in}}%
\pgfpathlineto{\pgfqpoint{4.135732in}{3.964491in}}%
\pgfpathlineto{\pgfqpoint{4.137030in}{3.447303in}}%
\pgfpathlineto{\pgfqpoint{4.138327in}{3.447303in}}%
\pgfpathlineto{\pgfqpoint{4.139625in}{2.930116in}}%
\pgfpathlineto{\pgfqpoint{4.140922in}{2.930116in}}%
\pgfpathlineto{\pgfqpoint{4.142220in}{3.447303in}}%
\pgfpathlineto{\pgfqpoint{4.143517in}{3.447303in}}%
\pgfpathlineto{\pgfqpoint{4.144815in}{3.964491in}}%
\pgfpathlineto{\pgfqpoint{4.146112in}{3.964491in}}%
\pgfpathlineto{\pgfqpoint{4.147410in}{2.930116in}}%
\pgfpathlineto{\pgfqpoint{4.151302in}{2.930116in}}%
\pgfpathlineto{\pgfqpoint{4.152599in}{3.964491in}}%
\pgfpathlineto{\pgfqpoint{4.153897in}{3.964491in}}%
\pgfpathlineto{\pgfqpoint{4.155194in}{2.930116in}}%
\pgfpathlineto{\pgfqpoint{4.159087in}{2.930116in}}%
\pgfpathlineto{\pgfqpoint{4.160384in}{3.964491in}}%
\pgfpathlineto{\pgfqpoint{4.166872in}{3.964491in}}%
\pgfpathlineto{\pgfqpoint{4.168169in}{3.447303in}}%
\pgfpathlineto{\pgfqpoint{4.169467in}{3.447303in}}%
\pgfpathlineto{\pgfqpoint{4.170765in}{2.930116in}}%
\pgfpathlineto{\pgfqpoint{4.172062in}{2.930116in}}%
\pgfpathlineto{\pgfqpoint{4.173360in}{3.964491in}}%
\pgfpathlineto{\pgfqpoint{4.174657in}{3.964491in}}%
\pgfpathlineto{\pgfqpoint{4.175955in}{4.481678in}}%
\pgfpathlineto{\pgfqpoint{4.177252in}{4.481678in}}%
\pgfpathlineto{\pgfqpoint{4.178550in}{2.930116in}}%
\pgfpathlineto{\pgfqpoint{4.179847in}{2.930116in}}%
\pgfpathlineto{\pgfqpoint{4.181145in}{3.447303in}}%
\pgfpathlineto{\pgfqpoint{4.185037in}{3.447303in}}%
\pgfpathlineto{\pgfqpoint{4.186335in}{3.964491in}}%
\pgfpathlineto{\pgfqpoint{4.192822in}{3.964491in}}%
\pgfpathlineto{\pgfqpoint{4.194119in}{2.930116in}}%
\pgfpathlineto{\pgfqpoint{4.195417in}{2.930116in}}%
\pgfpathlineto{\pgfqpoint{4.196714in}{4.481678in}}%
\pgfpathlineto{\pgfqpoint{4.198012in}{4.481678in}}%
\pgfpathlineto{\pgfqpoint{4.199309in}{3.964491in}}%
\pgfpathlineto{\pgfqpoint{4.200607in}{3.964491in}}%
\pgfpathlineto{\pgfqpoint{4.201904in}{4.481678in}}%
\pgfpathlineto{\pgfqpoint{4.203202in}{4.481678in}}%
\pgfpathlineto{\pgfqpoint{4.204499in}{3.964491in}}%
\pgfpathlineto{\pgfqpoint{4.205797in}{3.964491in}}%
\pgfpathlineto{\pgfqpoint{4.207094in}{2.930116in}}%
\pgfpathlineto{\pgfqpoint{4.210987in}{2.930116in}}%
\pgfpathlineto{\pgfqpoint{4.212284in}{3.447303in}}%
\pgfpathlineto{\pgfqpoint{4.213582in}{3.447303in}}%
\pgfpathlineto{\pgfqpoint{4.214879in}{3.964491in}}%
\pgfpathlineto{\pgfqpoint{4.216177in}{3.964491in}}%
\pgfpathlineto{\pgfqpoint{4.217474in}{4.481678in}}%
\pgfpathlineto{\pgfqpoint{4.221367in}{4.481678in}}%
\pgfpathlineto{\pgfqpoint{4.222665in}{2.930116in}}%
\pgfpathlineto{\pgfqpoint{4.226557in}{2.930116in}}%
\pgfpathlineto{\pgfqpoint{4.227855in}{3.447303in}}%
\pgfpathlineto{\pgfqpoint{4.229152in}{3.447303in}}%
\pgfpathlineto{\pgfqpoint{4.230450in}{3.964491in}}%
\pgfpathlineto{\pgfqpoint{4.231747in}{3.964491in}}%
\pgfpathlineto{\pgfqpoint{4.233044in}{3.447303in}}%
\pgfpathlineto{\pgfqpoint{4.234342in}{3.447303in}}%
\pgfpathlineto{\pgfqpoint{4.235639in}{2.930116in}}%
\pgfpathlineto{\pgfqpoint{4.236937in}{2.930116in}}%
\pgfpathlineto{\pgfqpoint{4.238234in}{3.964491in}}%
\pgfpathlineto{\pgfqpoint{4.239532in}{3.964491in}}%
\pgfpathlineto{\pgfqpoint{4.240829in}{3.447303in}}%
\pgfpathlineto{\pgfqpoint{4.244722in}{3.447303in}}%
\pgfpathlineto{\pgfqpoint{4.246019in}{3.964491in}}%
\pgfpathlineto{\pgfqpoint{4.247317in}{3.964491in}}%
\pgfpathlineto{\pgfqpoint{4.248614in}{3.447303in}}%
\pgfpathlineto{\pgfqpoint{4.252507in}{3.447303in}}%
\pgfpathlineto{\pgfqpoint{4.253804in}{2.930116in}}%
\pgfpathlineto{\pgfqpoint{4.257697in}{2.930116in}}%
\pgfpathlineto{\pgfqpoint{4.258994in}{3.964491in}}%
\pgfpathlineto{\pgfqpoint{4.260292in}{3.964491in}}%
\pgfpathlineto{\pgfqpoint{4.261589in}{2.930116in}}%
\pgfpathlineto{\pgfqpoint{4.262887in}{2.930116in}}%
\pgfpathlineto{\pgfqpoint{4.264184in}{3.964491in}}%
\pgfpathlineto{\pgfqpoint{4.268077in}{3.964491in}}%
\pgfpathlineto{\pgfqpoint{4.269374in}{2.930116in}}%
\pgfpathlineto{\pgfqpoint{4.270672in}{2.930116in}}%
\pgfpathlineto{\pgfqpoint{4.271969in}{4.481678in}}%
\pgfpathlineto{\pgfqpoint{4.273267in}{4.481678in}}%
\pgfpathlineto{\pgfqpoint{4.274564in}{2.930116in}}%
\pgfpathlineto{\pgfqpoint{4.275862in}{2.930116in}}%
\pgfpathlineto{\pgfqpoint{4.277159in}{4.481678in}}%
\pgfpathlineto{\pgfqpoint{4.278457in}{4.481678in}}%
\pgfpathlineto{\pgfqpoint{4.279754in}{3.447303in}}%
\pgfpathlineto{\pgfqpoint{4.283647in}{3.447303in}}%
\pgfpathlineto{\pgfqpoint{4.284944in}{2.930116in}}%
\pgfpathlineto{\pgfqpoint{4.286242in}{2.930116in}}%
\pgfpathlineto{\pgfqpoint{4.287539in}{4.481678in}}%
\pgfpathlineto{\pgfqpoint{4.288837in}{4.481678in}}%
\pgfpathlineto{\pgfqpoint{4.290134in}{2.930116in}}%
\pgfpathlineto{\pgfqpoint{4.294027in}{2.930116in}}%
\pgfpathlineto{\pgfqpoint{4.295324in}{3.447303in}}%
\pgfpathlineto{\pgfqpoint{4.296622in}{3.447303in}}%
\pgfpathlineto{\pgfqpoint{4.297919in}{4.481678in}}%
\pgfpathlineto{\pgfqpoint{4.299217in}{4.481678in}}%
\pgfpathlineto{\pgfqpoint{4.300514in}{2.930116in}}%
\pgfpathlineto{\pgfqpoint{4.312192in}{2.930116in}}%
\pgfpathlineto{\pgfqpoint{4.313489in}{4.481678in}}%
\pgfpathlineto{\pgfqpoint{4.314787in}{4.481678in}}%
\pgfpathlineto{\pgfqpoint{4.316084in}{3.964491in}}%
\pgfpathlineto{\pgfqpoint{4.317382in}{3.964491in}}%
\pgfpathlineto{\pgfqpoint{4.318679in}{4.481678in}}%
\pgfpathlineto{\pgfqpoint{4.322572in}{4.481678in}}%
\pgfpathlineto{\pgfqpoint{4.323869in}{2.930116in}}%
\pgfpathlineto{\pgfqpoint{4.327762in}{2.930116in}}%
\pgfpathlineto{\pgfqpoint{4.329059in}{4.481678in}}%
\pgfpathlineto{\pgfqpoint{4.332952in}{4.481678in}}%
\pgfpathlineto{\pgfqpoint{4.334249in}{3.447303in}}%
\pgfpathlineto{\pgfqpoint{4.335547in}{3.447303in}}%
\pgfpathlineto{\pgfqpoint{4.336844in}{4.481678in}}%
\pgfpathlineto{\pgfqpoint{4.343332in}{4.481678in}}%
\pgfpathlineto{\pgfqpoint{4.344629in}{3.447303in}}%
\pgfpathlineto{\pgfqpoint{4.345927in}{3.447303in}}%
\pgfpathlineto{\pgfqpoint{4.347224in}{4.481678in}}%
\pgfpathlineto{\pgfqpoint{4.348522in}{4.481678in}}%
\pgfpathlineto{\pgfqpoint{4.349819in}{3.964491in}}%
\pgfpathlineto{\pgfqpoint{4.356307in}{3.964491in}}%
\pgfpathlineto{\pgfqpoint{4.357604in}{2.930116in}}%
\pgfpathlineto{\pgfqpoint{4.358902in}{2.930116in}}%
\pgfpathlineto{\pgfqpoint{4.360199in}{3.964491in}}%
\pgfpathlineto{\pgfqpoint{4.361497in}{3.964491in}}%
\pgfpathlineto{\pgfqpoint{4.362794in}{3.447303in}}%
\pgfpathlineto{\pgfqpoint{4.364092in}{3.447303in}}%
\pgfpathlineto{\pgfqpoint{4.365389in}{3.964491in}}%
\pgfpathlineto{\pgfqpoint{4.366687in}{3.964491in}}%
\pgfpathlineto{\pgfqpoint{4.367984in}{2.930116in}}%
\pgfpathlineto{\pgfqpoint{4.369282in}{2.930116in}}%
\pgfpathlineto{\pgfqpoint{4.370579in}{3.964491in}}%
\pgfpathlineto{\pgfqpoint{4.371877in}{3.964491in}}%
\pgfpathlineto{\pgfqpoint{4.373174in}{4.481678in}}%
\pgfpathlineto{\pgfqpoint{4.374472in}{4.481678in}}%
\pgfpathlineto{\pgfqpoint{4.375769in}{3.964491in}}%
\pgfpathlineto{\pgfqpoint{4.379662in}{3.964491in}}%
\pgfpathlineto{\pgfqpoint{4.380959in}{4.481678in}}%
\pgfpathlineto{\pgfqpoint{4.384852in}{4.481678in}}%
\pgfpathlineto{\pgfqpoint{4.386149in}{3.447303in}}%
\pgfpathlineto{\pgfqpoint{4.390042in}{3.447303in}}%
\pgfpathlineto{\pgfqpoint{4.391339in}{2.930116in}}%
\pgfpathlineto{\pgfqpoint{4.395232in}{2.930116in}}%
\pgfpathlineto{\pgfqpoint{4.396529in}{3.964491in}}%
\pgfpathlineto{\pgfqpoint{4.397827in}{3.964491in}}%
\pgfpathlineto{\pgfqpoint{4.399124in}{4.481678in}}%
\pgfpathlineto{\pgfqpoint{4.400422in}{4.481678in}}%
\pgfpathlineto{\pgfqpoint{4.401719in}{3.964491in}}%
\pgfpathlineto{\pgfqpoint{4.403017in}{3.964491in}}%
\pgfpathlineto{\pgfqpoint{4.404314in}{3.447303in}}%
\pgfpathlineto{\pgfqpoint{4.405612in}{3.447303in}}%
\pgfpathlineto{\pgfqpoint{4.406909in}{2.930116in}}%
\pgfpathlineto{\pgfqpoint{4.408207in}{2.930116in}}%
\pgfpathlineto{\pgfqpoint{4.409504in}{3.447303in}}%
\pgfpathlineto{\pgfqpoint{4.413397in}{3.447303in}}%
\pgfpathlineto{\pgfqpoint{4.414694in}{4.481678in}}%
\pgfpathlineto{\pgfqpoint{4.415992in}{4.481678in}}%
\pgfpathlineto{\pgfqpoint{4.417289in}{3.964491in}}%
\pgfpathlineto{\pgfqpoint{4.418587in}{3.964491in}}%
\pgfpathlineto{\pgfqpoint{4.419884in}{2.930116in}}%
\pgfpathlineto{\pgfqpoint{4.421182in}{2.930116in}}%
\pgfpathlineto{\pgfqpoint{4.422479in}{4.481678in}}%
\pgfpathlineto{\pgfqpoint{4.423777in}{4.481678in}}%
\pgfpathlineto{\pgfqpoint{4.425074in}{2.930116in}}%
\pgfpathlineto{\pgfqpoint{4.426372in}{2.930116in}}%
\pgfpathlineto{\pgfqpoint{4.427669in}{4.481678in}}%
\pgfpathlineto{\pgfqpoint{4.428967in}{4.481678in}}%
\pgfpathlineto{\pgfqpoint{4.430264in}{2.930116in}}%
\pgfpathlineto{\pgfqpoint{4.431561in}{2.930116in}}%
\pgfpathlineto{\pgfqpoint{4.432859in}{3.447303in}}%
\pgfpathlineto{\pgfqpoint{4.434156in}{3.447303in}}%
\pgfpathlineto{\pgfqpoint{4.435454in}{3.964491in}}%
\pgfpathlineto{\pgfqpoint{4.436751in}{3.964491in}}%
\pgfpathlineto{\pgfqpoint{4.438049in}{2.930116in}}%
\pgfpathlineto{\pgfqpoint{4.439347in}{2.930116in}}%
\pgfpathlineto{\pgfqpoint{4.440644in}{3.964491in}}%
\pgfpathlineto{\pgfqpoint{4.441942in}{3.964491in}}%
\pgfpathlineto{\pgfqpoint{4.443239in}{2.930116in}}%
\pgfpathlineto{\pgfqpoint{4.444537in}{2.930116in}}%
\pgfpathlineto{\pgfqpoint{4.445834in}{3.964491in}}%
\pgfpathlineto{\pgfqpoint{4.447132in}{3.964491in}}%
\pgfpathlineto{\pgfqpoint{4.448429in}{2.930116in}}%
\pgfpathlineto{\pgfqpoint{4.449727in}{2.930116in}}%
\pgfpathlineto{\pgfqpoint{4.451024in}{3.964491in}}%
\pgfpathlineto{\pgfqpoint{4.452322in}{3.964491in}}%
\pgfpathlineto{\pgfqpoint{4.453619in}{4.481678in}}%
\pgfpathlineto{\pgfqpoint{4.454917in}{4.481678in}}%
\pgfpathlineto{\pgfqpoint{4.456214in}{2.930116in}}%
\pgfpathlineto{\pgfqpoint{4.457512in}{2.930116in}}%
\pgfpathlineto{\pgfqpoint{4.458809in}{4.481678in}}%
\pgfpathlineto{\pgfqpoint{4.460107in}{4.481678in}}%
\pgfpathlineto{\pgfqpoint{4.461404in}{3.964491in}}%
\pgfpathlineto{\pgfqpoint{4.462702in}{3.964491in}}%
\pgfpathlineto{\pgfqpoint{4.463999in}{2.930116in}}%
\pgfpathlineto{\pgfqpoint{4.465297in}{2.930116in}}%
\pgfpathlineto{\pgfqpoint{4.466594in}{3.447303in}}%
\pgfpathlineto{\pgfqpoint{4.467892in}{3.447303in}}%
\pgfpathlineto{\pgfqpoint{4.469189in}{3.964491in}}%
\pgfpathlineto{\pgfqpoint{4.470486in}{3.964491in}}%
\pgfpathlineto{\pgfqpoint{4.471784in}{3.447303in}}%
\pgfpathlineto{\pgfqpoint{4.473081in}{3.447303in}}%
\pgfpathlineto{\pgfqpoint{4.474379in}{3.964491in}}%
\pgfpathlineto{\pgfqpoint{4.478271in}{3.964491in}}%
\pgfpathlineto{\pgfqpoint{4.479569in}{2.930116in}}%
\pgfpathlineto{\pgfqpoint{4.480866in}{2.930116in}}%
\pgfpathlineto{\pgfqpoint{4.482164in}{3.447303in}}%
\pgfpathlineto{\pgfqpoint{4.483461in}{3.447303in}}%
\pgfpathlineto{\pgfqpoint{4.484759in}{2.930116in}}%
\pgfpathlineto{\pgfqpoint{4.486056in}{2.930116in}}%
\pgfpathlineto{\pgfqpoint{4.487354in}{3.964491in}}%
\pgfpathlineto{\pgfqpoint{4.488652in}{3.964491in}}%
\pgfpathlineto{\pgfqpoint{4.489949in}{3.447303in}}%
\pgfpathlineto{\pgfqpoint{4.491247in}{3.447303in}}%
\pgfpathlineto{\pgfqpoint{4.492544in}{4.481678in}}%
\pgfpathlineto{\pgfqpoint{4.501627in}{4.481678in}}%
\pgfpathlineto{\pgfqpoint{4.502924in}{3.447303in}}%
\pgfpathlineto{\pgfqpoint{4.504222in}{3.447303in}}%
\pgfpathlineto{\pgfqpoint{4.505519in}{2.930116in}}%
\pgfpathlineto{\pgfqpoint{4.506817in}{2.930116in}}%
\pgfpathlineto{\pgfqpoint{4.508114in}{3.447303in}}%
\pgfpathlineto{\pgfqpoint{4.509412in}{3.447303in}}%
\pgfpathlineto{\pgfqpoint{4.510709in}{3.964491in}}%
\pgfpathlineto{\pgfqpoint{4.512006in}{3.964491in}}%
\pgfpathlineto{\pgfqpoint{4.513304in}{3.447303in}}%
\pgfpathlineto{\pgfqpoint{4.514601in}{3.447303in}}%
\pgfpathlineto{\pgfqpoint{4.515899in}{4.481678in}}%
\pgfpathlineto{\pgfqpoint{4.519791in}{4.481678in}}%
\pgfpathlineto{\pgfqpoint{4.521089in}{3.964491in}}%
\pgfpathlineto{\pgfqpoint{4.522386in}{3.964491in}}%
\pgfpathlineto{\pgfqpoint{4.523684in}{3.447303in}}%
\pgfpathlineto{\pgfqpoint{4.524981in}{3.447303in}}%
\pgfpathlineto{\pgfqpoint{4.526279in}{4.481678in}}%
\pgfpathlineto{\pgfqpoint{4.527576in}{4.481678in}}%
\pgfpathlineto{\pgfqpoint{4.528874in}{3.447303in}}%
\pgfpathlineto{\pgfqpoint{4.532766in}{3.447303in}}%
\pgfpathlineto{\pgfqpoint{4.534064in}{2.930116in}}%
\pgfpathlineto{\pgfqpoint{4.535361in}{2.930116in}}%
\pgfpathlineto{\pgfqpoint{4.536659in}{3.964491in}}%
\pgfpathlineto{\pgfqpoint{4.540552in}{3.964491in}}%
\pgfpathlineto{\pgfqpoint{4.541849in}{4.481678in}}%
\pgfpathlineto{\pgfqpoint{4.545742in}{4.481678in}}%
\pgfpathlineto{\pgfqpoint{4.547039in}{3.964491in}}%
\pgfpathlineto{\pgfqpoint{4.548337in}{3.964491in}}%
\pgfpathlineto{\pgfqpoint{4.549634in}{4.481678in}}%
\pgfpathlineto{\pgfqpoint{4.550931in}{4.481678in}}%
\pgfpathlineto{\pgfqpoint{4.552229in}{2.930116in}}%
\pgfpathlineto{\pgfqpoint{4.553526in}{2.930116in}}%
\pgfpathlineto{\pgfqpoint{4.554824in}{4.481678in}}%
\pgfpathlineto{\pgfqpoint{4.556121in}{4.481678in}}%
\pgfpathlineto{\pgfqpoint{4.557419in}{3.447303in}}%
\pgfpathlineto{\pgfqpoint{4.558716in}{3.447303in}}%
\pgfpathlineto{\pgfqpoint{4.560014in}{4.481678in}}%
\pgfpathlineto{\pgfqpoint{4.566501in}{4.481678in}}%
\pgfpathlineto{\pgfqpoint{4.567799in}{3.447303in}}%
\pgfpathlineto{\pgfqpoint{4.569096in}{3.447303in}}%
\pgfpathlineto{\pgfqpoint{4.570394in}{2.930116in}}%
\pgfpathlineto{\pgfqpoint{4.571691in}{2.930116in}}%
\pgfpathlineto{\pgfqpoint{4.572989in}{3.447303in}}%
\pgfpathlineto{\pgfqpoint{4.574286in}{3.447303in}}%
\pgfpathlineto{\pgfqpoint{4.575584in}{2.930116in}}%
\pgfpathlineto{\pgfqpoint{4.576881in}{2.930116in}}%
\pgfpathlineto{\pgfqpoint{4.578179in}{4.481678in}}%
\pgfpathlineto{\pgfqpoint{4.579476in}{4.481678in}}%
\pgfpathlineto{\pgfqpoint{4.580774in}{3.964491in}}%
\pgfpathlineto{\pgfqpoint{4.582071in}{3.964491in}}%
\pgfpathlineto{\pgfqpoint{4.583369in}{4.481678in}}%
\pgfpathlineto{\pgfqpoint{4.584666in}{4.481678in}}%
\pgfpathlineto{\pgfqpoint{4.585964in}{3.964491in}}%
\pgfpathlineto{\pgfqpoint{4.587261in}{3.964491in}}%
\pgfpathlineto{\pgfqpoint{4.588559in}{4.481678in}}%
\pgfpathlineto{\pgfqpoint{4.589856in}{4.481678in}}%
\pgfpathlineto{\pgfqpoint{4.591154in}{3.447303in}}%
\pgfpathlineto{\pgfqpoint{4.597641in}{3.447303in}}%
\pgfpathlineto{\pgfqpoint{4.598939in}{4.481678in}}%
\pgfpathlineto{\pgfqpoint{4.600236in}{4.481678in}}%
\pgfpathlineto{\pgfqpoint{4.601534in}{3.964491in}}%
\pgfpathlineto{\pgfqpoint{4.602831in}{3.964491in}}%
\pgfpathlineto{\pgfqpoint{4.604129in}{4.481678in}}%
\pgfpathlineto{\pgfqpoint{4.605426in}{4.481678in}}%
\pgfpathlineto{\pgfqpoint{4.606724in}{3.964491in}}%
\pgfpathlineto{\pgfqpoint{4.610616in}{3.964491in}}%
\pgfpathlineto{\pgfqpoint{4.611914in}{3.447303in}}%
\pgfpathlineto{\pgfqpoint{4.615806in}{3.447303in}}%
\pgfpathlineto{\pgfqpoint{4.617104in}{2.930116in}}%
\pgfpathlineto{\pgfqpoint{4.620996in}{2.930116in}}%
\pgfpathlineto{\pgfqpoint{4.622294in}{3.964491in}}%
\pgfpathlineto{\pgfqpoint{4.623591in}{3.964491in}}%
\pgfpathlineto{\pgfqpoint{4.624889in}{2.930116in}}%
\pgfpathlineto{\pgfqpoint{4.626186in}{2.930116in}}%
\pgfpathlineto{\pgfqpoint{4.627484in}{3.447303in}}%
\pgfpathlineto{\pgfqpoint{4.631376in}{3.447303in}}%
\pgfpathlineto{\pgfqpoint{4.632674in}{3.964491in}}%
\pgfpathlineto{\pgfqpoint{4.633971in}{3.964491in}}%
\pgfpathlineto{\pgfqpoint{4.635269in}{2.930116in}}%
\pgfpathlineto{\pgfqpoint{4.639161in}{2.930116in}}%
\pgfpathlineto{\pgfqpoint{4.640459in}{3.964491in}}%
\pgfpathlineto{\pgfqpoint{4.644351in}{3.964491in}}%
\pgfpathlineto{\pgfqpoint{4.645649in}{4.481678in}}%
\pgfpathlineto{\pgfqpoint{4.646946in}{4.481678in}}%
\pgfpathlineto{\pgfqpoint{4.648244in}{3.964491in}}%
\pgfpathlineto{\pgfqpoint{4.654731in}{3.964491in}}%
\pgfpathlineto{\pgfqpoint{4.656029in}{4.481678in}}%
\pgfpathlineto{\pgfqpoint{4.657326in}{4.481678in}}%
\pgfpathlineto{\pgfqpoint{4.658624in}{3.964491in}}%
\pgfpathlineto{\pgfqpoint{4.659921in}{3.964491in}}%
\pgfpathlineto{\pgfqpoint{4.661219in}{2.930116in}}%
\pgfpathlineto{\pgfqpoint{4.665111in}{2.930116in}}%
\pgfpathlineto{\pgfqpoint{4.666409in}{3.964491in}}%
\pgfpathlineto{\pgfqpoint{4.670301in}{3.964491in}}%
\pgfpathlineto{\pgfqpoint{4.671599in}{2.930116in}}%
\pgfpathlineto{\pgfqpoint{4.672896in}{2.930116in}}%
\pgfpathlineto{\pgfqpoint{4.674194in}{4.481678in}}%
\pgfpathlineto{\pgfqpoint{4.675491in}{4.481678in}}%
\pgfpathlineto{\pgfqpoint{4.676789in}{3.447303in}}%
\pgfpathlineto{\pgfqpoint{4.678086in}{3.447303in}}%
\pgfpathlineto{\pgfqpoint{4.679384in}{3.964491in}}%
\pgfpathlineto{\pgfqpoint{4.680681in}{3.964491in}}%
\pgfpathlineto{\pgfqpoint{4.681979in}{2.930116in}}%
\pgfpathlineto{\pgfqpoint{4.683276in}{2.930116in}}%
\pgfpathlineto{\pgfqpoint{4.684574in}{3.447303in}}%
\pgfpathlineto{\pgfqpoint{4.688466in}{3.447303in}}%
\pgfpathlineto{\pgfqpoint{4.689764in}{4.481678in}}%
\pgfpathlineto{\pgfqpoint{4.691061in}{4.481678in}}%
\pgfpathlineto{\pgfqpoint{4.692359in}{3.964491in}}%
\pgfpathlineto{\pgfqpoint{4.698846in}{3.964491in}}%
\pgfpathlineto{\pgfqpoint{4.700144in}{3.447303in}}%
\pgfpathlineto{\pgfqpoint{4.701441in}{3.447303in}}%
\pgfpathlineto{\pgfqpoint{4.702739in}{4.481678in}}%
\pgfpathlineto{\pgfqpoint{4.704036in}{4.481678in}}%
\pgfpathlineto{\pgfqpoint{4.705334in}{3.447303in}}%
\pgfpathlineto{\pgfqpoint{4.706631in}{3.447303in}}%
\pgfpathlineto{\pgfqpoint{4.707929in}{3.964491in}}%
\pgfpathlineto{\pgfqpoint{4.709226in}{3.964491in}}%
\pgfpathlineto{\pgfqpoint{4.710524in}{3.447303in}}%
\pgfpathlineto{\pgfqpoint{4.717011in}{3.447303in}}%
\pgfpathlineto{\pgfqpoint{4.718309in}{3.964491in}}%
\pgfpathlineto{\pgfqpoint{4.719606in}{3.964491in}}%
\pgfpathlineto{\pgfqpoint{4.720904in}{4.481678in}}%
\pgfpathlineto{\pgfqpoint{4.722201in}{4.481678in}}%
\pgfpathlineto{\pgfqpoint{4.723499in}{2.930116in}}%
\pgfpathlineto{\pgfqpoint{4.724796in}{2.930116in}}%
\pgfpathlineto{\pgfqpoint{4.726094in}{3.447303in}}%
\pgfpathlineto{\pgfqpoint{4.729986in}{3.447303in}}%
\pgfpathlineto{\pgfqpoint{4.731284in}{4.481678in}}%
\pgfpathlineto{\pgfqpoint{4.732581in}{4.481678in}}%
\pgfpathlineto{\pgfqpoint{4.733879in}{2.930116in}}%
\pgfpathlineto{\pgfqpoint{4.735176in}{2.930116in}}%
\pgfpathlineto{\pgfqpoint{4.736474in}{3.964491in}}%
\pgfpathlineto{\pgfqpoint{4.737771in}{3.964491in}}%
\pgfpathlineto{\pgfqpoint{4.739069in}{2.930116in}}%
\pgfpathlineto{\pgfqpoint{4.740366in}{2.930116in}}%
\pgfpathlineto{\pgfqpoint{4.741664in}{3.447303in}}%
\pgfpathlineto{\pgfqpoint{4.745556in}{3.447303in}}%
\pgfpathlineto{\pgfqpoint{4.746854in}{2.930116in}}%
\pgfpathlineto{\pgfqpoint{4.748151in}{2.930116in}}%
\pgfpathlineto{\pgfqpoint{4.749448in}{3.447303in}}%
\pgfpathlineto{\pgfqpoint{4.750746in}{3.447303in}}%
\pgfpathlineto{\pgfqpoint{4.752043in}{3.964491in}}%
\pgfpathlineto{\pgfqpoint{4.753341in}{3.964491in}}%
\pgfpathlineto{\pgfqpoint{4.754638in}{3.447303in}}%
\pgfpathlineto{\pgfqpoint{4.755936in}{3.447303in}}%
\pgfpathlineto{\pgfqpoint{4.757234in}{4.481678in}}%
\pgfpathlineto{\pgfqpoint{4.758531in}{4.481678in}}%
\pgfpathlineto{\pgfqpoint{4.759829in}{2.930116in}}%
\pgfpathlineto{\pgfqpoint{4.766316in}{2.930116in}}%
\pgfpathlineto{\pgfqpoint{4.767614in}{4.481678in}}%
\pgfpathlineto{\pgfqpoint{4.768911in}{4.481678in}}%
\pgfpathlineto{\pgfqpoint{4.770209in}{3.447303in}}%
\pgfpathlineto{\pgfqpoint{4.776696in}{3.447303in}}%
\pgfpathlineto{\pgfqpoint{4.777994in}{4.481678in}}%
\pgfpathlineto{\pgfqpoint{4.779291in}{4.481678in}}%
\pgfpathlineto{\pgfqpoint{4.780589in}{3.964491in}}%
\pgfpathlineto{\pgfqpoint{4.781886in}{3.964491in}}%
\pgfpathlineto{\pgfqpoint{4.783184in}{2.930116in}}%
\pgfpathlineto{\pgfqpoint{4.784481in}{2.930116in}}%
\pgfpathlineto{\pgfqpoint{4.785779in}{3.447303in}}%
\pgfpathlineto{\pgfqpoint{4.789671in}{3.447303in}}%
\pgfpathlineto{\pgfqpoint{4.790968in}{3.964491in}}%
\pgfpathlineto{\pgfqpoint{4.792266in}{3.964491in}}%
\pgfpathlineto{\pgfqpoint{4.793563in}{3.447303in}}%
\pgfpathlineto{\pgfqpoint{4.794861in}{3.447303in}}%
\pgfpathlineto{\pgfqpoint{4.796158in}{2.930116in}}%
\pgfpathlineto{\pgfqpoint{4.797456in}{2.930116in}}%
\pgfpathlineto{\pgfqpoint{4.798753in}{4.481678in}}%
\pgfpathlineto{\pgfqpoint{4.800051in}{4.481678in}}%
\pgfpathlineto{\pgfqpoint{4.801348in}{3.705897in}}%
\pgfpathlineto{\pgfqpoint{6.098847in}{3.705897in}}%
\pgfpathlineto{\pgfqpoint{6.098847in}{3.705897in}}%
\pgfusepath{stroke}%
\end{pgfscope}%
\begin{pgfscope}%
\pgfsetrectcap%
\pgfsetmiterjoin%
\pgfsetlinewidth{0.803000pt}%
\definecolor{currentstroke}{rgb}{0.000000,0.000000,0.000000}%
\pgfsetstrokecolor{currentstroke}%
\pgfsetdash{}{0pt}%
\pgfpathmoveto{\pgfqpoint{0.649696in}{2.852538in}}%
\pgfpathlineto{\pgfqpoint{0.649696in}{4.559256in}}%
\pgfusepath{stroke}%
\end{pgfscope}%
\begin{pgfscope}%
\pgfsetrectcap%
\pgfsetmiterjoin%
\pgfsetlinewidth{0.803000pt}%
\definecolor{currentstroke}{rgb}{0.000000,0.000000,0.000000}%
\pgfsetstrokecolor{currentstroke}%
\pgfsetdash{}{0pt}%
\pgfpathmoveto{\pgfqpoint{6.358330in}{2.852538in}}%
\pgfpathlineto{\pgfqpoint{6.358330in}{4.559256in}}%
\pgfusepath{stroke}%
\end{pgfscope}%
\begin{pgfscope}%
\pgfsetrectcap%
\pgfsetmiterjoin%
\pgfsetlinewidth{0.803000pt}%
\definecolor{currentstroke}{rgb}{0.000000,0.000000,0.000000}%
\pgfsetstrokecolor{currentstroke}%
\pgfsetdash{}{0pt}%
\pgfpathmoveto{\pgfqpoint{0.649696in}{2.852538in}}%
\pgfpathlineto{\pgfqpoint{6.358330in}{2.852538in}}%
\pgfusepath{stroke}%
\end{pgfscope}%
\begin{pgfscope}%
\pgfsetrectcap%
\pgfsetmiterjoin%
\pgfsetlinewidth{0.803000pt}%
\definecolor{currentstroke}{rgb}{0.000000,0.000000,0.000000}%
\pgfsetstrokecolor{currentstroke}%
\pgfsetdash{}{0pt}%
\pgfpathmoveto{\pgfqpoint{0.649696in}{4.559256in}}%
\pgfpathlineto{\pgfqpoint{6.358330in}{4.559256in}}%
\pgfusepath{stroke}%
\end{pgfscope}%
\begin{pgfscope}%
\definecolor{textcolor}{rgb}{0.000000,0.000000,0.000000}%
\pgfsetstrokecolor{textcolor}%
\pgfsetfillcolor{textcolor}%
\pgftext[x=3.504013in,y=4.642590in,,base]{\color{textcolor}\rmfamily\fontsize{12.000000}{14.400000}\selectfont Signal}%
\end{pgfscope}%
\begin{pgfscope}%
\pgfsetbuttcap%
\pgfsetmiterjoin%
\definecolor{currentfill}{rgb}{1.000000,1.000000,1.000000}%
\pgfsetfillcolor{currentfill}%
\pgfsetlinewidth{0.000000pt}%
\definecolor{currentstroke}{rgb}{0.000000,0.000000,0.000000}%
\pgfsetstrokecolor{currentstroke}%
\pgfsetstrokeopacity{0.000000}%
\pgfsetdash{}{0pt}%
\pgfpathmoveto{\pgfqpoint{0.649696in}{0.456793in}}%
\pgfpathlineto{\pgfqpoint{6.358330in}{0.456793in}}%
\pgfpathlineto{\pgfqpoint{6.358330in}{2.163512in}}%
\pgfpathlineto{\pgfqpoint{0.649696in}{2.163512in}}%
\pgfpathclose%
\pgfusepath{fill}%
\end{pgfscope}%
\begin{pgfscope}%
\pgfpathrectangle{\pgfqpoint{0.649696in}{0.456793in}}{\pgfqpoint{5.708634in}{1.706718in}}%
\pgfusepath{clip}%
\pgfsetrectcap%
\pgfsetroundjoin%
\pgfsetlinewidth{0.803000pt}%
\definecolor{currentstroke}{rgb}{0.690196,0.690196,0.690196}%
\pgfsetstrokecolor{currentstroke}%
\pgfsetdash{}{0pt}%
\pgfpathmoveto{\pgfqpoint{0.909179in}{0.456793in}}%
\pgfpathlineto{\pgfqpoint{0.909179in}{2.163512in}}%
\pgfusepath{stroke}%
\end{pgfscope}%
\begin{pgfscope}%
\pgfsetbuttcap%
\pgfsetroundjoin%
\definecolor{currentfill}{rgb}{0.000000,0.000000,0.000000}%
\pgfsetfillcolor{currentfill}%
\pgfsetlinewidth{0.803000pt}%
\definecolor{currentstroke}{rgb}{0.000000,0.000000,0.000000}%
\pgfsetstrokecolor{currentstroke}%
\pgfsetdash{}{0pt}%
\pgfsys@defobject{currentmarker}{\pgfqpoint{0.000000in}{-0.048611in}}{\pgfqpoint{0.000000in}{0.000000in}}{%
\pgfpathmoveto{\pgfqpoint{0.000000in}{0.000000in}}%
\pgfpathlineto{\pgfqpoint{0.000000in}{-0.048611in}}%
\pgfusepath{stroke,fill}%
}%
\begin{pgfscope}%
\pgfsys@transformshift{0.909179in}{0.456793in}%
\pgfsys@useobject{currentmarker}{}%
\end{pgfscope}%
\end{pgfscope}%
\begin{pgfscope}%
\definecolor{textcolor}{rgb}{0.000000,0.000000,0.000000}%
\pgfsetstrokecolor{textcolor}%
\pgfsetfillcolor{textcolor}%
\pgftext[x=0.909179in,y=0.359571in,,top]{\color{textcolor}\rmfamily\fontsize{10.000000}{12.000000}\selectfont \(\displaystyle 0.0\)}%
\end{pgfscope}%
\begin{pgfscope}%
\pgfpathrectangle{\pgfqpoint{0.649696in}{0.456793in}}{\pgfqpoint{5.708634in}{1.706718in}}%
\pgfusepath{clip}%
\pgfsetrectcap%
\pgfsetroundjoin%
\pgfsetlinewidth{0.803000pt}%
\definecolor{currentstroke}{rgb}{0.690196,0.690196,0.690196}%
\pgfsetstrokecolor{currentstroke}%
\pgfsetdash{}{0pt}%
\pgfpathmoveto{\pgfqpoint{1.947048in}{0.456793in}}%
\pgfpathlineto{\pgfqpoint{1.947048in}{2.163512in}}%
\pgfusepath{stroke}%
\end{pgfscope}%
\begin{pgfscope}%
\pgfsetbuttcap%
\pgfsetroundjoin%
\definecolor{currentfill}{rgb}{0.000000,0.000000,0.000000}%
\pgfsetfillcolor{currentfill}%
\pgfsetlinewidth{0.803000pt}%
\definecolor{currentstroke}{rgb}{0.000000,0.000000,0.000000}%
\pgfsetstrokecolor{currentstroke}%
\pgfsetdash{}{0pt}%
\pgfsys@defobject{currentmarker}{\pgfqpoint{0.000000in}{-0.048611in}}{\pgfqpoint{0.000000in}{0.000000in}}{%
\pgfpathmoveto{\pgfqpoint{0.000000in}{0.000000in}}%
\pgfpathlineto{\pgfqpoint{0.000000in}{-0.048611in}}%
\pgfusepath{stroke,fill}%
}%
\begin{pgfscope}%
\pgfsys@transformshift{1.947048in}{0.456793in}%
\pgfsys@useobject{currentmarker}{}%
\end{pgfscope}%
\end{pgfscope}%
\begin{pgfscope}%
\definecolor{textcolor}{rgb}{0.000000,0.000000,0.000000}%
\pgfsetstrokecolor{textcolor}%
\pgfsetfillcolor{textcolor}%
\pgftext[x=1.947048in,y=0.359571in,,top]{\color{textcolor}\rmfamily\fontsize{10.000000}{12.000000}\selectfont \(\displaystyle 0.2\)}%
\end{pgfscope}%
\begin{pgfscope}%
\pgfpathrectangle{\pgfqpoint{0.649696in}{0.456793in}}{\pgfqpoint{5.708634in}{1.706718in}}%
\pgfusepath{clip}%
\pgfsetrectcap%
\pgfsetroundjoin%
\pgfsetlinewidth{0.803000pt}%
\definecolor{currentstroke}{rgb}{0.690196,0.690196,0.690196}%
\pgfsetstrokecolor{currentstroke}%
\pgfsetdash{}{0pt}%
\pgfpathmoveto{\pgfqpoint{2.984916in}{0.456793in}}%
\pgfpathlineto{\pgfqpoint{2.984916in}{2.163512in}}%
\pgfusepath{stroke}%
\end{pgfscope}%
\begin{pgfscope}%
\pgfsetbuttcap%
\pgfsetroundjoin%
\definecolor{currentfill}{rgb}{0.000000,0.000000,0.000000}%
\pgfsetfillcolor{currentfill}%
\pgfsetlinewidth{0.803000pt}%
\definecolor{currentstroke}{rgb}{0.000000,0.000000,0.000000}%
\pgfsetstrokecolor{currentstroke}%
\pgfsetdash{}{0pt}%
\pgfsys@defobject{currentmarker}{\pgfqpoint{0.000000in}{-0.048611in}}{\pgfqpoint{0.000000in}{0.000000in}}{%
\pgfpathmoveto{\pgfqpoint{0.000000in}{0.000000in}}%
\pgfpathlineto{\pgfqpoint{0.000000in}{-0.048611in}}%
\pgfusepath{stroke,fill}%
}%
\begin{pgfscope}%
\pgfsys@transformshift{2.984916in}{0.456793in}%
\pgfsys@useobject{currentmarker}{}%
\end{pgfscope}%
\end{pgfscope}%
\begin{pgfscope}%
\definecolor{textcolor}{rgb}{0.000000,0.000000,0.000000}%
\pgfsetstrokecolor{textcolor}%
\pgfsetfillcolor{textcolor}%
\pgftext[x=2.984916in,y=0.359571in,,top]{\color{textcolor}\rmfamily\fontsize{10.000000}{12.000000}\selectfont \(\displaystyle 0.4\)}%
\end{pgfscope}%
\begin{pgfscope}%
\pgfpathrectangle{\pgfqpoint{0.649696in}{0.456793in}}{\pgfqpoint{5.708634in}{1.706718in}}%
\pgfusepath{clip}%
\pgfsetrectcap%
\pgfsetroundjoin%
\pgfsetlinewidth{0.803000pt}%
\definecolor{currentstroke}{rgb}{0.690196,0.690196,0.690196}%
\pgfsetstrokecolor{currentstroke}%
\pgfsetdash{}{0pt}%
\pgfpathmoveto{\pgfqpoint{4.022785in}{0.456793in}}%
\pgfpathlineto{\pgfqpoint{4.022785in}{2.163512in}}%
\pgfusepath{stroke}%
\end{pgfscope}%
\begin{pgfscope}%
\pgfsetbuttcap%
\pgfsetroundjoin%
\definecolor{currentfill}{rgb}{0.000000,0.000000,0.000000}%
\pgfsetfillcolor{currentfill}%
\pgfsetlinewidth{0.803000pt}%
\definecolor{currentstroke}{rgb}{0.000000,0.000000,0.000000}%
\pgfsetstrokecolor{currentstroke}%
\pgfsetdash{}{0pt}%
\pgfsys@defobject{currentmarker}{\pgfqpoint{0.000000in}{-0.048611in}}{\pgfqpoint{0.000000in}{0.000000in}}{%
\pgfpathmoveto{\pgfqpoint{0.000000in}{0.000000in}}%
\pgfpathlineto{\pgfqpoint{0.000000in}{-0.048611in}}%
\pgfusepath{stroke,fill}%
}%
\begin{pgfscope}%
\pgfsys@transformshift{4.022785in}{0.456793in}%
\pgfsys@useobject{currentmarker}{}%
\end{pgfscope}%
\end{pgfscope}%
\begin{pgfscope}%
\definecolor{textcolor}{rgb}{0.000000,0.000000,0.000000}%
\pgfsetstrokecolor{textcolor}%
\pgfsetfillcolor{textcolor}%
\pgftext[x=4.022785in,y=0.359571in,,top]{\color{textcolor}\rmfamily\fontsize{10.000000}{12.000000}\selectfont \(\displaystyle 0.6\)}%
\end{pgfscope}%
\begin{pgfscope}%
\pgfpathrectangle{\pgfqpoint{0.649696in}{0.456793in}}{\pgfqpoint{5.708634in}{1.706718in}}%
\pgfusepath{clip}%
\pgfsetrectcap%
\pgfsetroundjoin%
\pgfsetlinewidth{0.803000pt}%
\definecolor{currentstroke}{rgb}{0.690196,0.690196,0.690196}%
\pgfsetstrokecolor{currentstroke}%
\pgfsetdash{}{0pt}%
\pgfpathmoveto{\pgfqpoint{5.060654in}{0.456793in}}%
\pgfpathlineto{\pgfqpoint{5.060654in}{2.163512in}}%
\pgfusepath{stroke}%
\end{pgfscope}%
\begin{pgfscope}%
\pgfsetbuttcap%
\pgfsetroundjoin%
\definecolor{currentfill}{rgb}{0.000000,0.000000,0.000000}%
\pgfsetfillcolor{currentfill}%
\pgfsetlinewidth{0.803000pt}%
\definecolor{currentstroke}{rgb}{0.000000,0.000000,0.000000}%
\pgfsetstrokecolor{currentstroke}%
\pgfsetdash{}{0pt}%
\pgfsys@defobject{currentmarker}{\pgfqpoint{0.000000in}{-0.048611in}}{\pgfqpoint{0.000000in}{0.000000in}}{%
\pgfpathmoveto{\pgfqpoint{0.000000in}{0.000000in}}%
\pgfpathlineto{\pgfqpoint{0.000000in}{-0.048611in}}%
\pgfusepath{stroke,fill}%
}%
\begin{pgfscope}%
\pgfsys@transformshift{5.060654in}{0.456793in}%
\pgfsys@useobject{currentmarker}{}%
\end{pgfscope}%
\end{pgfscope}%
\begin{pgfscope}%
\definecolor{textcolor}{rgb}{0.000000,0.000000,0.000000}%
\pgfsetstrokecolor{textcolor}%
\pgfsetfillcolor{textcolor}%
\pgftext[x=5.060654in,y=0.359571in,,top]{\color{textcolor}\rmfamily\fontsize{10.000000}{12.000000}\selectfont \(\displaystyle 0.8\)}%
\end{pgfscope}%
\begin{pgfscope}%
\pgfpathrectangle{\pgfqpoint{0.649696in}{0.456793in}}{\pgfqpoint{5.708634in}{1.706718in}}%
\pgfusepath{clip}%
\pgfsetrectcap%
\pgfsetroundjoin%
\pgfsetlinewidth{0.803000pt}%
\definecolor{currentstroke}{rgb}{0.690196,0.690196,0.690196}%
\pgfsetstrokecolor{currentstroke}%
\pgfsetdash{}{0pt}%
\pgfpathmoveto{\pgfqpoint{6.098522in}{0.456793in}}%
\pgfpathlineto{\pgfqpoint{6.098522in}{2.163512in}}%
\pgfusepath{stroke}%
\end{pgfscope}%
\begin{pgfscope}%
\pgfsetbuttcap%
\pgfsetroundjoin%
\definecolor{currentfill}{rgb}{0.000000,0.000000,0.000000}%
\pgfsetfillcolor{currentfill}%
\pgfsetlinewidth{0.803000pt}%
\definecolor{currentstroke}{rgb}{0.000000,0.000000,0.000000}%
\pgfsetstrokecolor{currentstroke}%
\pgfsetdash{}{0pt}%
\pgfsys@defobject{currentmarker}{\pgfqpoint{0.000000in}{-0.048611in}}{\pgfqpoint{0.000000in}{0.000000in}}{%
\pgfpathmoveto{\pgfqpoint{0.000000in}{0.000000in}}%
\pgfpathlineto{\pgfqpoint{0.000000in}{-0.048611in}}%
\pgfusepath{stroke,fill}%
}%
\begin{pgfscope}%
\pgfsys@transformshift{6.098522in}{0.456793in}%
\pgfsys@useobject{currentmarker}{}%
\end{pgfscope}%
\end{pgfscope}%
\begin{pgfscope}%
\definecolor{textcolor}{rgb}{0.000000,0.000000,0.000000}%
\pgfsetstrokecolor{textcolor}%
\pgfsetfillcolor{textcolor}%
\pgftext[x=6.098522in,y=0.359571in,,top]{\color{textcolor}\rmfamily\fontsize{10.000000}{12.000000}\selectfont \(\displaystyle 1.0\)}%
\end{pgfscope}%
\begin{pgfscope}%
\definecolor{textcolor}{rgb}{0.000000,0.000000,0.000000}%
\pgfsetstrokecolor{textcolor}%
\pgfsetfillcolor{textcolor}%
\pgftext[x=3.504013in,y=0.180559in,,top]{\color{textcolor}\rmfamily\fontsize{10.000000}{12.000000}\selectfont Frequency (f)}%
\end{pgfscope}%
\begin{pgfscope}%
\definecolor{textcolor}{rgb}{0.000000,0.000000,0.000000}%
\pgfsetstrokecolor{textcolor}%
\pgfsetfillcolor{textcolor}%
\pgftext[x=6.358330in,y=0.194448in,right,top]{\color{textcolor}\rmfamily\fontsize{10.000000}{12.000000}\selectfont \(\displaystyle \times10^{6}\)}%
\end{pgfscope}%
\begin{pgfscope}%
\pgfpathrectangle{\pgfqpoint{0.649696in}{0.456793in}}{\pgfqpoint{5.708634in}{1.706718in}}%
\pgfusepath{clip}%
\pgfsetrectcap%
\pgfsetroundjoin%
\pgfsetlinewidth{0.803000pt}%
\definecolor{currentstroke}{rgb}{0.690196,0.690196,0.690196}%
\pgfsetstrokecolor{currentstroke}%
\pgfsetdash{}{0pt}%
\pgfpathmoveto{\pgfqpoint{0.649696in}{0.456793in}}%
\pgfpathlineto{\pgfqpoint{6.358330in}{0.456793in}}%
\pgfusepath{stroke}%
\end{pgfscope}%
\begin{pgfscope}%
\pgfsetbuttcap%
\pgfsetroundjoin%
\definecolor{currentfill}{rgb}{0.000000,0.000000,0.000000}%
\pgfsetfillcolor{currentfill}%
\pgfsetlinewidth{0.803000pt}%
\definecolor{currentstroke}{rgb}{0.000000,0.000000,0.000000}%
\pgfsetstrokecolor{currentstroke}%
\pgfsetdash{}{0pt}%
\pgfsys@defobject{currentmarker}{\pgfqpoint{-0.048611in}{0.000000in}}{\pgfqpoint{0.000000in}{0.000000in}}{%
\pgfpathmoveto{\pgfqpoint{0.000000in}{0.000000in}}%
\pgfpathlineto{\pgfqpoint{-0.048611in}{0.000000in}}%
\pgfusepath{stroke,fill}%
}%
\begin{pgfscope}%
\pgfsys@transformshift{0.649696in}{0.456793in}%
\pgfsys@useobject{currentmarker}{}%
\end{pgfscope}%
\end{pgfscope}%
\begin{pgfscope}%
\definecolor{textcolor}{rgb}{0.000000,0.000000,0.000000}%
\pgfsetstrokecolor{textcolor}%
\pgfsetfillcolor{textcolor}%
\pgftext[x=0.236114in, y=0.408568in, left, base]{\color{textcolor}\rmfamily\fontsize{10.000000}{12.000000}\selectfont \(\displaystyle -100\)}%
\end{pgfscope}%
\begin{pgfscope}%
\pgfpathrectangle{\pgfqpoint{0.649696in}{0.456793in}}{\pgfqpoint{5.708634in}{1.706718in}}%
\pgfusepath{clip}%
\pgfsetrectcap%
\pgfsetroundjoin%
\pgfsetlinewidth{0.803000pt}%
\definecolor{currentstroke}{rgb}{0.690196,0.690196,0.690196}%
\pgfsetstrokecolor{currentstroke}%
\pgfsetdash{}{0pt}%
\pgfpathmoveto{\pgfqpoint{0.649696in}{0.753839in}}%
\pgfpathlineto{\pgfqpoint{6.358330in}{0.753839in}}%
\pgfusepath{stroke}%
\end{pgfscope}%
\begin{pgfscope}%
\pgfsetbuttcap%
\pgfsetroundjoin%
\definecolor{currentfill}{rgb}{0.000000,0.000000,0.000000}%
\pgfsetfillcolor{currentfill}%
\pgfsetlinewidth{0.803000pt}%
\definecolor{currentstroke}{rgb}{0.000000,0.000000,0.000000}%
\pgfsetstrokecolor{currentstroke}%
\pgfsetdash{}{0pt}%
\pgfsys@defobject{currentmarker}{\pgfqpoint{-0.048611in}{0.000000in}}{\pgfqpoint{0.000000in}{0.000000in}}{%
\pgfpathmoveto{\pgfqpoint{0.000000in}{0.000000in}}%
\pgfpathlineto{\pgfqpoint{-0.048611in}{0.000000in}}%
\pgfusepath{stroke,fill}%
}%
\begin{pgfscope}%
\pgfsys@transformshift{0.649696in}{0.753839in}%
\pgfsys@useobject{currentmarker}{}%
\end{pgfscope}%
\end{pgfscope}%
\begin{pgfscope}%
\definecolor{textcolor}{rgb}{0.000000,0.000000,0.000000}%
\pgfsetstrokecolor{textcolor}%
\pgfsetfillcolor{textcolor}%
\pgftext[x=0.305559in, y=0.705614in, left, base]{\color{textcolor}\rmfamily\fontsize{10.000000}{12.000000}\selectfont \(\displaystyle -90\)}%
\end{pgfscope}%
\begin{pgfscope}%
\pgfpathrectangle{\pgfqpoint{0.649696in}{0.456793in}}{\pgfqpoint{5.708634in}{1.706718in}}%
\pgfusepath{clip}%
\pgfsetrectcap%
\pgfsetroundjoin%
\pgfsetlinewidth{0.803000pt}%
\definecolor{currentstroke}{rgb}{0.690196,0.690196,0.690196}%
\pgfsetstrokecolor{currentstroke}%
\pgfsetdash{}{0pt}%
\pgfpathmoveto{\pgfqpoint{0.649696in}{1.050886in}}%
\pgfpathlineto{\pgfqpoint{6.358330in}{1.050886in}}%
\pgfusepath{stroke}%
\end{pgfscope}%
\begin{pgfscope}%
\pgfsetbuttcap%
\pgfsetroundjoin%
\definecolor{currentfill}{rgb}{0.000000,0.000000,0.000000}%
\pgfsetfillcolor{currentfill}%
\pgfsetlinewidth{0.803000pt}%
\definecolor{currentstroke}{rgb}{0.000000,0.000000,0.000000}%
\pgfsetstrokecolor{currentstroke}%
\pgfsetdash{}{0pt}%
\pgfsys@defobject{currentmarker}{\pgfqpoint{-0.048611in}{0.000000in}}{\pgfqpoint{0.000000in}{0.000000in}}{%
\pgfpathmoveto{\pgfqpoint{0.000000in}{0.000000in}}%
\pgfpathlineto{\pgfqpoint{-0.048611in}{0.000000in}}%
\pgfusepath{stroke,fill}%
}%
\begin{pgfscope}%
\pgfsys@transformshift{0.649696in}{1.050886in}%
\pgfsys@useobject{currentmarker}{}%
\end{pgfscope}%
\end{pgfscope}%
\begin{pgfscope}%
\definecolor{textcolor}{rgb}{0.000000,0.000000,0.000000}%
\pgfsetstrokecolor{textcolor}%
\pgfsetfillcolor{textcolor}%
\pgftext[x=0.305559in, y=1.002660in, left, base]{\color{textcolor}\rmfamily\fontsize{10.000000}{12.000000}\selectfont \(\displaystyle -80\)}%
\end{pgfscope}%
\begin{pgfscope}%
\pgfpathrectangle{\pgfqpoint{0.649696in}{0.456793in}}{\pgfqpoint{5.708634in}{1.706718in}}%
\pgfusepath{clip}%
\pgfsetrectcap%
\pgfsetroundjoin%
\pgfsetlinewidth{0.803000pt}%
\definecolor{currentstroke}{rgb}{0.690196,0.690196,0.690196}%
\pgfsetstrokecolor{currentstroke}%
\pgfsetdash{}{0pt}%
\pgfpathmoveto{\pgfqpoint{0.649696in}{1.347932in}}%
\pgfpathlineto{\pgfqpoint{6.358330in}{1.347932in}}%
\pgfusepath{stroke}%
\end{pgfscope}%
\begin{pgfscope}%
\pgfsetbuttcap%
\pgfsetroundjoin%
\definecolor{currentfill}{rgb}{0.000000,0.000000,0.000000}%
\pgfsetfillcolor{currentfill}%
\pgfsetlinewidth{0.803000pt}%
\definecolor{currentstroke}{rgb}{0.000000,0.000000,0.000000}%
\pgfsetstrokecolor{currentstroke}%
\pgfsetdash{}{0pt}%
\pgfsys@defobject{currentmarker}{\pgfqpoint{-0.048611in}{0.000000in}}{\pgfqpoint{0.000000in}{0.000000in}}{%
\pgfpathmoveto{\pgfqpoint{0.000000in}{0.000000in}}%
\pgfpathlineto{\pgfqpoint{-0.048611in}{0.000000in}}%
\pgfusepath{stroke,fill}%
}%
\begin{pgfscope}%
\pgfsys@transformshift{0.649696in}{1.347932in}%
\pgfsys@useobject{currentmarker}{}%
\end{pgfscope}%
\end{pgfscope}%
\begin{pgfscope}%
\definecolor{textcolor}{rgb}{0.000000,0.000000,0.000000}%
\pgfsetstrokecolor{textcolor}%
\pgfsetfillcolor{textcolor}%
\pgftext[x=0.305559in, y=1.299706in, left, base]{\color{textcolor}\rmfamily\fontsize{10.000000}{12.000000}\selectfont \(\displaystyle -70\)}%
\end{pgfscope}%
\begin{pgfscope}%
\pgfpathrectangle{\pgfqpoint{0.649696in}{0.456793in}}{\pgfqpoint{5.708634in}{1.706718in}}%
\pgfusepath{clip}%
\pgfsetrectcap%
\pgfsetroundjoin%
\pgfsetlinewidth{0.803000pt}%
\definecolor{currentstroke}{rgb}{0.690196,0.690196,0.690196}%
\pgfsetstrokecolor{currentstroke}%
\pgfsetdash{}{0pt}%
\pgfpathmoveto{\pgfqpoint{0.649696in}{1.644978in}}%
\pgfpathlineto{\pgfqpoint{6.358330in}{1.644978in}}%
\pgfusepath{stroke}%
\end{pgfscope}%
\begin{pgfscope}%
\pgfsetbuttcap%
\pgfsetroundjoin%
\definecolor{currentfill}{rgb}{0.000000,0.000000,0.000000}%
\pgfsetfillcolor{currentfill}%
\pgfsetlinewidth{0.803000pt}%
\definecolor{currentstroke}{rgb}{0.000000,0.000000,0.000000}%
\pgfsetstrokecolor{currentstroke}%
\pgfsetdash{}{0pt}%
\pgfsys@defobject{currentmarker}{\pgfqpoint{-0.048611in}{0.000000in}}{\pgfqpoint{0.000000in}{0.000000in}}{%
\pgfpathmoveto{\pgfqpoint{0.000000in}{0.000000in}}%
\pgfpathlineto{\pgfqpoint{-0.048611in}{0.000000in}}%
\pgfusepath{stroke,fill}%
}%
\begin{pgfscope}%
\pgfsys@transformshift{0.649696in}{1.644978in}%
\pgfsys@useobject{currentmarker}{}%
\end{pgfscope}%
\end{pgfscope}%
\begin{pgfscope}%
\definecolor{textcolor}{rgb}{0.000000,0.000000,0.000000}%
\pgfsetstrokecolor{textcolor}%
\pgfsetfillcolor{textcolor}%
\pgftext[x=0.305559in, y=1.596752in, left, base]{\color{textcolor}\rmfamily\fontsize{10.000000}{12.000000}\selectfont \(\displaystyle -60\)}%
\end{pgfscope}%
\begin{pgfscope}%
\pgfpathrectangle{\pgfqpoint{0.649696in}{0.456793in}}{\pgfqpoint{5.708634in}{1.706718in}}%
\pgfusepath{clip}%
\pgfsetrectcap%
\pgfsetroundjoin%
\pgfsetlinewidth{0.803000pt}%
\definecolor{currentstroke}{rgb}{0.690196,0.690196,0.690196}%
\pgfsetstrokecolor{currentstroke}%
\pgfsetdash{}{0pt}%
\pgfpathmoveto{\pgfqpoint{0.649696in}{1.942024in}}%
\pgfpathlineto{\pgfqpoint{6.358330in}{1.942024in}}%
\pgfusepath{stroke}%
\end{pgfscope}%
\begin{pgfscope}%
\pgfsetbuttcap%
\pgfsetroundjoin%
\definecolor{currentfill}{rgb}{0.000000,0.000000,0.000000}%
\pgfsetfillcolor{currentfill}%
\pgfsetlinewidth{0.803000pt}%
\definecolor{currentstroke}{rgb}{0.000000,0.000000,0.000000}%
\pgfsetstrokecolor{currentstroke}%
\pgfsetdash{}{0pt}%
\pgfsys@defobject{currentmarker}{\pgfqpoint{-0.048611in}{0.000000in}}{\pgfqpoint{0.000000in}{0.000000in}}{%
\pgfpathmoveto{\pgfqpoint{0.000000in}{0.000000in}}%
\pgfpathlineto{\pgfqpoint{-0.048611in}{0.000000in}}%
\pgfusepath{stroke,fill}%
}%
\begin{pgfscope}%
\pgfsys@transformshift{0.649696in}{1.942024in}%
\pgfsys@useobject{currentmarker}{}%
\end{pgfscope}%
\end{pgfscope}%
\begin{pgfscope}%
\definecolor{textcolor}{rgb}{0.000000,0.000000,0.000000}%
\pgfsetstrokecolor{textcolor}%
\pgfsetfillcolor{textcolor}%
\pgftext[x=0.305559in, y=1.893799in, left, base]{\color{textcolor}\rmfamily\fontsize{10.000000}{12.000000}\selectfont \(\displaystyle -50\)}%
\end{pgfscope}%
\begin{pgfscope}%
\definecolor{textcolor}{rgb}{0.000000,0.000000,0.000000}%
\pgfsetstrokecolor{textcolor}%
\pgfsetfillcolor{textcolor}%
\pgftext[x=0.180559in,y=1.310153in,,bottom,rotate=90.000000]{\color{textcolor}\rmfamily\fontsize{10.000000}{12.000000}\selectfont Power Spectral Density (dB/Hz)}%
\end{pgfscope}%
\begin{pgfscope}%
\pgfpathrectangle{\pgfqpoint{0.649696in}{0.456793in}}{\pgfqpoint{5.708634in}{1.706718in}}%
\pgfusepath{clip}%
\pgfsetrectcap%
\pgfsetroundjoin%
\pgfsetlinewidth{1.505625pt}%
\definecolor{currentstroke}{rgb}{0.121569,0.466667,0.705882}%
\pgfsetstrokecolor{currentstroke}%
\pgfsetdash{}{0pt}%
\pgfpathmoveto{\pgfqpoint{0.909179in}{1.910986in}}%
\pgfpathlineto{\pgfqpoint{0.929451in}{2.039256in}}%
\pgfpathlineto{\pgfqpoint{0.949723in}{2.024882in}}%
\pgfpathlineto{\pgfqpoint{0.969995in}{2.046553in}}%
\pgfpathlineto{\pgfqpoint{0.990268in}{2.026046in}}%
\pgfpathlineto{\pgfqpoint{1.010540in}{1.995607in}}%
\pgfpathlineto{\pgfqpoint{1.030812in}{1.948588in}}%
\pgfpathlineto{\pgfqpoint{1.051084in}{2.029989in}}%
\pgfpathlineto{\pgfqpoint{1.071356in}{2.024012in}}%
\pgfpathlineto{\pgfqpoint{1.091628in}{2.060603in}}%
\pgfpathlineto{\pgfqpoint{1.111900in}{2.064664in}}%
\pgfpathlineto{\pgfqpoint{1.132173in}{2.087209in}}%
\pgfpathlineto{\pgfqpoint{1.152445in}{2.047071in}}%
\pgfpathlineto{\pgfqpoint{1.172717in}{1.957626in}}%
\pgfpathlineto{\pgfqpoint{1.192989in}{1.994536in}}%
\pgfpathlineto{\pgfqpoint{1.213261in}{2.020017in}}%
\pgfpathlineto{\pgfqpoint{1.233533in}{2.024950in}}%
\pgfpathlineto{\pgfqpoint{1.253805in}{2.015628in}}%
\pgfpathlineto{\pgfqpoint{1.274078in}{2.010461in}}%
\pgfpathlineto{\pgfqpoint{1.294350in}{1.992861in}}%
\pgfpathlineto{\pgfqpoint{1.314622in}{2.015988in}}%
\pgfpathlineto{\pgfqpoint{1.334894in}{2.001693in}}%
\pgfpathlineto{\pgfqpoint{1.355166in}{1.949497in}}%
\pgfpathlineto{\pgfqpoint{1.375438in}{1.913415in}}%
\pgfpathlineto{\pgfqpoint{1.395710in}{1.918489in}}%
\pgfpathlineto{\pgfqpoint{1.415983in}{1.909078in}}%
\pgfpathlineto{\pgfqpoint{1.436255in}{1.906143in}}%
\pgfpathlineto{\pgfqpoint{1.456527in}{1.894777in}}%
\pgfpathlineto{\pgfqpoint{1.476799in}{1.923965in}}%
\pgfpathlineto{\pgfqpoint{1.497071in}{1.922275in}}%
\pgfpathlineto{\pgfqpoint{1.517343in}{1.852628in}}%
\pgfpathlineto{\pgfqpoint{1.537615in}{1.921977in}}%
\pgfpathlineto{\pgfqpoint{1.557888in}{1.949884in}}%
\pgfpathlineto{\pgfqpoint{1.578160in}{1.909212in}}%
\pgfpathlineto{\pgfqpoint{1.598432in}{1.820146in}}%
\pgfpathlineto{\pgfqpoint{1.618704in}{1.872343in}}%
\pgfpathlineto{\pgfqpoint{1.638976in}{1.861940in}}%
\pgfpathlineto{\pgfqpoint{1.659248in}{1.813940in}}%
\pgfpathlineto{\pgfqpoint{1.679520in}{1.810854in}}%
\pgfpathlineto{\pgfqpoint{1.699792in}{1.796282in}}%
\pgfpathlineto{\pgfqpoint{1.720065in}{1.788475in}}%
\pgfpathlineto{\pgfqpoint{1.740337in}{1.766131in}}%
\pgfpathlineto{\pgfqpoint{1.760609in}{1.782667in}}%
\pgfpathlineto{\pgfqpoint{1.780881in}{1.817324in}}%
\pgfpathlineto{\pgfqpoint{1.801153in}{1.816591in}}%
\pgfpathlineto{\pgfqpoint{1.821425in}{1.773623in}}%
\pgfpathlineto{\pgfqpoint{1.841697in}{1.770234in}}%
\pgfpathlineto{\pgfqpoint{1.882242in}{1.744149in}}%
\pgfpathlineto{\pgfqpoint{1.902514in}{1.719030in}}%
\pgfpathlineto{\pgfqpoint{1.922786in}{1.673533in}}%
\pgfpathlineto{\pgfqpoint{1.943058in}{1.607685in}}%
\pgfpathlineto{\pgfqpoint{1.963330in}{1.666821in}}%
\pgfpathlineto{\pgfqpoint{1.983602in}{1.688321in}}%
\pgfpathlineto{\pgfqpoint{2.003875in}{1.635024in}}%
\pgfpathlineto{\pgfqpoint{2.024147in}{1.599658in}}%
\pgfpathlineto{\pgfqpoint{2.044419in}{1.529701in}}%
\pgfpathlineto{\pgfqpoint{2.064691in}{1.496815in}}%
\pgfpathlineto{\pgfqpoint{2.084963in}{1.381173in}}%
\pgfpathlineto{\pgfqpoint{2.105235in}{1.361166in}}%
\pgfpathlineto{\pgfqpoint{2.125507in}{1.331090in}}%
\pgfpathlineto{\pgfqpoint{2.145780in}{1.278501in}}%
\pgfpathlineto{\pgfqpoint{2.166052in}{1.160510in}}%
\pgfpathlineto{\pgfqpoint{2.186324in}{0.996347in}}%
\pgfpathlineto{\pgfqpoint{2.206596in}{0.898496in}}%
\pgfpathlineto{\pgfqpoint{2.226868in}{0.986097in}}%
\pgfpathlineto{\pgfqpoint{2.247140in}{1.142112in}}%
\pgfpathlineto{\pgfqpoint{2.267412in}{1.254412in}}%
\pgfpathlineto{\pgfqpoint{2.287685in}{1.299558in}}%
\pgfpathlineto{\pgfqpoint{2.307957in}{1.322732in}}%
\pgfpathlineto{\pgfqpoint{2.328229in}{1.332376in}}%
\pgfpathlineto{\pgfqpoint{2.348501in}{1.442210in}}%
\pgfpathlineto{\pgfqpoint{2.368773in}{1.467108in}}%
\pgfpathlineto{\pgfqpoint{2.389045in}{1.529772in}}%
\pgfpathlineto{\pgfqpoint{2.409317in}{1.557271in}}%
\pgfpathlineto{\pgfqpoint{2.429589in}{1.602385in}}%
\pgfpathlineto{\pgfqpoint{2.449862in}{1.575543in}}%
\pgfpathlineto{\pgfqpoint{2.470134in}{1.506953in}}%
\pgfpathlineto{\pgfqpoint{2.490406in}{1.563132in}}%
\pgfpathlineto{\pgfqpoint{2.510678in}{1.601532in}}%
\pgfpathlineto{\pgfqpoint{2.530950in}{1.619152in}}%
\pgfpathlineto{\pgfqpoint{2.571494in}{1.628916in}}%
\pgfpathlineto{\pgfqpoint{2.591767in}{1.623235in}}%
\pgfpathlineto{\pgfqpoint{2.612039in}{1.657259in}}%
\pgfpathlineto{\pgfqpoint{2.632311in}{1.650997in}}%
\pgfpathlineto{\pgfqpoint{2.652583in}{1.607854in}}%
\pgfpathlineto{\pgfqpoint{2.672855in}{1.581499in}}%
\pgfpathlineto{\pgfqpoint{2.693127in}{1.594915in}}%
\pgfpathlineto{\pgfqpoint{2.713399in}{1.593493in}}%
\pgfpathlineto{\pgfqpoint{2.733672in}{1.598383in}}%
\pgfpathlineto{\pgfqpoint{2.753944in}{1.593292in}}%
\pgfpathlineto{\pgfqpoint{2.774216in}{1.630332in}}%
\pgfpathlineto{\pgfqpoint{2.794488in}{1.633482in}}%
\pgfpathlineto{\pgfqpoint{2.814760in}{1.570581in}}%
\pgfpathlineto{\pgfqpoint{2.835032in}{1.647328in}}%
\pgfpathlineto{\pgfqpoint{2.855304in}{1.679742in}}%
\pgfpathlineto{\pgfqpoint{2.875577in}{1.643530in}}%
\pgfpathlineto{\pgfqpoint{2.895849in}{1.560937in}}%
\pgfpathlineto{\pgfqpoint{2.916121in}{1.618672in}}%
\pgfpathlineto{\pgfqpoint{2.936393in}{1.611961in}}%
\pgfpathlineto{\pgfqpoint{2.956665in}{1.569427in}}%
\pgfpathlineto{\pgfqpoint{2.976937in}{1.570333in}}%
\pgfpathlineto{\pgfqpoint{2.997209in}{1.560426in}}%
\pgfpathlineto{\pgfqpoint{3.017482in}{1.556971in}}%
\pgfpathlineto{\pgfqpoint{3.037754in}{1.538759in}}%
\pgfpathlineto{\pgfqpoint{3.058026in}{1.559756in}}%
\pgfpathlineto{\pgfqpoint{3.078298in}{1.598200in}}%
\pgfpathlineto{\pgfqpoint{3.098570in}{1.600518in}}%
\pgfpathlineto{\pgfqpoint{3.118842in}{1.561405in}}%
\pgfpathlineto{\pgfqpoint{3.139114in}{1.561841in}}%
\pgfpathlineto{\pgfqpoint{3.179659in}{1.542445in}}%
\pgfpathlineto{\pgfqpoint{3.199931in}{1.520314in}}%
\pgfpathlineto{\pgfqpoint{3.220203in}{1.477595in}}%
\pgfpathlineto{\pgfqpoint{3.240475in}{1.415462in}}%
\pgfpathlineto{\pgfqpoint{3.260747in}{1.478150in}}%
\pgfpathlineto{\pgfqpoint{3.281019in}{1.501626in}}%
\pgfpathlineto{\pgfqpoint{3.301291in}{1.451279in}}%
\pgfpathlineto{\pgfqpoint{3.321564in}{1.418729in}}%
\pgfpathlineto{\pgfqpoint{3.341836in}{1.351299in}}%
\pgfpathlineto{\pgfqpoint{3.362108in}{1.321165in}}%
\pgfpathlineto{\pgfqpoint{3.382380in}{1.207452in}}%
\pgfpathlineto{\pgfqpoint{3.402652in}{1.190873in}}%
\pgfpathlineto{\pgfqpoint{3.422924in}{1.162999in}}%
\pgfpathlineto{\pgfqpoint{3.443196in}{1.112792in}}%
\pgfpathlineto{\pgfqpoint{3.463469in}{0.996534in}}%
\pgfpathlineto{\pgfqpoint{3.483741in}{0.834899in}}%
\pgfpathlineto{\pgfqpoint{3.504013in}{0.740012in}}%
\pgfpathlineto{\pgfqpoint{3.524285in}{0.830651in}}%
\pgfpathlineto{\pgfqpoint{3.544557in}{0.988918in}}%
\pgfpathlineto{\pgfqpoint{3.564829in}{1.102822in}}%
\pgfpathlineto{\pgfqpoint{3.585101in}{1.149953in}}%
\pgfpathlineto{\pgfqpoint{3.605374in}{1.174981in}}%
\pgfpathlineto{\pgfqpoint{3.625646in}{1.187295in}}%
\pgfpathlineto{\pgfqpoint{3.645918in}{1.298618in}}%
\pgfpathlineto{\pgfqpoint{3.666190in}{1.325483in}}%
\pgfpathlineto{\pgfqpoint{3.686462in}{1.389931in}}%
\pgfpathlineto{\pgfqpoint{3.706734in}{1.419282in}}%
\pgfpathlineto{\pgfqpoint{3.727006in}{1.466305in}}%
\pgfpathlineto{\pgfqpoint{3.747279in}{1.440673in}}%
\pgfpathlineto{\pgfqpoint{3.767551in}{1.374180in}}%
\pgfpathlineto{\pgfqpoint{3.787823in}{1.432444in}}%
\pgfpathlineto{\pgfqpoint{3.808095in}{1.472340in}}%
\pgfpathlineto{\pgfqpoint{3.828367in}{1.491504in}}%
\pgfpathlineto{\pgfqpoint{3.868911in}{1.504503in}}%
\pgfpathlineto{\pgfqpoint{3.889183in}{1.500549in}}%
\pgfpathlineto{\pgfqpoint{3.909456in}{1.536225in}}%
\pgfpathlineto{\pgfqpoint{3.929728in}{1.531234in}}%
\pgfpathlineto{\pgfqpoint{3.950000in}{1.489581in}}%
\pgfpathlineto{\pgfqpoint{3.970272in}{1.464904in}}%
\pgfpathlineto{\pgfqpoint{3.990544in}{1.479803in}}%
\pgfpathlineto{\pgfqpoint{4.010816in}{1.479870in}}%
\pgfpathlineto{\pgfqpoint{4.031088in}{1.486269in}}%
\pgfpathlineto{\pgfqpoint{4.051361in}{1.482426in}}%
\pgfpathlineto{\pgfqpoint{4.071633in}{1.521077in}}%
\pgfpathlineto{\pgfqpoint{4.091905in}{1.525259in}}%
\pgfpathlineto{\pgfqpoint{4.112177in}{1.463834in}}%
\pgfpathlineto{\pgfqpoint{4.132449in}{1.542236in}}%
\pgfpathlineto{\pgfqpoint{4.152721in}{1.575702in}}%
\pgfpathlineto{\pgfqpoint{4.172993in}{1.540540in}}%
\pgfpathlineto{\pgfqpoint{4.193266in}{1.459528in}}%
\pgfpathlineto{\pgfqpoint{4.213538in}{1.518648in}}%
\pgfpathlineto{\pgfqpoint{4.233810in}{1.512879in}}%
\pgfpathlineto{\pgfqpoint{4.254082in}{1.471777in}}%
\pgfpathlineto{\pgfqpoint{4.274354in}{1.473757in}}%
\pgfpathlineto{\pgfqpoint{4.294626in}{1.465124in}}%
\pgfpathlineto{\pgfqpoint{4.314898in}{1.462889in}}%
\pgfpathlineto{\pgfqpoint{4.335171in}{1.445858in}}%
\pgfpathlineto{\pgfqpoint{4.355443in}{1.468158in}}%
\pgfpathlineto{\pgfqpoint{4.375715in}{1.507728in}}%
\pgfpathlineto{\pgfqpoint{4.395987in}{1.510976in}}%
\pgfpathlineto{\pgfqpoint{4.416259in}{1.473051in}}%
\pgfpathlineto{\pgfqpoint{4.436531in}{1.474693in}}%
\pgfpathlineto{\pgfqpoint{4.477076in}{1.457460in}}%
\pgfpathlineto{\pgfqpoint{4.497348in}{1.436319in}}%
\pgfpathlineto{\pgfqpoint{4.517620in}{1.394537in}}%
\pgfpathlineto{\pgfqpoint{4.537892in}{1.333675in}}%
\pgfpathlineto{\pgfqpoint{4.558164in}{1.397603in}}%
\pgfpathlineto{\pgfqpoint{4.578436in}{1.421775in}}%
\pgfpathlineto{\pgfqpoint{4.598708in}{1.372493in}}%
\pgfpathlineto{\pgfqpoint{4.618980in}{1.340960in}}%
\pgfpathlineto{\pgfqpoint{4.639253in}{1.274470in}}%
\pgfpathlineto{\pgfqpoint{4.659525in}{1.245359in}}%
\pgfpathlineto{\pgfqpoint{4.679797in}{1.132387in}}%
\pgfpathlineto{\pgfqpoint{4.700069in}{1.117121in}}%
\pgfpathlineto{\pgfqpoint{4.720341in}{1.090112in}}%
\pgfpathlineto{\pgfqpoint{4.740613in}{1.040837in}}%
\pgfpathlineto{\pgfqpoint{4.760885in}{0.925284in}}%
\pgfpathlineto{\pgfqpoint{4.781158in}{0.764651in}}%
\pgfpathlineto{\pgfqpoint{4.801430in}{0.671011in}}%
\pgfpathlineto{\pgfqpoint{4.821702in}{0.762893in}}%
\pgfpathlineto{\pgfqpoint{4.841974in}{0.922130in}}%
\pgfpathlineto{\pgfqpoint{4.862246in}{1.036707in}}%
\pgfpathlineto{\pgfqpoint{4.882518in}{1.084710in}}%
\pgfpathlineto{\pgfqpoint{4.902790in}{1.110540in}}%
\pgfpathlineto{\pgfqpoint{4.923063in}{1.124041in}}%
\pgfpathlineto{\pgfqpoint{4.943335in}{1.236025in}}%
\pgfpathlineto{\pgfqpoint{4.963607in}{1.263784in}}%
\pgfpathlineto{\pgfqpoint{4.983879in}{1.329042in}}%
\pgfpathlineto{\pgfqpoint{5.004151in}{1.359253in}}%
\pgfpathlineto{\pgfqpoint{5.024423in}{1.407164in}}%
\pgfpathlineto{\pgfqpoint{5.044695in}{1.382103in}}%
\pgfpathlineto{\pgfqpoint{5.064968in}{1.316606in}}%
\pgfpathlineto{\pgfqpoint{5.085240in}{1.375875in}}%
\pgfpathlineto{\pgfqpoint{5.105512in}{1.416496in}}%
\pgfpathlineto{\pgfqpoint{5.125784in}{1.436417in}}%
\pgfpathlineto{\pgfqpoint{5.166328in}{1.451023in}}%
\pgfpathlineto{\pgfqpoint{5.186600in}{1.447941in}}%
\pgfpathlineto{\pgfqpoint{5.206873in}{1.484460in}}%
\pgfpathlineto{\pgfqpoint{5.227145in}{1.480119in}}%
\pgfpathlineto{\pgfqpoint{5.247417in}{1.439239in}}%
\pgfpathlineto{\pgfqpoint{5.267689in}{1.415439in}}%
\pgfpathlineto{\pgfqpoint{5.287961in}{1.431122in}}%
\pgfpathlineto{\pgfqpoint{5.308233in}{1.431980in}}%
\pgfpathlineto{\pgfqpoint{5.328505in}{1.439191in}}%
\pgfpathlineto{\pgfqpoint{5.348777in}{1.436022in}}%
\pgfpathlineto{\pgfqpoint{5.369050in}{1.475554in}}%
\pgfpathlineto{\pgfqpoint{5.389322in}{1.480303in}}%
\pgfpathlineto{\pgfqpoint{5.409594in}{1.419699in}}%
\pgfpathlineto{\pgfqpoint{5.429866in}{1.499025in}}%
\pgfpathlineto{\pgfqpoint{5.450138in}{1.533088in}}%
\pgfpathlineto{\pgfqpoint{5.470410in}{1.498520in}}%
\pgfpathlineto{\pgfqpoint{5.490682in}{1.418417in}}%
\pgfpathlineto{\pgfqpoint{5.510955in}{1.478335in}}%
\pgfpathlineto{\pgfqpoint{5.531227in}{1.473114in}}%
\pgfpathlineto{\pgfqpoint{5.551499in}{1.432852in}}%
\pgfpathlineto{\pgfqpoint{5.571771in}{1.435467in}}%
\pgfpathlineto{\pgfqpoint{5.592043in}{1.427592in}}%
\pgfpathlineto{\pgfqpoint{5.612315in}{1.426088in}}%
\pgfpathlineto{\pgfqpoint{5.632587in}{1.409772in}}%
\pgfpathlineto{\pgfqpoint{5.652860in}{1.432865in}}%
\pgfpathlineto{\pgfqpoint{5.673132in}{1.473126in}}%
\pgfpathlineto{\pgfqpoint{5.693404in}{1.476949in}}%
\pgfpathlineto{\pgfqpoint{5.713676in}{1.439762in}}%
\pgfpathlineto{\pgfqpoint{5.733948in}{1.442158in}}%
\pgfpathlineto{\pgfqpoint{5.774492in}{1.426296in}}%
\pgfpathlineto{\pgfqpoint{5.794765in}{1.405789in}}%
\pgfpathlineto{\pgfqpoint{5.815037in}{1.364608in}}%
\pgfpathlineto{\pgfqpoint{5.835309in}{1.304571in}}%
\pgfpathlineto{\pgfqpoint{5.855581in}{1.369308in}}%
\pgfpathlineto{\pgfqpoint{5.875853in}{1.393938in}}%
\pgfpathlineto{\pgfqpoint{5.896125in}{1.345359in}}%
\pgfpathlineto{\pgfqpoint{5.916397in}{1.314504in}}%
\pgfpathlineto{\pgfqpoint{5.936670in}{1.248643in}}%
\pgfpathlineto{\pgfqpoint{5.956942in}{1.220222in}}%
\pgfpathlineto{\pgfqpoint{5.977214in}{1.107752in}}%
\pgfpathlineto{\pgfqpoint{5.997486in}{1.093381in}}%
\pgfpathlineto{\pgfqpoint{6.017758in}{1.066968in}}%
\pgfpathlineto{\pgfqpoint{6.038030in}{1.018336in}}%
\pgfpathlineto{\pgfqpoint{6.058302in}{0.903274in}}%
\pgfpathlineto{\pgfqpoint{6.078574in}{0.743343in}}%
\pgfpathlineto{\pgfqpoint{6.098847in}{0.561161in}}%
\pgfpathlineto{\pgfqpoint{6.098847in}{0.561161in}}%
\pgfusepath{stroke}%
\end{pgfscope}%
\begin{pgfscope}%
\pgfsetrectcap%
\pgfsetmiterjoin%
\pgfsetlinewidth{0.803000pt}%
\definecolor{currentstroke}{rgb}{0.000000,0.000000,0.000000}%
\pgfsetstrokecolor{currentstroke}%
\pgfsetdash{}{0pt}%
\pgfpathmoveto{\pgfqpoint{0.649696in}{0.456793in}}%
\pgfpathlineto{\pgfqpoint{0.649696in}{2.163512in}}%
\pgfusepath{stroke}%
\end{pgfscope}%
\begin{pgfscope}%
\pgfsetrectcap%
\pgfsetmiterjoin%
\pgfsetlinewidth{0.803000pt}%
\definecolor{currentstroke}{rgb}{0.000000,0.000000,0.000000}%
\pgfsetstrokecolor{currentstroke}%
\pgfsetdash{}{0pt}%
\pgfpathmoveto{\pgfqpoint{6.358330in}{0.456793in}}%
\pgfpathlineto{\pgfqpoint{6.358330in}{2.163512in}}%
\pgfusepath{stroke}%
\end{pgfscope}%
\begin{pgfscope}%
\pgfsetrectcap%
\pgfsetmiterjoin%
\pgfsetlinewidth{0.803000pt}%
\definecolor{currentstroke}{rgb}{0.000000,0.000000,0.000000}%
\pgfsetstrokecolor{currentstroke}%
\pgfsetdash{}{0pt}%
\pgfpathmoveto{\pgfqpoint{0.649696in}{0.456793in}}%
\pgfpathlineto{\pgfqpoint{6.358330in}{0.456793in}}%
\pgfusepath{stroke}%
\end{pgfscope}%
\begin{pgfscope}%
\pgfsetrectcap%
\pgfsetmiterjoin%
\pgfsetlinewidth{0.803000pt}%
\definecolor{currentstroke}{rgb}{0.000000,0.000000,0.000000}%
\pgfsetstrokecolor{currentstroke}%
\pgfsetdash{}{0pt}%
\pgfpathmoveto{\pgfqpoint{0.649696in}{2.163512in}}%
\pgfpathlineto{\pgfqpoint{6.358330in}{2.163512in}}%
\pgfusepath{stroke}%
\end{pgfscope}%
\end{pgfpicture}%
\makeatother%
\endgroup%

        \label{fig:sim-psd}
        \caption{The simulated PSD of a 1000 bit long message}
    \end{center}
\end{figure}

\subsection{Power Spectral Density of 4-PAM (Nyquist Pulse Shaping)}
The for the design of a Nyquist pulse a roll off factor of $\alpha = 1$ was used as there is adequate bandwidth
for the signal and the high roll off causes the tail oscillations of the pulse to fade more quickly, reducing the
chance of intersymbol interference (ISI)

The $S_x$ value is the same as above, using a root raised cosine pulse shaping the pulse is:

\begin{equation}
    \label{eqn:pf-w-niquist}
    P(f) =
    \begin{dcases}
        \frac{1}{4W} [1+\textrm{cos}(\frac{\pi f}{2W})] \quad \textrm{when} \quad 0<|f|<2W \\
        0 \quad \quad \textrm{when} |f| \geq 2W
    \end{dcases}
\end{equation}

Therefore $|P(f)|^2$ is,

\begin{equation}
    \label{eqn-pf2-nyquist}
    |P(f)|^2 =
    \begin{dcases}
        \frac{1}{16W^2}[1+\textrm{cos}(\frac{\pi f}{2W})]^2 \quad \textrm{when} \quad 0<|f|<2W \\
        0 \quad \textrm{when} \quad |f| \geq 2W
    \end{dcases}
\end{equation}

As $2W = \frac{1}{T_s}$,

\begin{equation}
    \label{eqn-pf2-nyquist-ts}
    |P(f)|^2 =
    \begin{dcases}
        \frac{T_s^2}{4}[1+\textrm{cos}(\pi f T_s)]^2 \quad \textrm{when} \quad 0<|f|<\frac{1}{T_s} \\
        0 \quad \textrm{when} \quad |f| \geq \frac{1}{T_s}
    \end{dcases}
\end{equation}

Therefore the power spectral density is,

\begin{equation}
    \label{eqn:psd-nyquist}
    S_y(f) = \frac{5T_s}{4}[1+\textrm{cos}(\pi f T_s)]^2
\end{equation}

As shown is the PSD in Figure \ref{fig:fig-nyquist} there is no signal energy outside of the band limits
given in the specifications, this means that the signal will not be distorted when bandlimited. This shows
that using a root raised cosine pulse shape for a bandlimited channel will have better performance than
the rectangular pulse shaping in Figure \ref{fig:psd-rect}

\begin{figure}[h]
    \begin{center}
        \begin{tikzpicture}
    \begin{axis}[
        axis lines = left,
        xlabel = $f$,
        ylabel = {$S_y$},
    ]
    \addplot [
        domain=-25000000:25000000,
        samples=2000,
        color=red,
    ]
    {(5*0.000002)/4*(1 + cos(pi*x*0.000002))^2};
    \addlegendentry{$S_y$}
    \end{axis}
\end{tikzpicture}

        \caption{The PSD of a RRCOS pulse shaped 4-PAM Signal}
        \label{fig:psd-nyquist}
    \end{center}
\end{figure}

\subsection{Pulse shaping a signal with a Nyquist pulse}

Figure ~\ref{fig:ny-pulse-shape} shows a raised cosine pulse shaped signal carrying the data `0010110111', this corresponds with the 4-PAM
levels: [-3, 1, 3, -1, 3].

\begin{figure}[H]
    \begin{center}
        %% Creator: Matplotlib, PGF backend
%%
%% To include the figure in your LaTeX document, write
%%   \input{<filename>.pgf}
%%
%% Make sure the required packages are loaded in your preamble
%%   \usepackage{pgf}
%%
%% and, on pdftex
%%   \usepackage[utf8]{inputenc}\DeclareUnicodeCharacter{2212}{-}
%%
%% or, on luatex and xetex
%%   \usepackage{unicode-math}
%%
%% Figures using additional raster images can only be included by \input if
%% they are in the same directory as the main LaTeX file. For loading figures
%% from other directories you can use the `import` package
%%   \usepackage{import}
%%
%% and then include the figures with
%%   \import{<path to file>}{<filename>.pgf}
%%
%% Matplotlib used the following preamble
%%
\begingroup%
\makeatletter%
\begin{pgfpicture}%
\pgfpathrectangle{\pgfpointorigin}{\pgfqpoint{6.400000in}{4.800000in}}%
\pgfusepath{use as bounding box, clip}%
\begin{pgfscope}%
\pgfsetbuttcap%
\pgfsetmiterjoin%
\definecolor{currentfill}{rgb}{1.000000,1.000000,1.000000}%
\pgfsetfillcolor{currentfill}%
\pgfsetlinewidth{0.000000pt}%
\definecolor{currentstroke}{rgb}{1.000000,1.000000,1.000000}%
\pgfsetstrokecolor{currentstroke}%
\pgfsetdash{}{0pt}%
\pgfpathmoveto{\pgfqpoint{0.000000in}{0.000000in}}%
\pgfpathlineto{\pgfqpoint{6.400000in}{0.000000in}}%
\pgfpathlineto{\pgfqpoint{6.400000in}{4.800000in}}%
\pgfpathlineto{\pgfqpoint{0.000000in}{4.800000in}}%
\pgfpathclose%
\pgfusepath{fill}%
\end{pgfscope}%
\begin{pgfscope}%
\pgfsetbuttcap%
\pgfsetmiterjoin%
\definecolor{currentfill}{rgb}{1.000000,1.000000,1.000000}%
\pgfsetfillcolor{currentfill}%
\pgfsetlinewidth{0.000000pt}%
\definecolor{currentstroke}{rgb}{0.000000,0.000000,0.000000}%
\pgfsetstrokecolor{currentstroke}%
\pgfsetstrokeopacity{0.000000}%
\pgfsetdash{}{0pt}%
\pgfpathmoveto{\pgfqpoint{0.800000in}{0.528000in}}%
\pgfpathlineto{\pgfqpoint{5.760000in}{0.528000in}}%
\pgfpathlineto{\pgfqpoint{5.760000in}{4.224000in}}%
\pgfpathlineto{\pgfqpoint{0.800000in}{4.224000in}}%
\pgfpathclose%
\pgfusepath{fill}%
\end{pgfscope}%
\begin{pgfscope}%
\pgfsetbuttcap%
\pgfsetroundjoin%
\definecolor{currentfill}{rgb}{0.000000,0.000000,0.000000}%
\pgfsetfillcolor{currentfill}%
\pgfsetlinewidth{0.803000pt}%
\definecolor{currentstroke}{rgb}{0.000000,0.000000,0.000000}%
\pgfsetstrokecolor{currentstroke}%
\pgfsetdash{}{0pt}%
\pgfsys@defobject{currentmarker}{\pgfqpoint{0.000000in}{-0.048611in}}{\pgfqpoint{0.000000in}{0.000000in}}{%
\pgfpathmoveto{\pgfqpoint{0.000000in}{0.000000in}}%
\pgfpathlineto{\pgfqpoint{0.000000in}{-0.048611in}}%
\pgfusepath{stroke,fill}%
}%
\begin{pgfscope}%
\pgfsys@transformshift{1.025455in}{0.528000in}%
\pgfsys@useobject{currentmarker}{}%
\end{pgfscope}%
\end{pgfscope}%
\begin{pgfscope}%
\definecolor{textcolor}{rgb}{0.000000,0.000000,0.000000}%
\pgfsetstrokecolor{textcolor}%
\pgfsetfillcolor{textcolor}%
\pgftext[x=1.025455in,y=0.430778in,,top]{\color{textcolor}\rmfamily\fontsize{10.000000}{12.000000}\selectfont \(\displaystyle 0.0000\)}%
\end{pgfscope}%
\begin{pgfscope}%
\pgfsetbuttcap%
\pgfsetroundjoin%
\definecolor{currentfill}{rgb}{0.000000,0.000000,0.000000}%
\pgfsetfillcolor{currentfill}%
\pgfsetlinewidth{0.803000pt}%
\definecolor{currentstroke}{rgb}{0.000000,0.000000,0.000000}%
\pgfsetstrokecolor{currentstroke}%
\pgfsetdash{}{0pt}%
\pgfsys@defobject{currentmarker}{\pgfqpoint{0.000000in}{-0.048611in}}{\pgfqpoint{0.000000in}{0.000000in}}{%
\pgfpathmoveto{\pgfqpoint{0.000000in}{0.000000in}}%
\pgfpathlineto{\pgfqpoint{0.000000in}{-0.048611in}}%
\pgfusepath{stroke,fill}%
}%
\begin{pgfscope}%
\pgfsys@transformshift{1.596226in}{0.528000in}%
\pgfsys@useobject{currentmarker}{}%
\end{pgfscope}%
\end{pgfscope}%
\begin{pgfscope}%
\definecolor{textcolor}{rgb}{0.000000,0.000000,0.000000}%
\pgfsetstrokecolor{textcolor}%
\pgfsetfillcolor{textcolor}%
\pgftext[x=1.596226in,y=0.430778in,,top]{\color{textcolor}\rmfamily\fontsize{10.000000}{12.000000}\selectfont \(\displaystyle 0.0025\)}%
\end{pgfscope}%
\begin{pgfscope}%
\pgfsetbuttcap%
\pgfsetroundjoin%
\definecolor{currentfill}{rgb}{0.000000,0.000000,0.000000}%
\pgfsetfillcolor{currentfill}%
\pgfsetlinewidth{0.803000pt}%
\definecolor{currentstroke}{rgb}{0.000000,0.000000,0.000000}%
\pgfsetstrokecolor{currentstroke}%
\pgfsetdash{}{0pt}%
\pgfsys@defobject{currentmarker}{\pgfqpoint{0.000000in}{-0.048611in}}{\pgfqpoint{0.000000in}{0.000000in}}{%
\pgfpathmoveto{\pgfqpoint{0.000000in}{0.000000in}}%
\pgfpathlineto{\pgfqpoint{0.000000in}{-0.048611in}}%
\pgfusepath{stroke,fill}%
}%
\begin{pgfscope}%
\pgfsys@transformshift{2.166996in}{0.528000in}%
\pgfsys@useobject{currentmarker}{}%
\end{pgfscope}%
\end{pgfscope}%
\begin{pgfscope}%
\definecolor{textcolor}{rgb}{0.000000,0.000000,0.000000}%
\pgfsetstrokecolor{textcolor}%
\pgfsetfillcolor{textcolor}%
\pgftext[x=2.166996in,y=0.430778in,,top]{\color{textcolor}\rmfamily\fontsize{10.000000}{12.000000}\selectfont \(\displaystyle 0.0050\)}%
\end{pgfscope}%
\begin{pgfscope}%
\pgfsetbuttcap%
\pgfsetroundjoin%
\definecolor{currentfill}{rgb}{0.000000,0.000000,0.000000}%
\pgfsetfillcolor{currentfill}%
\pgfsetlinewidth{0.803000pt}%
\definecolor{currentstroke}{rgb}{0.000000,0.000000,0.000000}%
\pgfsetstrokecolor{currentstroke}%
\pgfsetdash{}{0pt}%
\pgfsys@defobject{currentmarker}{\pgfqpoint{0.000000in}{-0.048611in}}{\pgfqpoint{0.000000in}{0.000000in}}{%
\pgfpathmoveto{\pgfqpoint{0.000000in}{0.000000in}}%
\pgfpathlineto{\pgfqpoint{0.000000in}{-0.048611in}}%
\pgfusepath{stroke,fill}%
}%
\begin{pgfscope}%
\pgfsys@transformshift{2.737767in}{0.528000in}%
\pgfsys@useobject{currentmarker}{}%
\end{pgfscope}%
\end{pgfscope}%
\begin{pgfscope}%
\definecolor{textcolor}{rgb}{0.000000,0.000000,0.000000}%
\pgfsetstrokecolor{textcolor}%
\pgfsetfillcolor{textcolor}%
\pgftext[x=2.737767in,y=0.430778in,,top]{\color{textcolor}\rmfamily\fontsize{10.000000}{12.000000}\selectfont \(\displaystyle 0.0075\)}%
\end{pgfscope}%
\begin{pgfscope}%
\pgfsetbuttcap%
\pgfsetroundjoin%
\definecolor{currentfill}{rgb}{0.000000,0.000000,0.000000}%
\pgfsetfillcolor{currentfill}%
\pgfsetlinewidth{0.803000pt}%
\definecolor{currentstroke}{rgb}{0.000000,0.000000,0.000000}%
\pgfsetstrokecolor{currentstroke}%
\pgfsetdash{}{0pt}%
\pgfsys@defobject{currentmarker}{\pgfqpoint{0.000000in}{-0.048611in}}{\pgfqpoint{0.000000in}{0.000000in}}{%
\pgfpathmoveto{\pgfqpoint{0.000000in}{0.000000in}}%
\pgfpathlineto{\pgfqpoint{0.000000in}{-0.048611in}}%
\pgfusepath{stroke,fill}%
}%
\begin{pgfscope}%
\pgfsys@transformshift{3.308538in}{0.528000in}%
\pgfsys@useobject{currentmarker}{}%
\end{pgfscope}%
\end{pgfscope}%
\begin{pgfscope}%
\definecolor{textcolor}{rgb}{0.000000,0.000000,0.000000}%
\pgfsetstrokecolor{textcolor}%
\pgfsetfillcolor{textcolor}%
\pgftext[x=3.308538in,y=0.430778in,,top]{\color{textcolor}\rmfamily\fontsize{10.000000}{12.000000}\selectfont \(\displaystyle 0.0100\)}%
\end{pgfscope}%
\begin{pgfscope}%
\pgfsetbuttcap%
\pgfsetroundjoin%
\definecolor{currentfill}{rgb}{0.000000,0.000000,0.000000}%
\pgfsetfillcolor{currentfill}%
\pgfsetlinewidth{0.803000pt}%
\definecolor{currentstroke}{rgb}{0.000000,0.000000,0.000000}%
\pgfsetstrokecolor{currentstroke}%
\pgfsetdash{}{0pt}%
\pgfsys@defobject{currentmarker}{\pgfqpoint{0.000000in}{-0.048611in}}{\pgfqpoint{0.000000in}{0.000000in}}{%
\pgfpathmoveto{\pgfqpoint{0.000000in}{0.000000in}}%
\pgfpathlineto{\pgfqpoint{0.000000in}{-0.048611in}}%
\pgfusepath{stroke,fill}%
}%
\begin{pgfscope}%
\pgfsys@transformshift{3.879309in}{0.528000in}%
\pgfsys@useobject{currentmarker}{}%
\end{pgfscope}%
\end{pgfscope}%
\begin{pgfscope}%
\definecolor{textcolor}{rgb}{0.000000,0.000000,0.000000}%
\pgfsetstrokecolor{textcolor}%
\pgfsetfillcolor{textcolor}%
\pgftext[x=3.879309in,y=0.430778in,,top]{\color{textcolor}\rmfamily\fontsize{10.000000}{12.000000}\selectfont \(\displaystyle 0.0125\)}%
\end{pgfscope}%
\begin{pgfscope}%
\pgfsetbuttcap%
\pgfsetroundjoin%
\definecolor{currentfill}{rgb}{0.000000,0.000000,0.000000}%
\pgfsetfillcolor{currentfill}%
\pgfsetlinewidth{0.803000pt}%
\definecolor{currentstroke}{rgb}{0.000000,0.000000,0.000000}%
\pgfsetstrokecolor{currentstroke}%
\pgfsetdash{}{0pt}%
\pgfsys@defobject{currentmarker}{\pgfqpoint{0.000000in}{-0.048611in}}{\pgfqpoint{0.000000in}{0.000000in}}{%
\pgfpathmoveto{\pgfqpoint{0.000000in}{0.000000in}}%
\pgfpathlineto{\pgfqpoint{0.000000in}{-0.048611in}}%
\pgfusepath{stroke,fill}%
}%
\begin{pgfscope}%
\pgfsys@transformshift{4.450080in}{0.528000in}%
\pgfsys@useobject{currentmarker}{}%
\end{pgfscope}%
\end{pgfscope}%
\begin{pgfscope}%
\definecolor{textcolor}{rgb}{0.000000,0.000000,0.000000}%
\pgfsetstrokecolor{textcolor}%
\pgfsetfillcolor{textcolor}%
\pgftext[x=4.450080in,y=0.430778in,,top]{\color{textcolor}\rmfamily\fontsize{10.000000}{12.000000}\selectfont \(\displaystyle 0.0150\)}%
\end{pgfscope}%
\begin{pgfscope}%
\pgfsetbuttcap%
\pgfsetroundjoin%
\definecolor{currentfill}{rgb}{0.000000,0.000000,0.000000}%
\pgfsetfillcolor{currentfill}%
\pgfsetlinewidth{0.803000pt}%
\definecolor{currentstroke}{rgb}{0.000000,0.000000,0.000000}%
\pgfsetstrokecolor{currentstroke}%
\pgfsetdash{}{0pt}%
\pgfsys@defobject{currentmarker}{\pgfqpoint{0.000000in}{-0.048611in}}{\pgfqpoint{0.000000in}{0.000000in}}{%
\pgfpathmoveto{\pgfqpoint{0.000000in}{0.000000in}}%
\pgfpathlineto{\pgfqpoint{0.000000in}{-0.048611in}}%
\pgfusepath{stroke,fill}%
}%
\begin{pgfscope}%
\pgfsys@transformshift{5.020851in}{0.528000in}%
\pgfsys@useobject{currentmarker}{}%
\end{pgfscope}%
\end{pgfscope}%
\begin{pgfscope}%
\definecolor{textcolor}{rgb}{0.000000,0.000000,0.000000}%
\pgfsetstrokecolor{textcolor}%
\pgfsetfillcolor{textcolor}%
\pgftext[x=5.020851in,y=0.430778in,,top]{\color{textcolor}\rmfamily\fontsize{10.000000}{12.000000}\selectfont \(\displaystyle 0.0175\)}%
\end{pgfscope}%
\begin{pgfscope}%
\pgfsetbuttcap%
\pgfsetroundjoin%
\definecolor{currentfill}{rgb}{0.000000,0.000000,0.000000}%
\pgfsetfillcolor{currentfill}%
\pgfsetlinewidth{0.803000pt}%
\definecolor{currentstroke}{rgb}{0.000000,0.000000,0.000000}%
\pgfsetstrokecolor{currentstroke}%
\pgfsetdash{}{0pt}%
\pgfsys@defobject{currentmarker}{\pgfqpoint{0.000000in}{-0.048611in}}{\pgfqpoint{0.000000in}{0.000000in}}{%
\pgfpathmoveto{\pgfqpoint{0.000000in}{0.000000in}}%
\pgfpathlineto{\pgfqpoint{0.000000in}{-0.048611in}}%
\pgfusepath{stroke,fill}%
}%
\begin{pgfscope}%
\pgfsys@transformshift{5.591622in}{0.528000in}%
\pgfsys@useobject{currentmarker}{}%
\end{pgfscope}%
\end{pgfscope}%
\begin{pgfscope}%
\definecolor{textcolor}{rgb}{0.000000,0.000000,0.000000}%
\pgfsetstrokecolor{textcolor}%
\pgfsetfillcolor{textcolor}%
\pgftext[x=5.591622in,y=0.430778in,,top]{\color{textcolor}\rmfamily\fontsize{10.000000}{12.000000}\selectfont \(\displaystyle 0.0200\)}%
\end{pgfscope}%
\begin{pgfscope}%
\definecolor{textcolor}{rgb}{0.000000,0.000000,0.000000}%
\pgfsetstrokecolor{textcolor}%
\pgfsetfillcolor{textcolor}%
\pgftext[x=3.280000in,y=0.251766in,,top]{\color{textcolor}\rmfamily\fontsize{10.000000}{12.000000}\selectfont Time (s)}%
\end{pgfscope}%
\begin{pgfscope}%
\pgfsetbuttcap%
\pgfsetroundjoin%
\definecolor{currentfill}{rgb}{0.000000,0.000000,0.000000}%
\pgfsetfillcolor{currentfill}%
\pgfsetlinewidth{0.803000pt}%
\definecolor{currentstroke}{rgb}{0.000000,0.000000,0.000000}%
\pgfsetstrokecolor{currentstroke}%
\pgfsetdash{}{0pt}%
\pgfsys@defobject{currentmarker}{\pgfqpoint{-0.048611in}{0.000000in}}{\pgfqpoint{0.000000in}{0.000000in}}{%
\pgfpathmoveto{\pgfqpoint{0.000000in}{0.000000in}}%
\pgfpathlineto{\pgfqpoint{-0.048611in}{0.000000in}}%
\pgfusepath{stroke,fill}%
}%
\begin{pgfscope}%
\pgfsys@transformshift{0.800000in}{0.696000in}%
\pgfsys@useobject{currentmarker}{}%
\end{pgfscope}%
\end{pgfscope}%
\begin{pgfscope}%
\definecolor{textcolor}{rgb}{0.000000,0.000000,0.000000}%
\pgfsetstrokecolor{textcolor}%
\pgfsetfillcolor{textcolor}%
\pgftext[x=0.525308in, y=0.647775in, left, base]{\color{textcolor}\rmfamily\fontsize{10.000000}{12.000000}\selectfont \(\displaystyle -3\)}%
\end{pgfscope}%
\begin{pgfscope}%
\pgfsetbuttcap%
\pgfsetroundjoin%
\definecolor{currentfill}{rgb}{0.000000,0.000000,0.000000}%
\pgfsetfillcolor{currentfill}%
\pgfsetlinewidth{0.803000pt}%
\definecolor{currentstroke}{rgb}{0.000000,0.000000,0.000000}%
\pgfsetstrokecolor{currentstroke}%
\pgfsetdash{}{0pt}%
\pgfsys@defobject{currentmarker}{\pgfqpoint{-0.048611in}{0.000000in}}{\pgfqpoint{0.000000in}{0.000000in}}{%
\pgfpathmoveto{\pgfqpoint{0.000000in}{0.000000in}}%
\pgfpathlineto{\pgfqpoint{-0.048611in}{0.000000in}}%
\pgfusepath{stroke,fill}%
}%
\begin{pgfscope}%
\pgfsys@transformshift{0.800000in}{1.256000in}%
\pgfsys@useobject{currentmarker}{}%
\end{pgfscope}%
\end{pgfscope}%
\begin{pgfscope}%
\definecolor{textcolor}{rgb}{0.000000,0.000000,0.000000}%
\pgfsetstrokecolor{textcolor}%
\pgfsetfillcolor{textcolor}%
\pgftext[x=0.525308in, y=1.207775in, left, base]{\color{textcolor}\rmfamily\fontsize{10.000000}{12.000000}\selectfont \(\displaystyle -2\)}%
\end{pgfscope}%
\begin{pgfscope}%
\pgfsetbuttcap%
\pgfsetroundjoin%
\definecolor{currentfill}{rgb}{0.000000,0.000000,0.000000}%
\pgfsetfillcolor{currentfill}%
\pgfsetlinewidth{0.803000pt}%
\definecolor{currentstroke}{rgb}{0.000000,0.000000,0.000000}%
\pgfsetstrokecolor{currentstroke}%
\pgfsetdash{}{0pt}%
\pgfsys@defobject{currentmarker}{\pgfqpoint{-0.048611in}{0.000000in}}{\pgfqpoint{0.000000in}{0.000000in}}{%
\pgfpathmoveto{\pgfqpoint{0.000000in}{0.000000in}}%
\pgfpathlineto{\pgfqpoint{-0.048611in}{0.000000in}}%
\pgfusepath{stroke,fill}%
}%
\begin{pgfscope}%
\pgfsys@transformshift{0.800000in}{1.816000in}%
\pgfsys@useobject{currentmarker}{}%
\end{pgfscope}%
\end{pgfscope}%
\begin{pgfscope}%
\definecolor{textcolor}{rgb}{0.000000,0.000000,0.000000}%
\pgfsetstrokecolor{textcolor}%
\pgfsetfillcolor{textcolor}%
\pgftext[x=0.525308in, y=1.767775in, left, base]{\color{textcolor}\rmfamily\fontsize{10.000000}{12.000000}\selectfont \(\displaystyle -1\)}%
\end{pgfscope}%
\begin{pgfscope}%
\pgfsetbuttcap%
\pgfsetroundjoin%
\definecolor{currentfill}{rgb}{0.000000,0.000000,0.000000}%
\pgfsetfillcolor{currentfill}%
\pgfsetlinewidth{0.803000pt}%
\definecolor{currentstroke}{rgb}{0.000000,0.000000,0.000000}%
\pgfsetstrokecolor{currentstroke}%
\pgfsetdash{}{0pt}%
\pgfsys@defobject{currentmarker}{\pgfqpoint{-0.048611in}{0.000000in}}{\pgfqpoint{0.000000in}{0.000000in}}{%
\pgfpathmoveto{\pgfqpoint{0.000000in}{0.000000in}}%
\pgfpathlineto{\pgfqpoint{-0.048611in}{0.000000in}}%
\pgfusepath{stroke,fill}%
}%
\begin{pgfscope}%
\pgfsys@transformshift{0.800000in}{2.376000in}%
\pgfsys@useobject{currentmarker}{}%
\end{pgfscope}%
\end{pgfscope}%
\begin{pgfscope}%
\definecolor{textcolor}{rgb}{0.000000,0.000000,0.000000}%
\pgfsetstrokecolor{textcolor}%
\pgfsetfillcolor{textcolor}%
\pgftext[x=0.633333in, y=2.327775in, left, base]{\color{textcolor}\rmfamily\fontsize{10.000000}{12.000000}\selectfont \(\displaystyle 0\)}%
\end{pgfscope}%
\begin{pgfscope}%
\pgfsetbuttcap%
\pgfsetroundjoin%
\definecolor{currentfill}{rgb}{0.000000,0.000000,0.000000}%
\pgfsetfillcolor{currentfill}%
\pgfsetlinewidth{0.803000pt}%
\definecolor{currentstroke}{rgb}{0.000000,0.000000,0.000000}%
\pgfsetstrokecolor{currentstroke}%
\pgfsetdash{}{0pt}%
\pgfsys@defobject{currentmarker}{\pgfqpoint{-0.048611in}{0.000000in}}{\pgfqpoint{0.000000in}{0.000000in}}{%
\pgfpathmoveto{\pgfqpoint{0.000000in}{0.000000in}}%
\pgfpathlineto{\pgfqpoint{-0.048611in}{0.000000in}}%
\pgfusepath{stroke,fill}%
}%
\begin{pgfscope}%
\pgfsys@transformshift{0.800000in}{2.936000in}%
\pgfsys@useobject{currentmarker}{}%
\end{pgfscope}%
\end{pgfscope}%
\begin{pgfscope}%
\definecolor{textcolor}{rgb}{0.000000,0.000000,0.000000}%
\pgfsetstrokecolor{textcolor}%
\pgfsetfillcolor{textcolor}%
\pgftext[x=0.633333in, y=2.887775in, left, base]{\color{textcolor}\rmfamily\fontsize{10.000000}{12.000000}\selectfont \(\displaystyle 1\)}%
\end{pgfscope}%
\begin{pgfscope}%
\pgfsetbuttcap%
\pgfsetroundjoin%
\definecolor{currentfill}{rgb}{0.000000,0.000000,0.000000}%
\pgfsetfillcolor{currentfill}%
\pgfsetlinewidth{0.803000pt}%
\definecolor{currentstroke}{rgb}{0.000000,0.000000,0.000000}%
\pgfsetstrokecolor{currentstroke}%
\pgfsetdash{}{0pt}%
\pgfsys@defobject{currentmarker}{\pgfqpoint{-0.048611in}{0.000000in}}{\pgfqpoint{0.000000in}{0.000000in}}{%
\pgfpathmoveto{\pgfqpoint{0.000000in}{0.000000in}}%
\pgfpathlineto{\pgfqpoint{-0.048611in}{0.000000in}}%
\pgfusepath{stroke,fill}%
}%
\begin{pgfscope}%
\pgfsys@transformshift{0.800000in}{3.496000in}%
\pgfsys@useobject{currentmarker}{}%
\end{pgfscope}%
\end{pgfscope}%
\begin{pgfscope}%
\definecolor{textcolor}{rgb}{0.000000,0.000000,0.000000}%
\pgfsetstrokecolor{textcolor}%
\pgfsetfillcolor{textcolor}%
\pgftext[x=0.633333in, y=3.447775in, left, base]{\color{textcolor}\rmfamily\fontsize{10.000000}{12.000000}\selectfont \(\displaystyle 2\)}%
\end{pgfscope}%
\begin{pgfscope}%
\pgfsetbuttcap%
\pgfsetroundjoin%
\definecolor{currentfill}{rgb}{0.000000,0.000000,0.000000}%
\pgfsetfillcolor{currentfill}%
\pgfsetlinewidth{0.803000pt}%
\definecolor{currentstroke}{rgb}{0.000000,0.000000,0.000000}%
\pgfsetstrokecolor{currentstroke}%
\pgfsetdash{}{0pt}%
\pgfsys@defobject{currentmarker}{\pgfqpoint{-0.048611in}{0.000000in}}{\pgfqpoint{0.000000in}{0.000000in}}{%
\pgfpathmoveto{\pgfqpoint{0.000000in}{0.000000in}}%
\pgfpathlineto{\pgfqpoint{-0.048611in}{0.000000in}}%
\pgfusepath{stroke,fill}%
}%
\begin{pgfscope}%
\pgfsys@transformshift{0.800000in}{4.056000in}%
\pgfsys@useobject{currentmarker}{}%
\end{pgfscope}%
\end{pgfscope}%
\begin{pgfscope}%
\definecolor{textcolor}{rgb}{0.000000,0.000000,0.000000}%
\pgfsetstrokecolor{textcolor}%
\pgfsetfillcolor{textcolor}%
\pgftext[x=0.633333in, y=4.007775in, left, base]{\color{textcolor}\rmfamily\fontsize{10.000000}{12.000000}\selectfont \(\displaystyle 3\)}%
\end{pgfscope}%
\begin{pgfscope}%
\definecolor{textcolor}{rgb}{0.000000,0.000000,0.000000}%
\pgfsetstrokecolor{textcolor}%
\pgfsetfillcolor{textcolor}%
\pgftext[x=0.469752in,y=2.376000in,,bottom,rotate=90.000000]{\color{textcolor}\rmfamily\fontsize{10.000000}{12.000000}\selectfont Amplitude}%
\end{pgfscope}%
\begin{pgfscope}%
\pgfpathrectangle{\pgfqpoint{0.800000in}{0.528000in}}{\pgfqpoint{4.960000in}{3.696000in}}%
\pgfusepath{clip}%
\pgfsetrectcap%
\pgfsetroundjoin%
\pgfsetlinewidth{1.505625pt}%
\definecolor{currentstroke}{rgb}{0.121569,0.466667,0.705882}%
\pgfsetstrokecolor{currentstroke}%
\pgfsetdash{}{0pt}%
\pgfpathmoveto{\pgfqpoint{1.025455in}{2.377844in}}%
\pgfpathlineto{\pgfqpoint{1.082532in}{2.381239in}}%
\pgfpathlineto{\pgfqpoint{1.139609in}{2.382225in}}%
\pgfpathlineto{\pgfqpoint{1.196686in}{2.379098in}}%
\pgfpathlineto{\pgfqpoint{1.253763in}{2.372232in}}%
\pgfpathlineto{\pgfqpoint{1.310840in}{2.364774in}}%
\pgfpathlineto{\pgfqpoint{1.367917in}{2.361912in}}%
\pgfpathlineto{\pgfqpoint{1.424994in}{2.368529in}}%
\pgfpathlineto{\pgfqpoint{1.482071in}{2.385182in}}%
\pgfpathlineto{\pgfqpoint{1.539148in}{2.406205in}}%
\pgfpathlineto{\pgfqpoint{1.596226in}{2.418751in}}%
\pgfpathlineto{\pgfqpoint{1.653303in}{2.402360in}}%
\pgfpathlineto{\pgfqpoint{1.710380in}{2.333870in}}%
\pgfpathlineto{\pgfqpoint{1.767457in}{2.194426in}}%
\pgfpathlineto{\pgfqpoint{1.824534in}{1.977258in}}%
\pgfpathlineto{\pgfqpoint{1.881611in}{1.693668in}}%
\pgfpathlineto{\pgfqpoint{1.938688in}{1.373007in}}%
\pgfpathlineto{\pgfqpoint{1.995765in}{1.062901in}}%
\pgfpathlineto{\pgfqpoint{2.052842in}{0.821767in}}%
\pgfpathlineto{\pgfqpoint{2.109919in}{0.700633in}}%
\pgfpathlineto{\pgfqpoint{2.166996in}{0.731091in}}%
\pgfpathlineto{\pgfqpoint{2.224074in}{0.915932in}}%
\pgfpathlineto{\pgfqpoint{2.281151in}{1.227358in}}%
\pgfpathlineto{\pgfqpoint{2.338228in}{1.613862in}}%
\pgfpathlineto{\pgfqpoint{2.395305in}{2.014624in}}%
\pgfpathlineto{\pgfqpoint{2.452382in}{2.374760in}}%
\pgfpathlineto{\pgfqpoint{2.509459in}{2.658955in}}%
\pgfpathlineto{\pgfqpoint{2.566536in}{2.860633in}}%
\pgfpathlineto{\pgfqpoint{2.623613in}{2.999903in}}%
\pgfpathlineto{\pgfqpoint{2.680690in}{3.113422in}}%
\pgfpathlineto{\pgfqpoint{2.737767in}{3.239096in}}%
\pgfpathlineto{\pgfqpoint{2.794845in}{3.400756in}}%
\pgfpathlineto{\pgfqpoint{2.851922in}{3.596108in}}%
\pgfpathlineto{\pgfqpoint{2.908999in}{3.799012in}}%
\pgfpathlineto{\pgfqpoint{2.966076in}{3.968329in}}%
\pgfpathlineto{\pgfqpoint{3.023153in}{4.055866in}}%
\pgfpathlineto{\pgfqpoint{3.080230in}{4.022411in}}%
\pgfpathlineto{\pgfqpoint{3.137307in}{3.849803in}}%
\pgfpathlineto{\pgfqpoint{3.194384in}{3.547407in}}%
\pgfpathlineto{\pgfqpoint{3.251461in}{3.151658in}}%
\pgfpathlineto{\pgfqpoint{3.308538in}{2.720887in}}%
\pgfpathlineto{\pgfqpoint{3.365616in}{2.319780in}}%
\pgfpathlineto{\pgfqpoint{3.422693in}{2.009209in}}%
\pgfpathlineto{\pgfqpoint{3.479770in}{1.838870in}}%
\pgfpathlineto{\pgfqpoint{3.536847in}{1.836581in}}%
\pgfpathlineto{\pgfqpoint{3.593924in}{2.003392in}}%
\pgfpathlineto{\pgfqpoint{3.651001in}{2.313338in}}%
\pgfpathlineto{\pgfqpoint{3.708078in}{2.717783in}}%
\pgfpathlineto{\pgfqpoint{3.765155in}{3.153080in}}%
\pgfpathlineto{\pgfqpoint{3.822232in}{3.552818in}}%
\pgfpathlineto{\pgfqpoint{3.879309in}{3.858520in}}%
\pgfpathlineto{\pgfqpoint{3.936386in}{4.028445in}}%
\pgfpathlineto{\pgfqpoint{3.993464in}{4.045172in}}%
\pgfpathlineto{\pgfqpoint{4.050541in}{3.917783in}}%
\pgfpathlineto{\pgfqpoint{4.107618in}{3.678621in}}%
\pgfpathlineto{\pgfqpoint{4.164695in}{3.375305in}}%
\pgfpathlineto{\pgfqpoint{4.221772in}{3.058332in}}%
\pgfpathlineto{\pgfqpoint{4.278849in}{2.774742in}}%
\pgfpathlineto{\pgfqpoint{4.335926in}{2.557574in}}%
\pgfpathlineto{\pgfqpoint{4.393003in}{2.418130in}}%
\pgfpathlineto{\pgfqpoint{4.450080in}{2.349640in}}%
\pgfpathlineto{\pgfqpoint{4.507158in}{2.333249in}}%
\pgfpathlineto{\pgfqpoint{4.564235in}{2.345795in}}%
\pgfpathlineto{\pgfqpoint{4.621312in}{2.366818in}}%
\pgfpathlineto{\pgfqpoint{4.678389in}{2.383471in}}%
\pgfpathlineto{\pgfqpoint{4.735466in}{2.390088in}}%
\pgfpathlineto{\pgfqpoint{4.792543in}{2.387226in}}%
\pgfpathlineto{\pgfqpoint{4.849620in}{2.379768in}}%
\pgfpathlineto{\pgfqpoint{4.906697in}{2.372902in}}%
\pgfpathlineto{\pgfqpoint{4.963774in}{2.369775in}}%
\pgfpathlineto{\pgfqpoint{5.020851in}{2.370761in}}%
\pgfpathlineto{\pgfqpoint{5.077929in}{2.374156in}}%
\pgfpathlineto{\pgfqpoint{5.135005in}{2.376000in}}%
\pgfpathlineto{\pgfqpoint{5.192083in}{2.376000in}}%
\pgfpathlineto{\pgfqpoint{5.249160in}{2.376000in}}%
\pgfpathlineto{\pgfqpoint{5.306237in}{2.376000in}}%
\pgfpathlineto{\pgfqpoint{5.363314in}{2.376000in}}%
\pgfpathlineto{\pgfqpoint{5.420391in}{2.376000in}}%
\pgfpathlineto{\pgfqpoint{5.477468in}{2.376000in}}%
\pgfpathlineto{\pgfqpoint{5.534545in}{2.376000in}}%
\pgfusepath{stroke}%
\end{pgfscope}%
\begin{pgfscope}%
\pgfpathrectangle{\pgfqpoint{0.800000in}{0.528000in}}{\pgfqpoint{4.960000in}{3.696000in}}%
\pgfusepath{clip}%
\pgfsetrectcap%
\pgfsetroundjoin%
\pgfsetlinewidth{1.505625pt}%
\definecolor{currentstroke}{rgb}{0.000000,0.000000,1.000000}%
\pgfsetstrokecolor{currentstroke}%
\pgfsetdash{}{0pt}%
\pgfpathmoveto{\pgfqpoint{1.025455in}{2.377844in}}%
\pgfpathlineto{\pgfqpoint{1.082532in}{2.381239in}}%
\pgfpathlineto{\pgfqpoint{1.139609in}{2.382225in}}%
\pgfpathlineto{\pgfqpoint{1.196686in}{2.379098in}}%
\pgfpathlineto{\pgfqpoint{1.253763in}{2.372232in}}%
\pgfpathlineto{\pgfqpoint{1.310840in}{2.364774in}}%
\pgfpathlineto{\pgfqpoint{1.367917in}{2.361912in}}%
\pgfpathlineto{\pgfqpoint{1.424994in}{2.368529in}}%
\pgfpathlineto{\pgfqpoint{1.482071in}{2.385182in}}%
\pgfpathlineto{\pgfqpoint{1.539148in}{2.406205in}}%
\pgfpathlineto{\pgfqpoint{1.596226in}{2.418751in}}%
\pgfpathlineto{\pgfqpoint{1.653303in}{2.402360in}}%
\pgfpathlineto{\pgfqpoint{1.710380in}{2.333870in}}%
\pgfpathlineto{\pgfqpoint{1.767457in}{2.194426in}}%
\pgfpathlineto{\pgfqpoint{1.824534in}{1.977258in}}%
\pgfpathlineto{\pgfqpoint{1.881611in}{1.693668in}}%
\pgfpathlineto{\pgfqpoint{1.938688in}{1.373007in}}%
\pgfpathlineto{\pgfqpoint{1.995765in}{1.062901in}}%
\pgfpathlineto{\pgfqpoint{2.052842in}{0.821767in}}%
\pgfpathlineto{\pgfqpoint{2.109919in}{0.700633in}}%
\pgfpathlineto{\pgfqpoint{2.166996in}{0.731091in}}%
\pgfpathlineto{\pgfqpoint{2.224074in}{0.915932in}}%
\pgfpathlineto{\pgfqpoint{2.281151in}{1.227358in}}%
\pgfpathlineto{\pgfqpoint{2.338228in}{1.613862in}}%
\pgfpathlineto{\pgfqpoint{2.395305in}{2.014624in}}%
\pgfpathlineto{\pgfqpoint{2.452382in}{2.374760in}}%
\pgfpathlineto{\pgfqpoint{2.509459in}{2.658955in}}%
\pgfpathlineto{\pgfqpoint{2.566536in}{2.860633in}}%
\pgfpathlineto{\pgfqpoint{2.623613in}{2.999903in}}%
\pgfpathlineto{\pgfqpoint{2.680690in}{3.113422in}}%
\pgfpathlineto{\pgfqpoint{2.737767in}{3.239096in}}%
\pgfpathlineto{\pgfqpoint{2.794845in}{3.400756in}}%
\pgfpathlineto{\pgfqpoint{2.851922in}{3.596108in}}%
\pgfpathlineto{\pgfqpoint{2.908999in}{3.799012in}}%
\pgfpathlineto{\pgfqpoint{2.966076in}{3.968329in}}%
\pgfpathlineto{\pgfqpoint{3.023153in}{4.055866in}}%
\pgfpathlineto{\pgfqpoint{3.080230in}{4.022411in}}%
\pgfpathlineto{\pgfqpoint{3.137307in}{3.849803in}}%
\pgfpathlineto{\pgfqpoint{3.194384in}{3.547407in}}%
\pgfpathlineto{\pgfqpoint{3.251461in}{3.151658in}}%
\pgfpathlineto{\pgfqpoint{3.308538in}{2.720887in}}%
\pgfpathlineto{\pgfqpoint{3.365616in}{2.319780in}}%
\pgfpathlineto{\pgfqpoint{3.422693in}{2.009209in}}%
\pgfpathlineto{\pgfqpoint{3.479770in}{1.838870in}}%
\pgfpathlineto{\pgfqpoint{3.536847in}{1.836581in}}%
\pgfpathlineto{\pgfqpoint{3.593924in}{2.003392in}}%
\pgfpathlineto{\pgfqpoint{3.651001in}{2.313338in}}%
\pgfpathlineto{\pgfqpoint{3.708078in}{2.717783in}}%
\pgfpathlineto{\pgfqpoint{3.765155in}{3.153080in}}%
\pgfpathlineto{\pgfqpoint{3.822232in}{3.552818in}}%
\pgfpathlineto{\pgfqpoint{3.879309in}{3.858520in}}%
\pgfpathlineto{\pgfqpoint{3.936386in}{4.028445in}}%
\pgfpathlineto{\pgfqpoint{3.993464in}{4.045172in}}%
\pgfpathlineto{\pgfqpoint{4.050541in}{3.917783in}}%
\pgfpathlineto{\pgfqpoint{4.107618in}{3.678621in}}%
\pgfpathlineto{\pgfqpoint{4.164695in}{3.375305in}}%
\pgfpathlineto{\pgfqpoint{4.221772in}{3.058332in}}%
\pgfpathlineto{\pgfqpoint{4.278849in}{2.774742in}}%
\pgfpathlineto{\pgfqpoint{4.335926in}{2.557574in}}%
\pgfpathlineto{\pgfqpoint{4.393003in}{2.418130in}}%
\pgfpathlineto{\pgfqpoint{4.450080in}{2.349640in}}%
\pgfpathlineto{\pgfqpoint{4.507158in}{2.333249in}}%
\pgfpathlineto{\pgfqpoint{4.564235in}{2.345795in}}%
\pgfpathlineto{\pgfqpoint{4.621312in}{2.366818in}}%
\pgfpathlineto{\pgfqpoint{4.678389in}{2.383471in}}%
\pgfpathlineto{\pgfqpoint{4.735466in}{2.390088in}}%
\pgfpathlineto{\pgfqpoint{4.792543in}{2.387226in}}%
\pgfpathlineto{\pgfqpoint{4.849620in}{2.379768in}}%
\pgfpathlineto{\pgfqpoint{4.906697in}{2.372902in}}%
\pgfpathlineto{\pgfqpoint{4.963774in}{2.369775in}}%
\pgfpathlineto{\pgfqpoint{5.020851in}{2.370761in}}%
\pgfpathlineto{\pgfqpoint{5.077929in}{2.374156in}}%
\pgfpathlineto{\pgfqpoint{5.135005in}{2.376000in}}%
\pgfpathlineto{\pgfqpoint{5.192083in}{2.376000in}}%
\pgfpathlineto{\pgfqpoint{5.249160in}{2.376000in}}%
\pgfpathlineto{\pgfqpoint{5.306237in}{2.376000in}}%
\pgfpathlineto{\pgfqpoint{5.363314in}{2.376000in}}%
\pgfpathlineto{\pgfqpoint{5.420391in}{2.376000in}}%
\pgfpathlineto{\pgfqpoint{5.477468in}{2.376000in}}%
\pgfpathlineto{\pgfqpoint{5.534545in}{2.376000in}}%
\pgfusepath{stroke}%
\end{pgfscope}%
\begin{pgfscope}%
\pgfpathrectangle{\pgfqpoint{0.800000in}{0.528000in}}{\pgfqpoint{4.960000in}{3.696000in}}%
\pgfusepath{clip}%
\pgfsetbuttcap%
\pgfsetroundjoin%
\pgfsetlinewidth{1.505625pt}%
\definecolor{currentstroke}{rgb}{1.000000,0.000000,0.000000}%
\pgfsetstrokecolor{currentstroke}%
\pgfsetdash{{5.550000pt}{2.400000pt}}{0.000000pt}%
\pgfpathmoveto{\pgfqpoint{1.025455in}{2.376000in}}%
\pgfpathlineto{\pgfqpoint{1.082532in}{2.376000in}}%
\pgfpathlineto{\pgfqpoint{1.139609in}{2.376000in}}%
\pgfpathlineto{\pgfqpoint{1.196686in}{2.376000in}}%
\pgfpathlineto{\pgfqpoint{1.253763in}{2.376000in}}%
\pgfpathlineto{\pgfqpoint{1.310840in}{2.376000in}}%
\pgfpathlineto{\pgfqpoint{1.367917in}{2.376000in}}%
\pgfpathlineto{\pgfqpoint{1.424994in}{2.376000in}}%
\pgfpathlineto{\pgfqpoint{1.482071in}{2.376000in}}%
\pgfpathlineto{\pgfqpoint{1.539148in}{2.376000in}}%
\pgfpathlineto{\pgfqpoint{1.596226in}{2.376000in}}%
\pgfpathlineto{\pgfqpoint{1.653303in}{2.376000in}}%
\pgfpathlineto{\pgfqpoint{1.710380in}{2.376000in}}%
\pgfpathlineto{\pgfqpoint{1.767457in}{2.376000in}}%
\pgfpathlineto{\pgfqpoint{1.824534in}{2.376000in}}%
\pgfpathlineto{\pgfqpoint{1.881611in}{2.376000in}}%
\pgfpathlineto{\pgfqpoint{1.938688in}{2.376000in}}%
\pgfpathlineto{\pgfqpoint{1.995765in}{2.376000in}}%
\pgfpathlineto{\pgfqpoint{2.052842in}{2.376000in}}%
\pgfpathlineto{\pgfqpoint{2.109919in}{2.376000in}}%
\pgfpathlineto{\pgfqpoint{2.166996in}{0.696000in}}%
\pgfpathlineto{\pgfqpoint{2.224074in}{0.696000in}}%
\pgfpathlineto{\pgfqpoint{2.281151in}{0.696000in}}%
\pgfpathlineto{\pgfqpoint{2.338228in}{0.696000in}}%
\pgfpathlineto{\pgfqpoint{2.395305in}{0.696000in}}%
\pgfpathlineto{\pgfqpoint{2.452382in}{0.696000in}}%
\pgfpathlineto{\pgfqpoint{2.509459in}{0.696000in}}%
\pgfpathlineto{\pgfqpoint{2.566536in}{0.696000in}}%
\pgfpathlineto{\pgfqpoint{2.623613in}{2.936000in}}%
\pgfpathlineto{\pgfqpoint{2.680690in}{2.936000in}}%
\pgfpathlineto{\pgfqpoint{2.737767in}{2.936000in}}%
\pgfpathlineto{\pgfqpoint{2.794845in}{2.936000in}}%
\pgfpathlineto{\pgfqpoint{2.851922in}{2.936000in}}%
\pgfpathlineto{\pgfqpoint{2.908999in}{2.936000in}}%
\pgfpathlineto{\pgfqpoint{2.966076in}{2.936000in}}%
\pgfpathlineto{\pgfqpoint{3.023153in}{2.936000in}}%
\pgfpathlineto{\pgfqpoint{3.080230in}{4.056000in}}%
\pgfpathlineto{\pgfqpoint{3.137307in}{4.056000in}}%
\pgfpathlineto{\pgfqpoint{3.194384in}{4.056000in}}%
\pgfpathlineto{\pgfqpoint{3.251461in}{4.056000in}}%
\pgfpathlineto{\pgfqpoint{3.308538in}{4.056000in}}%
\pgfpathlineto{\pgfqpoint{3.365616in}{4.056000in}}%
\pgfpathlineto{\pgfqpoint{3.422693in}{4.056000in}}%
\pgfpathlineto{\pgfqpoint{3.479770in}{4.056000in}}%
\pgfpathlineto{\pgfqpoint{3.536847in}{1.816000in}}%
\pgfpathlineto{\pgfqpoint{3.593924in}{1.816000in}}%
\pgfpathlineto{\pgfqpoint{3.651001in}{1.816000in}}%
\pgfpathlineto{\pgfqpoint{3.708078in}{1.816000in}}%
\pgfpathlineto{\pgfqpoint{3.765155in}{1.816000in}}%
\pgfpathlineto{\pgfqpoint{3.822232in}{1.816000in}}%
\pgfpathlineto{\pgfqpoint{3.879309in}{1.816000in}}%
\pgfpathlineto{\pgfqpoint{3.936386in}{1.816000in}}%
\pgfpathlineto{\pgfqpoint{3.993464in}{4.056000in}}%
\pgfpathlineto{\pgfqpoint{4.050541in}{4.056000in}}%
\pgfpathlineto{\pgfqpoint{4.107618in}{4.056000in}}%
\pgfpathlineto{\pgfqpoint{4.164695in}{4.056000in}}%
\pgfpathlineto{\pgfqpoint{4.221772in}{4.056000in}}%
\pgfpathlineto{\pgfqpoint{4.278849in}{4.056000in}}%
\pgfpathlineto{\pgfqpoint{4.335926in}{4.056000in}}%
\pgfpathlineto{\pgfqpoint{4.393003in}{4.056000in}}%
\pgfpathlineto{\pgfqpoint{4.450080in}{2.376000in}}%
\pgfpathlineto{\pgfqpoint{4.507158in}{2.376000in}}%
\pgfpathlineto{\pgfqpoint{4.564235in}{2.376000in}}%
\pgfpathlineto{\pgfqpoint{4.621312in}{2.376000in}}%
\pgfpathlineto{\pgfqpoint{4.678389in}{2.376000in}}%
\pgfpathlineto{\pgfqpoint{4.735466in}{2.376000in}}%
\pgfpathlineto{\pgfqpoint{4.792543in}{2.376000in}}%
\pgfpathlineto{\pgfqpoint{4.849620in}{2.376000in}}%
\pgfpathlineto{\pgfqpoint{4.906697in}{2.376000in}}%
\pgfpathlineto{\pgfqpoint{4.963774in}{2.376000in}}%
\pgfpathlineto{\pgfqpoint{5.020851in}{2.376000in}}%
\pgfpathlineto{\pgfqpoint{5.077929in}{2.376000in}}%
\pgfpathlineto{\pgfqpoint{5.135005in}{2.376000in}}%
\pgfpathlineto{\pgfqpoint{5.192083in}{2.376000in}}%
\pgfpathlineto{\pgfqpoint{5.249160in}{2.376000in}}%
\pgfpathlineto{\pgfqpoint{5.306237in}{2.376000in}}%
\pgfpathlineto{\pgfqpoint{5.363314in}{2.376000in}}%
\pgfpathlineto{\pgfqpoint{5.420391in}{2.376000in}}%
\pgfpathlineto{\pgfqpoint{5.477468in}{2.376000in}}%
\pgfpathlineto{\pgfqpoint{5.534545in}{2.376000in}}%
\pgfusepath{stroke}%
\end{pgfscope}%
\begin{pgfscope}%
\pgfsetrectcap%
\pgfsetmiterjoin%
\pgfsetlinewidth{0.803000pt}%
\definecolor{currentstroke}{rgb}{0.000000,0.000000,0.000000}%
\pgfsetstrokecolor{currentstroke}%
\pgfsetdash{}{0pt}%
\pgfpathmoveto{\pgfqpoint{0.800000in}{0.528000in}}%
\pgfpathlineto{\pgfqpoint{0.800000in}{4.224000in}}%
\pgfusepath{stroke}%
\end{pgfscope}%
\begin{pgfscope}%
\pgfsetrectcap%
\pgfsetmiterjoin%
\pgfsetlinewidth{0.803000pt}%
\definecolor{currentstroke}{rgb}{0.000000,0.000000,0.000000}%
\pgfsetstrokecolor{currentstroke}%
\pgfsetdash{}{0pt}%
\pgfpathmoveto{\pgfqpoint{5.760000in}{0.528000in}}%
\pgfpathlineto{\pgfqpoint{5.760000in}{4.224000in}}%
\pgfusepath{stroke}%
\end{pgfscope}%
\begin{pgfscope}%
\pgfsetrectcap%
\pgfsetmiterjoin%
\pgfsetlinewidth{0.803000pt}%
\definecolor{currentstroke}{rgb}{0.000000,0.000000,0.000000}%
\pgfsetstrokecolor{currentstroke}%
\pgfsetdash{}{0pt}%
\pgfpathmoveto{\pgfqpoint{0.800000in}{0.528000in}}%
\pgfpathlineto{\pgfqpoint{5.760000in}{0.528000in}}%
\pgfusepath{stroke}%
\end{pgfscope}%
\begin{pgfscope}%
\pgfsetrectcap%
\pgfsetmiterjoin%
\pgfsetlinewidth{0.803000pt}%
\definecolor{currentstroke}{rgb}{0.000000,0.000000,0.000000}%
\pgfsetstrokecolor{currentstroke}%
\pgfsetdash{}{0pt}%
\pgfpathmoveto{\pgfqpoint{0.800000in}{4.224000in}}%
\pgfpathlineto{\pgfqpoint{5.760000in}{4.224000in}}%
\pgfusepath{stroke}%
\end{pgfscope}%
\begin{pgfscope}%
\definecolor{textcolor}{rgb}{0.000000,0.000000,0.000000}%
\pgfsetstrokecolor{textcolor}%
\pgfsetfillcolor{textcolor}%
\pgftext[x=3.280000in,y=4.307333in,,base]{\color{textcolor}\rmfamily\fontsize{12.000000}{14.400000}\selectfont Nyquist Pulse shaped Signal}%
\end{pgfscope}%
\end{pgfpicture}%
\makeatother%
\endgroup%

    \end{center}
    \label{fig:ny-pulse-shape}
    \caption{A Nyquist Pulse Shaped Signal carrying the data `0010110111'}
\end{figure}


    
